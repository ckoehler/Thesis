%\documentclass[letter,12pt,dissertation]{OUdissertation5} %for Ph.D. disseration style
\documentclass[letter,12pt,thesis]{OUdissertation} %for M.S. thesis style
\usepackage{graphicx} %allows .eps and .epsi graphics to be inserted
\usepackage{epic} %allows use of latex graphics
\usepackage{eepic} %allows use of latex graphics
\usepackage{amssymb}
\usepackage{amsfonts}
\usepackage{psfrag}
\usepackage{float}
\usepackage{listings}
\usepackage{siunitx}


% for hyperlinks
%\usepackage{hyperref}

%\floatstyle{boxed} 
%\restylefloat{figure}

%\usepackage{bar}
% Brian added this on July 15 2006 from suggestion of Di Jin:
\usepackage[centerfoot]{pageno}

%\usepackage{harvard} %use with agsm.bst, see near end
%\newcommand{\citet}[1]{\citeasnoun{#1}} %use this with harvard 
%\newcommand{\citep}[1]{\cite{#1}} %use this with harvard 

\usepackage{natbib} %with amermeteorsoc.bst
\usepackage{amsmath}
%% For AMS,citations should be of the form ``author year''  not ``author, year'':
\bibpunct{(}{)}{;}{a}{}{,}

\newcommand{\be}{\begin{equation}} %these definitions save typing
\newcommand{\ee}{\end{equation}}
\newcommand{\bea}{\begin{eqnarray}}
\newcommand{\eea}{\end{eqnarray}}
\newcommand{\pd}[2]{\frac{\partial#1}{\partial#2}}
\newcommand{\pdd}[3]{\frac{\partial^{2}#1}{\partial#2 \partial#3}}
\newcommand{\imgsize}{440 pt}
\newcommand{\imgsizes}{340 pt}


\begin{document}
\title{OPTIMAL WAVEFORM DESIGN FOR WEATHER RADARS}
\author{CHRISTOPH KOEHLER}
\depositdate{2011}
\majorfield{SCHOOL OF ELECTRICAL AND COMPUTER ENGINEERING}
\memberone{Dr. Robert D. Palmer, Chair}
\membertwo{Dr. Boonleng Cheong}
\memberthree{Dr. Mark Yeary}
%next two members are used in dissertation
\memberfour{Dr. Guifu Zhang}
%\memberfive{Dr. Strangelove}
%another member is possible, but you must hack OUdissertation.cls
%\membersix{Dr. Spock}
%%%%
\begin{preface}
%\prefacesection{Dedication} %Dedication must be without a pagenumber, see end of this file.
%must leave a blank line next to avoid bug:

\tableofcontents
%\listoftables
\listoffigures
\prefacesection{Abstract}
Most of the literature about radar waveforms has been done in the context of application other than weather.  This
thesis will focus the discussion on pulse compression for weather radar applications. First, a general overview of using
radar for weather applications is given, followed by a brief discussion of general waveform design, in particular the
different components of a waveform that can be manipulated for its design. Next, the concepts of pulse compression will
be presented and the need for pulse compression explained.  Then, various different waveforms, like rectangular pulses,
phase coded, linear frequency modulated, and non-linear frequency modulated signals will be examined in regards to their
ambiguity function, which serves as the basis for performance metrics.  Each will be quantified in respect to integrated
sidelobe level, peak sidelobe level, and range resolution. Further, amplitude tapering will be applied to determine its
performance benefits.  For the non-linear frequency modulated waveforms, the frequency sweep function is parameterized
and optimal values for weather applications are presented. Lastly, the effects of a system impulse response on the
signal will be discussed and their performance analyzed.
\end{preface}
\chapter{Introduction}
\label{intro}
There are numerous works available on waveform design, the most comprehensive of which being \cite{Levanon:2004}.  This
work covers the
most common waveforms used today, such as pulses, pulse trains, frequency and phase modulated signals, as well as
continuous wave signals, each accompanied by comprehensive discussion and examples.  \cite{Richards:2005} also discusses
radar waveforms with an emphasis on frequency modulation and pulse trains, but focuses more generally on radar signal
processing.  The work of \cite{Keeler:1999} has much information for weather radar applications, concentrating on filter
design in phased array radars.  \cite{Mudukutore:1998} discusses the simulation of pulse compression for weather radars
for phase coded signals and different filters, with an emphasis on sidelobe reduction.

\section{Using Radar for Observing the Atmosphere}
From hard target observation during WWII and the latter part of the 20th century, radar quickly became a potential way
to observe the atmosphere. As with all remote sensing applications, radar can give advanced warning and better insights
into what is happening in the atmosphere in places that are not accessible or difficult to observe using other means,
such as localized weather stations or weather balloons. This made dense networks of instruments unnecessary, drastically
lowering maintenance and operation expenses and increasing the area of observation.

The most basic pulsed radar is able to determine the reflectivity of the atmosphere by measuring the return signal of
the radar beam. Additionally, coherent radars can measure the Doppler shift of the return signal and use it to find the
radial velocity of the target, i.e., the velocity component of the target along the radar beam. This is normally
achieved with a stable local oscillator (STALO). Simplified, the STALO signal is split into an in-phase and a 90 degree
phase shifted signal, which are then mixed with the return signal and passed through a set of low-pass filters that
remove the undesired frequency components. The two outputs of the filters are called $I$, the in-phase component, and
$Q$, the quadrature component, respectively. It is necessary to measure the phase of the $I$ and $Q$ components over
several pulses to determine the Doppler frequency, due to the long decorrelation time of the signal.


The frequency of the radar, and thus its wavelength, contribute to the performance for weather observation, because some
wavelengths perform better than others, especially with respect to attenuation through hydrometeors \citep[p.
448]{Skolnik:2001}. Specifically, attenuation due to rain depends on the wavelength, as does attenuation in clouds, with
signals of shorter wavelengths being more strongly attenuated \citep[pp. 42-43]{Doviak:2006}.   The WSR-88D network
radars operate at S-band, more specifically between 2700 and 3000 MHz \citep[p. 47]{Doviak:2006}.  However, S-band is
highly desired for other applications, like cellular phones, so C-band or X-band radars are also used for observing
weather to avoid interference between different systems operating at the same frequency band. For radars of shorter
wavelengths, the effects of attenuation can be mitigated by, e.g., using techniques that are based on polarimetric
measurements (\cite{Ryzhkov:2008}, \cite{Brandes:2002}).  

One advantage of higher frequency radars is the possibility of a smaller antenna size
for the same angular resolution. Other components such as waveguides, amplifiers and support structure are also
smaller, so the whole system requires less space and is often less expensive to manufacture and easier to deploy.
Another beneficial cost factor is the increased availability of commercial off-the-shelf (COTS) components due to an
increase in consumer RF devices like wireless routers and other networking equipment. These components are produced
in high quantities and available at lower costs to the radar manufacturer. One example of taking advantage of COTS
components for Digital Array Radars is given in \cite{Tarran:2008}.

As technology advances, demands on radar performance increase as well. Finer resolution offers better detail, while dual
polarization radars provide even more data that can be used to better analyze the atmosphere \citep[p.
242]{Doviak:2006}.  These data are derived from the so-called \emph{backscattering covariance matrix}, which relates the
backscattered electric field to the incident electric field. With two orthogonal polarizations, additional parametric
variables are derived, e.g. $Z_{\mathrm{dr}}$, the differential reflectivity, $\rho_{\mathrm{hv}}$, the correlation
coefficient, or $\Phi_{\mathrm{dp}}$, the differential phase. They are mostly used for hydrometeor classification \citep{Ryzhkov:2008} and
Quantitative Precipitation Estimation (QPE) \citep{Brandes:2002}.

\section{Phased Array Radars}
\label{s:par}
Phased array radars move away from parabolic dish antennas to so-called antenna-arrays.  An antenna array is made up of
discrete antennas distributed in space. Each antenna element may function as a receiver and/or transmitter.  The NWRT
Phased Array Radar, for example, has 4352 elements distributed across a plane \citep{Zrnic:2007}. Each of the elements
is able to transmit a signal with a different phase shift. The distribution of the phase shifts will result in the
signals of all the array elements to add constructively, steering the radar beam electronically in a certain direction.

This electronic steering has the advantages of being very agile and flexible and can reduce overall scan time, focus on
particular targets in space in any order, or reduce data correlation by steering the beam out-of-order across a storm, a
technique called beam multiplexing (BMX) \citep{Yu:2007}. BMX provides fast updates of weather information with higher
statistical accuracy because collected sample pairs are independent since they have not been collected sequentially, allowing
the atmosphere to decorrelate before taking the next pair of samples in the same area.

Some phased array radars, so-called \emph{active arrays}, do not have just one transmitter. Instead, they use one smaller
amplifier for each element. These amplifiers are solid-state amplifiers, whose peak power is generally lower
compared to a klystron or magnetron. As such, they operate at lower voltages,
have a lower gain, and require longer duty cycles, but have higher reliability. To achieve the needed power, many of
them must be operated in parallel and in multiple stages as modules \citep{Skolnik:2001}. So, instead of using a klystron
or magnetron, an active phased array radar incorporates individual amplifier modules into each antenna element.
One major advantage of this kind of modular design is the high reliability and fault tolerance of the
system.  Failure of a small percentage of individual elements still allows the radar to operate, and failed elements can simply be replaced
at a convenient time.  This lowers maintenance and manufacturing costs because the components of the identical elements can be mass
produced and could be available COTS.

One way to compensate for the lower peak power of solid-state amplifiers is to use longer pulses \citep[p.
702]{Skolnik:2001}, given their longer duty cycle. However, a longer pulse decreases range resolution, so a method is
needed to improve the range resolution while keeping the sensitivity acceptable. One such method is called pulse
compression.


\section{The Need for Pulse Compression}
Pulse compression is a technique to increase sensitivity (increase the signal-to-noise ratio (SNR)) by using waveforms
that allow the duration and bandwidth to be controlled separately \citep[pp. 42-43]{Richards:2005}.  An early paper by
Bell Labs introduced the concept of a chirp, or frequency modulated pulse, for exactly this purpose
\citep{Klauder:1960}. Since then, other ways of achieving the same goal have been found and will be discussed in this
work. Compression is achieved by using one or more matched filters, the concept of which will be introduced in Chapter
\ref{genwaveform}.  This allows the design of waveforms such that their energy is focused in a mainlobe with low
sidelobes, concentrating the signal energy as narrowly as possible \citep[p. 44]{Richards:2005}. This is similar to
the main- and sidelobes of an antenna pattern, but in range-time instead.

As mentioned previously, solid-state amplifiers are economical, but have lower peak power and higher duty cycles, so
long pulses are necessary to achieve the same sensitivity. On one hand, range resolution is important to detect small
features and would suffer severely from longer pulses.  On the other hand, the SNR is directly related to the
probability of detection \citep[p. 44]{Skolnik:2001}, so it is very important to maintain a sufficient SNR.
Furthermore, since weather signals have a large dynamic range, required for detecting anything from light rain or even
just clear air measurements, to heavy rain and hail, a high SNR is absolutely necessary.  These requirements for good
range resolution and SNR make pulse compression a welcome technique to improve weather radars. The different ways of
accomplishing pulse compression are discussed in this thesis.

Chapter 2 discusses the general aspects of waveforms and their design and introduces a way to measure their performance.
The topic of Chapter 3 is pulse compression and the various ways of applying modulation to the waveform. Chapter 4
explains the performance metrics used in this work to quantify waveform performance, with Chapter 5 presenting the
results. Chapter 6 provides the conclusion and future work.



\chapter{General Waveform Design}
\label{genwaveform}
\section{Pulsed Radar}
\section{Staggered PRT}
\section{Coded Pulse}
\section{Pulse Compression}




\chapter{Pulse Compression}
\label{chap:pulsecompression}
Having looked at general waveform design, several techniques of pulse compression are considered next. As previously
mentioned, the goal is to increase sensitivity without reducing the range resolution and thus control the sensitivity
and bandwidth separately.  To that end, different types of modulation are examined that allow the signal to be designed
to the desired specifications.


\section{Amplitude Modulation}
\label{s:ampmod}
One way to suppress range sidelobes is to taper the amplitude of the signal such that less energy is transmitted towards
the edges of the signal in order to reduce so-called \emph{spectral leakage} caused by working with finite sequences of
a signal \citep{Harris:1978}. In terms of bandwidth, the discontinuities of finite signals introduce sharp edges
that, similar to a perfectly rectangular pulse, increase the bandwidth needed to properly represent the signal. Since the
bandwidth of any system is limited, amplitude tapering can help condition the signal to match the bandwidth requirements
of the system.  However, tapering has at least two disadvantages.  First, less energy is illuminating the target because
the amplifier does not transmit as much power for an amplitude-tapered waveform as a rectangular pulse. This results in
a lower SNR, which has a negative impact on detection. Second, the mainlobe is usually broadened, resulting in a decrease in
spatial resolution.

In this work, the Kaiser window is focused upon because of its flexibility and the fact that its shape can be changed based
on only one parameter, $\alpha$.  Evaluating different waveforms tapered by the Kaiser window with different values of
$\alpha$ allows the amplitude tapering to be optimized for weather radar applications.  The Kaiser window is defined as:

\begin{equation}
  \label{eq:kaiser}
  w_n = \frac{I_0\left(\pi \alpha \sqrt{1- \left( \frac{n}{N/2} \right) ^2} \right)} { I_0(\pi \alpha)}, 0 \leq |n| \leq N/2,
\end{equation}
where $I_0$ is the zeroth order Modified Bessel function of the first kind, $N$ is the length of the sequence, and
$\alpha$ is the parameter that determines the shape of the window \citep{Harris:1978}.
For the sake of simplicity, the function argument is redefined as $\beta = \pi \alpha$, resulting in the function:

\begin{equation}
  \label{eq:kaiser2}
  w_n = \frac{I_0\left(\beta \sqrt{1- \left( \frac{n}{N/2} \right) ^2} \right)} { I_0(\beta)}, 0 \leq |n| \leq N/2,
\end{equation}
This work will use $\beta$ as the parameter describing the window shape.  Figure \ref{fig:kaiserparams} shows the Kaiser
Window function for varying values of the parameter $\beta$.

\begin{figure}[H]
\includegraphics[width=\imgsizes]{figures/kaiserparams.png}
\caption{For amplitude tapering, a Kaiser window is applied to the discussed waveforms.
  This figure shows the Kaiser window function for varying values of its parameter $\beta$.}
\label{fig:kaiserparams}
\includegraphics[width=\imgsizes]{figures/fftspec.png}
\caption{Frequency spectrum of a \SI{15}{\micro\second} pulse that has been tapered with various windows. The spectrum
of a rectangular pulse is given as a reference.}
\label{fig:fftspec}
\end{figure}


Figure \ref{fig:fftspec} shows the frequency spectrum of a rectangular pulse compared to pulses that have been tapered
with different Kaiser windows. The spectrum of the rectangular pulse is a sinc function with sidelobes extending into
infinity. Applying a window lowers the sidelobes and broadens the mainlobe. As mentioned previously, tapering can also
limit the bandwidth of the signal.

To make use of amplitude tapering, one important consideration is the choice of amplifier used in the system. Most
transmitters operate most efficiently in saturation \citep[p. 692]{Skolnik:2001}. For amplitude
tapering to work, the transmitter must operate linearly, which means it is not operating at peak efficiency. This loss
in power will result in decreased sensitivity, so the trade-offs of this method must be considered.



\section{Frequency Modulation}
Frequency modulation is generally divided into two areas, Linear Frequency
Modulation (LFM) and Non-Linear Frequency Modulation (NLFM). This section will
examine how both schemes work and compare them to amplitude modulation, discussed above.


\subsection{Linear Frequency Modulation}
\label{s:lfm}
For LFM, the frequency of the pulse is simply modulated linearly over a certain bandwidth $B$. This is called a chirp
and is defined by:

\begin{equation}
\label{eq:LFM}
s(t) = \exp(-j \pi k t^2 ) ,\ \ 0 \leq t \leq \tau,
\end{equation}
where $k= \pm \frac{B}{\tau}$ \citep[p.57]{Levanon:2004}. The real part of the waveform is shown in Figure
\ref{fig:lfm-waveform}.

\begin{figure}[h]
\includegraphics[width=\imgsize]{figures/lfm-waveform-5us.png}
\caption{Varying the frequency linearly over a certain bandwidth for the duration of the pulse is called Linear
Frequency Modulation. This plot shows the frequency increase over the pulse length $\tau=\SI{5}{\micro\second}$.}
\label{fig:lfm-waveform}
\end{figure}
A quadratic phase modulation is the equivalent of a linear frequency modulation, because the instantaneous frequency is
the derivative of the phase. Thus, for a phase $\phi = 2 \pi (f_\mathrm{0} t + k t^2 / 2)$, the instantaneous frequency
is \citep{Klauder:1960}:
\begin{equation}
  f_\mathrm{i} = \frac{1}{2 \pi} \frac{d \phi}{dt} = f_\mathrm{0} + k t,
\end{equation}
where $f_\mathrm{0}$ is the carrier frequency which, in these examples, is $0$.

The maximum theoretical range resolution of an LFM waveform is only dependent on the bandwidth $B$, not the pulse length $\tau$,
and is expressed as \citep[p. 196]{Richards:2005}:
\begin{equation}
  \label{eq:lfm-rangeres}
  \Delta r = \frac{c}{2 B}.
\end{equation}
If $B=\frac{1}{\tau}$, this result is consistent with a traditional pulsed radar, as shown in Equation
\eqref{eq:rangeresbwpulse}.

As seen in Figure \ref{fig:lfm-full}, the slow drop-off of the peak in the AF, also called a ridge, makes LFM pulse
compression very Doppler tolerant and thus convenient for radar applications that have to deal with targets whose
velocity is unknown. The mismatch in range caused by the shifted ridge is called \emph{range-Doppler coupling} and can
be corrected by alternating between an up- and a down-chirp and averaging the resulting range of both to get the true
range \citep[p. 343]{Skolnik:2001}. However, since the range mismatch is in range-time and changes the location of range
gates, the averaging process to get the true range will result in range gate misalignments for distributed targets and
introduce inaccuracies.

The Doppler tolerance of the LFM waveform is a consequence of a property of the AF. The AF of a waveform that has a linear
frequency shift applied to it, i.e., $s'(t) = s(t) \exp(-j \pi k t^2)$, is given as follows:

\begin{align*}
  |\chi(\tau_\mathrm{d},f_\mathrm{d})|
  &= \left| \int_{-\infty}^\infty{s(t) \exp(-j \pi k t^2) s^*(t+\tau_\mathrm{d}) \exp[j \pi k (t + \tau_\mathrm{d})^2]
  \exp(-j2\pi f_\mathrm{d} t) dt} \right| \\
  &= \left| \exp(j \pi k \tau_\mathrm{d}^2)\int_{-\infty}^\infty{s(t) s^*(t+\tau_\mathrm{d})
  \exp[j 2 \pi (  k \tau_\mathrm{d} - f_\mathrm{d}) t] dt} \right| \\
  &= \left| \exp(j \pi k \tau_\mathrm{d}^2) ~ \chi'(\tau_\mathrm{d}, k \tau_\mathrm{d} - f_\mathrm{d} ) \right| \\
  &= \left| \chi'(\tau_\mathrm{d}, k \tau_\mathrm{d} - f_\mathrm{d} ) \right|
\end{align*}
This shows that, for a signal modulated with a quadratic phase shift, the Doppler dimension changes based on the range
delay, causing the shifted ridge.

\begin{figure}[H]
\includegraphics[width=\imgsizes]{figures/lfm-full-15us.png}
\caption{AF for an up-chirp LFM with $\tau=\SI{15}{\micro\second}$ for a \SI{9550}{\mega\hertz} radar. LFM waveforms are very Doppler
tolerant as seen by the ridge. Notice the extreme radial velocities and the comparatively small range mismatch.}
\label{fig:lfm-full}
\includegraphics[width=\imgsizes]{figures/lfm-full-downchirp-15us.png}
\caption{AF for a down-chirp LFM with $\tau=\SI{15}{\micro\second}$ for a \SI{9550}{\mega\hertz} radar. Chirping down
instead of up changes the direction of the ridge and makes range corrections possible.}
\label{fig:lfm-full-down}
\end{figure}



\subsection{Non-linear Frequency Modulation}
\label{s:nlfm}
Equation \eqref{eq:LFM} is a special case that describes
linear frequency modulation. A more general expression for a frequency modulated waveform is:

\begin{equation}
\label{eq:NLFM}
s(t) =  \exp(-j \Phi(t)) = \exp(-j 2 \pi \int_0^t{f(x) dx}) ,\ \ 0 \leq t \leq \tau,
\end{equation}
where $f(x)$ defines the function of the frequency sweep \citep[p. 88]{Levanon:2004}. 
In the LFM case, $f(x)=kx$, which simplifies Equation \eqref{eq:NLFM} to:

\begin{equation}
\label{eq:NLFM_simp}
s(t) = \exp(j 2 \pi \int_0^t{f(x) dx}) =  \exp(j 2 \pi \frac{k t^2}{2}) = \exp(j \pi k t^2),
\end{equation}
which indicates that a quadratic phase modulation corresponds to a linear frequency change.  

However, $f(x)$ can be any function that describes the sweep pattern within the frequency band.  In this work, various
sweep patterns are investigated and their performance evaluated. The results are presented in Chapter
\ref{chap:optimization}.


\subsection{Early Use of Frequency Modulation} 
Since LFM is simply taking advantage of a property of the AF, as described in Section \ref{s:lfm}, it was one the
earliest ways of improving sensitivity \citep[p. 57]{Levanon:2004} (see also \cite{Klauder:1960}).  It is simple to implement in analog hardware by
using a voltage controlled oscillator (VCO) and is Doppler tolerant due to the slow decrease in slope, making it a
popular choice for early radars that had to detect fast moving targets. With more processing being done in digital
systems, other forms of pulse compression are becoming increasingly more common because they are easier to implement in
digital systems.


\section{Phase Modulation}
Similar to amplitude modulation and frequency modulation, waveforms can be modulated in phase as well. A pulse $s(t)$ of
length $\tau$ is divided into $M$ different parts, and each part is assigned a different phase \citep[p. 100]{Levanon:2004}.
The set of phases associated with the different parts of the pulse is called the \emph{phase code} of
$s(t)$. There is an unlimited number of phase codes and much work has gone into finding codes that improve the
performance of waveform. Only a few different types will be discussed here.  The reader is encouraged to refer to
\cite{Levanon:2004} for a much more comprehensive treatment of phase codes.


\subsection{Binary Codes}
Binary codes vary the phase of the signal in increments of $\pi$.  The most commonly known phase code is the Barker code
\citep[pp. 273-287]{Barker:1953}, which is believed to exist up to a length of 13 \citep[p. 106]{Levanon:2004}.  The
Barker 13 code looks like +1 +1 +1 +1 +1 -1 -1 +1 +1 -1 +1 -1 +1 . It is called a ``perfect" code because its ACF has a
main lobe of height 13 and sidelobes with heights of no more than 1, as depicted in Figure \ref{fig:barker13-acf}.

\begin{figure}[h]
\includegraphics[width=\imgsize]{figures/barker13-acf.png}
\caption{ACF of the Barker code with length 13. Barker codes are ``perfect'', meaning they have minimal
sidelobes.}
\label{fig:barker13-acf}
\end{figure}

\subsection{$n$-ary Codes}
\label{s:narycodes}
Instead of varying the phase of the signal between only 0 and $\pi$, they can be varied arbitrarily for any value
between 0 and $2\pi$ in an attempt to gain a better performing code. These types of codes are also known as
\emph{polyphase codes}. Examples, including a table of polyphase Barker codes up to length 45, are given in \cite[pp.
110-112]{Levanon:2004}.

Other polyphase codes include Frank codes and their derivatives, like P1, P2, and Px codes \citep[pp. 113-122]{Levanon:2004}, which
are derived from sampling the phase of an LFM waveform and thus called \emph{chirplike}.


\subsection{Bandwidth for Phase Codes}
The bandwidth of a phase-modulated waveform is approximately the inverse of the length of a subpulse \citep[p.
350]{Skolnik:2001}. Given a pulse with length $\tau$ and $M$ subpulses, the length of a subpulse is $T =
\frac{\tau}{M}$ and the bandwidth of the whole pulse will be approximately $B = \frac{1}{T}$ Hz. This is a great example about how
pulse compression allows the decoupling of pulse length and signal bandwidth.


\chapter{Performance of Pulse Compression}
\label{performance}
This chapter examines the different approaches to quantifying waveform performance in the context of weather radar
applications, as well as associated limitations.

\section{Quantifications}
\label{s:quantifications}
In order to evaluate the pulse compression techniques examined in this work, several different metrics are used to
quantify their performance. Together they can assist in creating and evaluating waveforms that perform optimally for
weather radars.

\subsection{3 dB Range Resolution}
The \SI{3}{\decibel} range resolution, given in units of meters, is the range resolution of the mainlobe
\SI{3}{\decibel} down from the peak.  Lower value means finer resolution of the waveform. The factors that influence
range resolution are mainly pulse length and bandwidth, as described in Equations \eqref{eq:rangeres} and
\eqref{eq:rangeresbwpulse}. For pulse length, the shorter the pulse, the better the resolution; for bandwidth, the
higher the bandwidth the better the resolution. Amplitude tapering or NLFM can also change the resolution because they
can affect the bandwidth of the waveform.


\subsection{Integrated Sidelobe Level}
The performance of each code is measured by calculating the Integrated Sidelobe Level (ISL), which is defined as
\citep{Keeler:1999}:

\begin{equation}
\label{ISL}
\text{ISL} = 10 \log \left(\frac{\sum\limits_{j} |s_j|^2}{\sum\limits_{k} |m_k|^2}\right)
\end{equation}
where $s_j$ is the amplitude of the $j^{th}$ sidelobe data point, and $m_k$ the amplitude of the $k^\mathrm{th}$
mainlobe data point. The ISL is thus the ratio between the area under the sidelobes and the area under the mainlobe, as
depicted in Figure \ref{fig:barker13-isl}.

\begin{figure}[H]
\includegraphics[width=\imgsize]{figures/barker13-isl.png}
\caption{The ACF of a Barker 13 code. The ISL describes the ratio between the area under the sidelobes (red) and the area under the
mainlobe (blue).}
\label{fig:barker13-isl}
\end{figure}

Lower ISL means better performance of the code. In this work, the ISL is calculated across the whole AF, but limited in
the Doppler dimension for values $v \in \left[-50, +50\right]\ \si{\metre\per\second}$ as the main focus is weather
applications and encountering targets that move faster than this is rare. 

\subsection{Peak Sidelobe Level}
The Peak Sidelobe Level (PSL) is the ratio between
the highest sidelobe and the mainlobe in the plane of the AF and complements the ISL. It is given by
\citep{Keeler:1999}:

\begin{equation}
\label{PSL1}
\text{PSL} = 20 \log \left(\frac{\max(s_j)}{\max(m_k)}\right).
\end{equation}
Although the ISL is a good way to quantify waveform performance, it can be misleading. Sidelobes are very undesirable,
so a waveform that has overall low sidelobes except for one very dominant one may have a low ISL, but may not be
performing well enough for actual use due to the outlier sidelobe. 

\subsection{Pulse Power Ratio}
For amplitude tapering, another factor needs to be taken into consideration, called the Pulse Power Ratio (PPR). For a
rectangular pulse, \SI{100}{\percent} of the signal power is transmitted. For an amplitude-tapered pulse, however, that
is not the case. The PPR is defined as follows:


\begin{equation}
\label{PPR}
\mathrm{PPR} = 10 \log \left(\frac{\int_{\tau} w^2(t)dt}{A_\mathrm{p}^2 \tau}\right), 
\end{equation}
where $w(t)$ is the tapering function applied to the pulse, integrated over the pulse length $\tau$, and $A_\mathrm{p}$ is the
original amplitude.  This metric indicates how much power is lost when amplitude tapering is applied compared to a
rectangular pulse with equal pulse length, and can help quantify the loss in sensitivity caused by tapering.


\section{Limitations}
As mentioned before, the core of pulse compression is the matched filtering. Any kind of moving target, including weather,
will introduce distortions to the waveform that cause a mismatch between the filter and the waveform. Depending on the
waveform, this mismatch can be quite severe and greatly impact the performance of the weather radar. 
Besides weather, other sources of distortion limit the performance of the waveform. In this section, another limitation
that will be examined is the effect of a system impulse response on the waveform.


\subsection{Doppler Effect for Weather Radars}
\label{s:dopplereffect}
One concern for pulse compression performance is the effect of a Doppler shift on the matched filter in the receiver. A
received signal that has been affected by a Doppler shift causes a mismatch in the matched filter processing of the
receiver.  This is especially pronounced at high target velocities and is dependent on the used waveform.

This mismatch effect for a pulse waveform on the matched filter can be quantified as follows. Referring back to Equation
\eqref{eq:pulseaf}, consider the Doppler dimension only, i.e. $\tau_\mathrm{d} = 0$. This is the familiar
$\mathrm{sinc}$ function:

\begin{equation}
\label{eq:pulseafsinc}
  |\chi(0,f_\mathrm{d})| = \left| \frac{\sin(\pi f_\mathrm{d} \tau ) }{\pi
  f_\mathrm{d} \tau} \right| \text{, for } -\tau \leq \tau_\mathrm{d} \leq \tau.
\end{equation}
This function has a value of $0$ exactly when $f_\mathrm{d}$ is a multiple of $1/\tau$, as shown in Figure
\ref{fig:sincnulls}. Thus, if a Doppler frequency is much less than the inverse of the pulse length, i.e., $f_\mathrm{d}
\ll 1/\tau$, its effect on the waveform can be neglected.

\begin{figure}[h!]
\includegraphics[width=\imgsize]{figures/sincnulls.png}
\caption{The $\mathrm{sinc}$ function describes the AF of a pulse in the Doppler dimension. It is $0$ when
$f_\mathrm{d}$ is a multiple of $1/\tau$.}
\label{fig:sincnulls}
\end{figure}

For atmospheric weather radars, such as the WSR-88D with a wavelength of \SI{10}{\centi\metre}, and a target velocity of \SI{50}{\metre\per\second}, a Doppler frequency of $|f_\mathrm{d}| =
\left|\frac{2v}{\lambda}\right| = \frac{100}{0.1} = \SI{1}{\kilo\hertz}$ is observed. That is very low compared to the first null occurring
at $1/\tau$, where $\tau=\SI{4.57}{\micro\second}$, the long pulse of a WSR-88D radar \citep[p.  47]{Doviak:2006}. In
this case, $1/\tau = 1/\SI{4.57}{\micro\second} = \SI{218.82}{\kilo\hertz}$. Thus a sufficiently short pulse is very
Doppler tolerant, although that Doppler tolerance decreases with increased pulse length, as the optimization
calculations will show.



\subsection{Effects of Amplifier Impulse Response on the Ambiguity Function}
\label{s:ireffects}

Next, the effects of the components in the transmit and receive chains on the pulse will be examined. For
simplification, these
components, excluding the matched filter, are represented by some equivalent system with
impulse response $g(t)$, in the case of the receive chain, and $g'(t)$, in the case of the transmit chain. A block
diagram is shown in Figure \ref{fig:distortblock}.  Starting with Equation \eqref{eq:AF} describing the
AF, the signal is convolved with $g(t)$, the impulse response of the system. The AF then becomes:

\begin{align*}
\label{eq:afwithir}
   \chi(\tau_\mathrm{d},f_\mathrm{d}) &= \left|\int_{-\infty}^\infty{\left[(s(t) \exp(-j2\pi
   f_\mathrm{d} t)) * g(t)\right] s^*(t+\tau_\mathrm{d}) \ dt}\right|
\end{align*}
The results for various waveforms are presented in Section \ref{s:ireffectsopti}.

\begin{figure}[h!]
\includegraphics[width=\imgsize]{figures/distortblock.png}
\caption{Block diagram showing the radar TX and RX chains. The components are represented by equivalent systems with
impulse responses $g(t)$ for the RX chain and $g'(t)$ for the TX chain. The matched filter impulse response is denoted
as $h(t)$.}
\label{fig:distortblock}
\end{figure}


\chapter{Waveform Optimization}
\label{chap:optimization}
In this section the setup and results of the simulations will be presented. First, the way the 
simulations were conducted is explained. After that, the results for various waveforms will be 
presented.

\section{Setup and Configuration}
All simulations were performed with the MATLAB\textsuperscript{\textregistered} software package.
The general algorithm to compute the ambiguity function is presented, with the complete code
given in Appendix \ref{app:code}.


\subsection{Generating Waveforms}
The simulation is done in multiple parts. First, a waveform is generated based on a given
amplitude, phase, frequency modulation, and impulse response that is applied to the signal.
The returned waveform is generated at a proper sampling frequency in order to avoid aliasing.

If an impulse response was supplied, both the distorted and clean waveform will be returned, as well as
the new pulse length $\tau$ which will be approximately twice the original $\tau$. That lengthening
is due to the convolution of the original waveform $u(t)$ with the impulse response $h(t)$.

\subsection{Calculating the Ambiguity Function}
Once the waveform has been returned, the AF is generated and plotted in a surface plot. 
In pseudo-code, given the complex signal \emph{signal} and an array of Doppler frequencies \emph{doppler\_frequencies},
the normalized AF is computed as follows.

\lstset{
  basicstyle=\footnotesize
}
\begin{lstlisting}[language=Ruby]
af = []
for dfreq in doppler_freqencies do
  doppler_shift = exp(-j*2*pi*dfreq)
  shifted_signal = signal*doppler_shift
  af << abs(xcorr(signal, shifted_signal))
end
return (af / max(af))

\end{lstlisting}









\section{Simple pulse}
A rectangular pulse is the simplest waveform for a radar. These graphs show the AF for pulses of varying lengths so 
that the effect of pulse length on Doppler resistance can be examined. Figure \ref{fig:simplepulse-1us} shows the AF
of a \SI{1}{\micro\second} pulse. Across the delay axis, the characteristic triangle shape can be seen. Across the 
Doppler axis, there is no visible change in the AF. The different Doppler cuts are identical, so for weather radar 
applications, the commonly encountered Doppler shifts do not cause any distortion to the signal for short pulses.

Similarly, Figure \ref{fig:simplepulse-200us} shows the AF of a much longer, \SI{200}{\micro\second} pulse. Comparing
it to the shorter pulse, it looks very similar. However, unlike for a short pulse, there is some distortion at higher
Doppler velocities. This distortion is visibly minimal, so the impact on weather radar application is negligible because,
first, the distortion doesn't occur until higher velocity weather targets, and second, pulse lengths of \SI{200}{\micro\second}
are not common. For comparison, the two pulse lengths used in the WSR-88D radar network are \SI{1.57}{\micro\second} and 
\SI{4.57}{\micro\second} \citep[p. 47]{Doviak:2006}.




\begin{figure}[H]
\includegraphics[width=\imgsize]{figures/pulse-1us.png}
\caption{ Simple pulse of length 1 $\mu s$ }
\label{fig:simplepulse-1us}
\end{figure}

\begin{figure}[H]
\includegraphics[width=\imgsize]{figures/pulse-200us.png}
\caption{ Simple pulse of length 200 $\mu s$ }
\label{fig:simplepulse-200us}
\end{figure}









\section{Barker code}
Next, a length 13 Barker code of different pulse lengths will be examined. Figures \ref{fig:barker13-1us} and
\ref{fig:barker13-1us-0D} show a \SI{1}{\micro\second} pulse that has been phase coded with a length 13 Barker 
code. As for a rectangular pulse, there is no visible degradation of the signal with increasing Doppler shifts.
The zero-Doppler cut shows the typical shape of the Barker code and sidelobes of about
\SI{-22}{\decibel} \citep[p. 351]{Skolnik:2001}.

Comparing these images to Figures \ref{fig:barker13-200us} and \ref{fig:barker13-200us-0D} shows little difference
in the Doppler dimension, confirming again that for weather radar applications, the Doppler distortions can be 
neglected. The only difference is the expected loss of range resolution caused by the longer pulse.

\begin{figure}[H]
\includegraphics[width=\imgsize]{figures/barker-1us.png}
\caption{ Barker 13 of length 1 $\mu s$ }
\label{fig:barker13-1us}
\includegraphics[width=\imgsize]{figures/barker-1us-0D.png}
\caption{Zero Doppler cut of the AF}
\label{fig:barker13-1us-0D}
\end{figure}

\begin{figure}[H]
\includegraphics[width=\imgsize]{figures/barker-200us.png}
\caption{ Barker 13 of length 200 $\mu s$ }
\label{fig:barker13-200us}
\includegraphics[width=\imgsize]{figures/barker-200us-0D.png}
\caption{Zero Doppler Cut of the AF}
\label{fig:barker13-200us-0D}
\end{figure}











\section{Linear Frequency Modulation}
Here the results of different variations of linear frequency modulation will be presented, specifically LFM with
and without amplitude tapering.

\subsection{Linear Frequency Modulation}
In this section the performance results of a linearly frequency modulated waveform will be presented.

The frequency modulation function is shown in figure \ref{fig:lfm-fmf}. It is a linear sweep from \SI{-2.5}{\mega\hertz} to
\SI{2.5}{\mega\hertz}, over a total of \SI{5}{\mega\hertz}.

\begin{figure}[H]
\includegraphics[width=\imgsize]{figures/lfm-fmf.png}
\caption{Linear frequency modulation function}
\label{fig:lfm-fmf}
\end{figure}


Figures \ref{fig:lfm-15us} and \ref{fig:lfm-200us} show the AFs of two LFM waveforms, of \SI{15}{\micro\second} and
\SI{200}{\micro\second} lengths, respectively.
Comparing the two for differences shows once again that in the Doppler dimension, the pulse length has
little effect for weather radar applications. 

One observation of significance is the range resolution of both signals.
Figures \ref{fig:lfm-15us-0d} and \ref{fig:lfm-200us-0d} show the zero-Doppler cut of both signals.
As expected, the sidelobes are around -13 dB in both AFs, but also, the shapes are approximately the same
for each case, which means that the pulse length has little effect on the range resolution. Instead, range
resolution is determined by the sweep bandwidth, not the pulse length. 

\begin{figure}[H]
\includegraphics[width=\imgsize]{figures/lfm-15us.png}
\caption{ LFM of length 15 $\mu s$ }
\label{fig:lfm-15us}
\includegraphics[width=\imgsize]{figures/lfm-200us.png}
\caption{ LFM of length 200 $\mu s$ }
\label{fig:lfm-200us}
\end{figure}



\begin{figure}[H]
\includegraphics[width=\imgsize]{figures/lfm-15us-0D.png}
\caption{ LFM of length 15 $\mu s$, zero Doppler cut. }
\label{fig:lfm-15us-0d}
\includegraphics[width=\imgsize]{figures/lfm-200us-0D.png}
\caption{ LFM of length 200 $\mu s$, zero Doppler cut. }
\label{fig:lfm-200us-0d}
\end{figure}


\subsection{Linear Frequency Modulation with Amplitude Tapering}
LFM has the disadvantage of very high sidelobes at about \SI{-13}{\decibel}.  
These can be mitigate with amplitude tapering, so a Kaiser window is applied to the amplitude. 

Based on Equation \eqref{eq:kaiser2}, the parameter $\beta$ is
varied from 0 to 50, and the AF generated and evaluated for each by quantifiying its performance according to the metrics
described in Section \ref{s:quantifications}. The results are shown in figure \ref{fig:lfm-isl-15us-opti}.

As expected, the ISL dramatically decreases, as does the MSL. At the same time, the range resolution decreases from
an initial 40 m to about 200 m. The PPR may be the limiting metric here because the tapered signal quickly
loses half of its energy with only minor amplitude tapering and eventually decreases to about \SI{20}{\percent} of
the original amplitude, which means the signal loses \SI{80}{\percent} of its energy. This, in turn, means reduced sensitivity.



\begin{figure}[H]
\includegraphics[width=\imgsize]{figures/lfm-isl-15us-opti.png}
\caption{Optimization results for LFM with Kaiser amplitude tapering}
\label{fig:lfm-isl-15us-opti}
\end{figure}


Clearly, a tradeoff must be made. If the half power point is selected, the waveform would have an ISL of about \SI{-20}{\decibel}, 
MSL of \SI{-33}{\decibel}, and a range resolution of \SI{68}{\metre}.

The corresponding zero Doppler cut of the AF at this point is shown in figure \ref{fig:lfm-isl-15us-opti-slice13}. It 
clearly shows the sidelobes at about -70 dB, which means this waveform would be adequate for typical weather observations.


\begin{figure}[H]
\includegraphics[width=\imgsize]{figures/lfm-isl-15us-opti-slice13.png}
\caption{Zero Doppler cut of the AF at the half power point.}
\label{fig:lfm-isl-15us-opti-slice13}
\end{figure}






\section{Non-Linear Frequency Modulation}
As described in Section \ref{s:nlfm}, the frequency of a frequency modulated signal can be swept non-linearly as well. In
order to find more useful waveforms for weather radar applications, this work evaluates several different sweep functions 
described by Equation \eqref{eq:nlfm-sweep} below. The general form is

\begin{equation}
\label{eq:nlfm-sweep}
  f(t) = a t^7 + t,
\end{equation}
where $a$ varies from 0 to 10. Whole functions for different values of $a$ are shown in figure \ref{fig:nlfm-functions}. 
Each function is applied over the pulse length $\tau$ of the signal.

\begin{figure}[H]
\includegraphics[width=\imgsize]{figures/nlfm-functions.png}
\caption{Functions describing the non-linear frequency sweeps}
\label{fig:nlfm-functions}
\end{figure}

For each function a waveform and ambiguity functions are generated and then evaluated according to the optimizations described
in Section \ref{s:quantifications}. It can then be determined which sweep function is best suited for weather radar applications, 
or further changes to the setup can be made to combine it with, e.g. amplitude tapering to achieve even better performance.


\subsection{Parameterized NLFM Results}
\label{s:nlfm-only-results}
Figure \ref{fig:nlfm-side-15us-opti} shows the results for varying NLFM waveforms in regards to ISL, MSL, and range resolution.
We find that the closer the frequency modulation function approaches an S-curve, the better the ISL becomes, though it is never
fantastic. The maximum ISL seems to be around \SI{-15}{\decibel}. The maximum sidelobe level decreases rapidly first to about 
\SI{-13}{\decibel} and then slowly decreases to about \SI{-20}{\decibel}, where it stays fairly constant. The range resolution
gets progressively worse, starting at about \SI{30}{\metre} and increasing to about \SI{210}{\metre}, meaning the mainlobe broadens.
This also explains the decreasing and eventually constant MSL, because the whole AF is one big mainlobe with only minimal sidelobes.

\begin{figure}[H]
\includegraphics[width=\imgsize]{figures/nlfm-side-15us-opti.png}
\caption{Optimization results for NLFM}
\label{fig:nlfm-side-15us-opti}
\end{figure}

To get a better picture about NLFM waveforms, consider Figure \ref{fig:nlfm-side-15us-0D-1}, showing the zero-Doppler slice of
the NLFM waveform generated with the sweep function with $a=2.9293$. The mainlobe is fairly wide, though only about \SI{50}{\metre}
at the \SI{3}{\decibel} point. The sidelobes are better than those of the LFM waveform at slightly higher than \SI{-30}{\decibel}.
In comparison, Figure \ref{fig:nlfm-side-15us-0D-2} shows the NLFM waveform generated with the sweep function with $a=6.46$. 
Altough the sidelobes are a little lower at \SI{-40}{\decibel} and the range resolution at the \SI{3}{\decibel} point is still 
only about \SI{70}{\metre}, the mainlobe quickly becomes very wide, making it not ideal for weather radar applications.

\begin{figure}[H]
\includegraphics[width=\imgsize]{figures/nlfm-side-15us-0D-1.png}
\caption{Zero-Doppler slice for AF with $a=2.92$}
\label{fig:nlfm-side-15us-0D-1}
\includegraphics[width=\imgsize]{figures/nlfm-side-15us-0D-2.png}
\caption{Zero-Doppler slice for AF with $a=6.46$}
\label{fig:nlfm-side-15us-0D-2}
\end{figure}


\subsection{Parameterized NLFM with Amplitude Tapering}
In order to improve the performance of the varying NLFM waveforms, Kaiser windows are applied to taper the amplitude.
The results are presented in Figures \ref{fig:nlfm-kaiser-15us-opti-isl}-\ref{fig:nlfm-kaiser-15us-opti-res}.

Looking at the ISL in figure \ref{fig:nlfm-kaiser-15us-opti-isl}, it can be seen that the waveform that is most aggressively 
tapered generally has better ISL performance. The same is true for the NLFM parameter $a$. 
Next, the MSL shown in figure \ref{fig:nlfm-kaiser-15us-opti-msl} closely mirrors the shape of the ISL optimization,
so the same discussion applies to it.

\begin{figure}[H]
\includegraphics[width=\imgsize]{figures/nlfm-kaiser-15us-opti-isl.png}
\caption{ISL optimization results for NLFM combined with Kaiser amplitude tapering}
\label{fig:nlfm-kaiser-15us-opti-isl}
\includegraphics[width=\imgsize]{figures/nlfm-kaiser-15us-opti-msl.png}
\caption{MSL optimization results for NLFM combined with Kaiser amplitude tapering}
\label{fig:nlfm-kaiser-15us-opti-msl}
\end{figure}


The PPR in Figure \ref{fig:nlfm-kaiser-15us-opti-ppr} is the same for each waveform, because it only depends on 
the shape of the amplitude tapering function determined by the Kaiser parameter. As described previously, the power decreases rapidly
as $\beta$ in the Kaiser window function increases.

Similar to the NLFM-only case in Section \ref{s:nlfm-only-results}, the range resolution
decreases slightly as the NLFM parameter $a$ increases. The same happens for an increasing Kaiser parameter $\beta$.
The effect of both together, however, is compounded such that a large $\beta$ and large $a$ result in a decrease in range
resolution that is much larger than a mere linear superposition of both individually.


\begin{figure}[H]
\includegraphics[width=\imgsize]{figures/nlfm-kaiser-15us-opti-ppr.png}
\caption{PPR optimization results for NLFM combined with Kaiser amplitude tapering}
\label{fig:nlfm-kaiser-15us-opti-ppr}
\includegraphics[width=\imgsize]{figures/nlfm-kaiser-15us-opti-res.png}
\caption{Resolution optimization results for NLFM combined with Kaiser amplitude tapering}
\label{fig:nlfm-kaiser-15us-opti-res}
\end{figure}



To give some examples, Figure \ref{fig:nlfm-kaiser-slices-15us} shows zero-Doppler cuts of a few waveforms with
varying values for $a$ and $\beta$. Similar to the untapered NLFM waveforms, the top left figure shows sidelobes of about
\SI{-40}{\decibel} with only mild tapering, with range resolution of about \SI{180}{\metre}. 
With more aggressive tapering, sidelobes decrease to about \SI{-70}{\decibel}
while sacrificing range resolution, which is about \SI{280}{\metre}, shown on the top right,

The bottom two figures show very low sidelobes, but fairly low range resolution at \SI{300}{\metre} and \SI{400}{\metre} on 
the left and right, respectively. These are not ideal for weather radar applications because even though the sidelobes are
sufficiently low, range resolution is not fine enough.

\begin{figure}[H]
\includegraphics[width=\imgsize]{figures/nlfm-kaiser-slices-15us.png}
\caption{NLFM waveform zero-Doppler cuts with varying a and $\beta$ parameters}
\label{fig:nlfm-kaiser-slices-15us}
\end{figure}





\section{Effects of Impulse Response on the Ambiguity Function}
As discussed in Section \ref{s:ireffects}, as the return signal passes through the receive chain, it will be distorted by the
impulse response of the filter and amplifier. The effects of that on the waveform are presented here.
To provide a baseline, Figure \ref{fig:pulsewoir-15us} shows a rectangular pulse that is undistorted as a reference.

\begin{figure}[!ht]
\includegraphics[width=\imgsize]{figures/pulsewoir-15us.png}
\caption{ Rectangular pulse without IR distortion}
\label{fig:pulsewoir-15us}
\end{figure}


Figure \ref{fig:pulsewir-15us} shows the AF for a pulse that
has been distorted by the transfer function shown in figure \ref{fig:tf1}. Since the exact shape of the impulse response of 
the PX1000 mobile radar is still unknown, it is approximated with a Rayleigh-shaped function.

\begin{figure}[!ht]
\includegraphics[width=\imgsize]{figures/tf1.png}
\caption{Impulse Response}
\label{fig:tf1}
\end{figure}

\begin{figure}[!ht]
\includegraphics[width=\imgsize]{figures/pulsewir-15us.png}
\caption{ Pulse with IR distortion}
\label{fig:pulsewir-15us}
\end{figure}


The result is not unexpected. First, there is again
no visible distortion in the Doppler dimension, which is the same as the simple rectangular pulse. The only effect is in the 
range delay dimension. There, the asymmetric shape of the impulse response is reflected in the range delay of the AF. This
distortion is not desirable and should be corrected.



\chapter{Conclusion and Future Work}
\label{conclusionfuture}
\section{Conclusions}
This work examined several different techniques of pulse compression and evaluated their performance 
in regards to weather radar applications. 
It looks like the waveform with the best performance is a NLFM signal with 
$a=xxxx$ and amplitude tapering.


\section{Future Work}
There are several ways to improve on this work in the future.

First, other waveform can be evaluated based on the presented metrics, like more complex
phase coded signals. Other sweeping functions than the one presented could be evaluated
to perhaps find waveforms that perform better for weather radar applications.

Second, the effects of the amplifier impulse response and other system components
chould be better quantified and the results compared with the theoretical
results that are expected. Particularly, attempts to correct for the distortion caused by the
impulse response could be presented. One possible candidate is pulse pre-distortion that will cancel
out the distortion caused in the receive chain.



%
%\references{agsm}{myreferences}
\references{amermeteorsoc}{references}
%comment out the next four lines if you don't have an appendix
\appendix
\chapter{Code Listing}
\label{app:code}
\OUdoublespace
All of the code will be available on GitHub for the foreseeable future: \\
\texttt{http://github.com/ckoehler/Thesis-code}  \\
The two main functions are given here.

\section{af.m}

\lstset{
  basicstyle=\scriptsize
}
\begin{lstlisting}[language=Matlab]

function [delay v af] = af(impulse_response, signal, tau, fs, v_max, f_points,
    carrier, full_af) 
  % Ambiguity function calculation
  % ambiguity function is af(t,f) = sum_over_t(u(t) * u'(t-tau) * exp(j*2*pi*f*t))
  %
  % fs = Range dimension sampling frequency
  % signal = the signal we need the AF of.
  % clean_signal = if different from signal, this is the signal we use to create 
  %                the time-shifted version, or u'(t-tau) above.
  % tau = signal length

  m = length(signal);
  
  % if no signal is given, assume that all we have is a clean signal
  % and use that.
  if isempty(impulse_response)
    ir=false;
    m_ir = m;
  else
    ir=true;
    m_ir = m;
    tau = 2*tau;
  end

  if nargin < 8
    full_af=false;
  end

  % speed of light
  c = 3e8;
  % wavelength
  lam = c/carrier;

  af = [];
  
  % convert v_max to a frequency
  f_max = 2*v_max/lam;

  % frequency span
  if full_af
    f = linspace(-f_max,f_max, 2*f_points+1);
  else
    f = linspace(0,f_max, f_points);
  end

  for i=1:length(f)
    dshift = exp(1i*2*pi.*f(i).*(0:m-1)/fs);
    shifted_s = signal.*dshift;

    if ir
      shifted_s = conv(shifted_s, impulse_response); 
      m_ir = length(shifted_s);
    end

    af(i,:) = abs(xcorr(signal, shifted_s));
    
  end
  af = af./max(max(af));

  % compute delay
  t = linspace(0, tau, m_ir);
  delay = [-fliplr(t) t(2:end)] * c / 2;

  % convert doppler frequency to velocity
  v = f .* lam ./ 2;
end


\end{lstlisting}



\section{makesignal.m}
\begin{lstlisting}[language=Matlab]
function signal = makesignal(amp, phase, freq_mod, tau, fs)

  if ~isempty(freq_mod)
    fm = true;
  else
    fm = false;
  end


  if isempty(amp)
    m = length(phase);
    amp = ones(1,m);
  end

  if isempty(phase)
    m = length(amp);
    phase = zeros(1,m);
  end

  signal = amp .* exp(j.*phase);
  
  if fm
    freq_mod = 2.*pi.*cumsum(freq_mod./fs);
  else
    freq_mod = 0;
  end

  % reassemble the signal
  signal = signal .* exp(j.*freq_mod);

end

\end{lstlisting}

\end{document}
