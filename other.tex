Note that \LaTeX\ will place Figure \ref{flowchart} after the list,
because I did not use the {\tt [H]} option after the \verb9\begin{figure}9.
\begin{figure}
\begin{center}
\setlength{\unitlength}{1in}
\begin{picture}(4,4)

\put(1.25,3.5){\framebox(1.5,0.5){\shortstack{
control variables}}}

\put(2.,3.4){\vector(0,-1){.25}}
\put(2.1,3.3){\makebox(0,0)[l]{\shortstack{
first estimate of I.C.}}}


\put(1.25,2.5){\framebox(1.5,0.5){\shortstack{
Make forecast using \\ forward model}}}

\put(1.5,2.4){\vector(-1,-1){.25}}
\put(1.3,2.3){\makebox(0,0)[r]{\shortstack{
model output}}}


\put(2.5,2.1){\vector(-1,1){.25}}
\put(2.5,2.2){\makebox(0,0)[l]{\shortstack{
new estimate of I.C.}}}

\put(.25,1.5){\framebox(1.5,.5){\shortstack{
compute \\ cost function J}}}
\put(1.5,1.4){\vector(0,-1){.25}}
\put(-.0,1.75){\vector(1,0){.25}}
\put(-.05,1.75){\makebox(0,0)[r]{\shortstack{
observations}}}


\put(2.25,1.5){\framebox(1.5,.5){\shortstack{
Minimization \\ procedure}}}
\put(2.5,1.1){\vector(0,1){.25}}


\put(.25,.25){\framebox(1.5,.75){\shortstack{
compute \\ Lagrangian \\ multipliers}}}

\put(1.9,.5){\vector(1,0){.25}}

\put(2.25,.25){\framebox(1.5,.75){\shortstack{
compute gradient\\ $\nabla J(c)$ \\ using adjoint model}}}

%\drawline(0.1,1.8)(0.5,1.8)
%\drawline(0.5,1.8)(0.9,3.0)
\end{picture}




\caption[A flowchart]
{Flowchart of data assimilation using the adjoint method.}
\label{flowchart}
\end{center}
\end{figure}

\begin{description}
\item[Step 1:] Make a first estimate of the initial condition (I.C.).
\item[Step 2:] Run the forward model to generate a forecast over the 
assimilation interval.
\item[Step 3:] Compute the cost function, J, using the observations 
and the model output.
\item[Step 4:] Compute the gradient of the cost function with respect 
to the control variable, $\nabla J(c)$, by integrating the adjoint 
model (which is a combination of the forward model and cost 
functional) backwards using the Lagrangian multiplier method.
\item[Step 5:] Generate a new estimate of the initial condition 
(I.C.) for another forecast using a minimization procedure ({\it 
e.g.}, Steepest Descent or Conjugate Gradient Method).  This is done 
iteratively toward the minimum of the cost function using the 
calculated cost function and its gradient.
\item[Step 6:] The process is repeated until some convergence criteria 
are met, {\it i.e.}, the cost functional, J, is near its local minimum.  In 
practice, this is determined by an insignificant change in the 
gradient from one iteration to the next.  If not, the process will be 
terminated when some maximum number of iterations have been achieved.
\end{description}
\section{A Bargraph}
Hey, where did that bargraph go?  I thought I put it in this section!
\label{bargraph}
\begin{figure}
\begin{center}
\centering
\fbox{{\bf Best dense, row-wise algorithms}} \\
\vspace{.5in}
\textbf{Algorithm 1, p=2, gaxpy}\\[5mm]
\begin{barenv}
\setstretch{.3}
\setyaxis{0}{500}{100}\setyname{secs}
\bar{.903}{1}[\texttt{s}]
\bar{116.}{6}[\texttt{1}]
\bar{116.}{1}[\texttt{1}]
\bar{117.}{6}[\texttt{2}]
\bar{58.}{1}[\texttt{2}]
\bar{233.}{6}[\texttt{4}]
\bar{58.}{1}[\texttt{4}]
\bar{465.}{6}[\texttt{8}]
\bar{58.}{1}[\texttt{8}]
\end{barenv}
\caption[CPU Time]{CPU time (hatched) and wall clock time (white) 
for srowdg (s) and a1rowdg with various 
numbers of processors. m=128 N=64}
\label{a1rowdg}
\end{center}
\end{figure}

\section{Some tables}
\label{table}
\begin{table}
\begin{center}
\begin{tabular}{|c|c|} \hline
	time (hours)   &       T (C) \\ \hline
  				   0           &             17.0\\
				   1.5         &            18.8\\
				   3.0         &            19.8\\
				   4.5         &            20.5\\
				   6.0         &            21.0\\
				   7.5         &            21.3\\
				   9.0         &            21.5\\
				  10.5         &           21.8\\
				  12.0         &           22.2\\
				  13.5         &           22.8\\
				  15.0         &           23.5\\
				  16.5         &           24.1\\
				  18.0         &           24.3\\ \hline

\end{tabular}
\caption{T(t) for the initial 18 hours}\label{Tinit}
\end{center}
\end{table}

I will put some text here just to see what happens to it.
 
\begin{table}
\begin{center}
\begin{tabular}{|l|l|l|l|} \hline
Step 1 & \multicolumn{3}{c|}{Step 2} \\ \hline
 & Stage 1 & Stage 2 & Stage 3  \\ \hline
$x_{1}$ & & & \\
$x_{1:2}$ & & & \\
$x_{1:3}$ & & &  \\
$x_{1:4}$ & & & \\ \hline
$x_{5}$ &$x_{1:5}$ & &  \\
$x_{5:6}$,\ $A_{5:6}$ & $x_{1:6}$ & & \\
$x_{5:7}$,\ $A_{5:7}$ & $x_{1:7}$ & &  \\
$x_{5:8}$,\ $A_{5:8}$ & $x_{1:8}$ &&\\ \hline
$x_{9}$& & $x_{1:9}$& \\
$x_{9:10}$,\ $A_{9:10}$ && $x_{1:10}$&  \\
$x_{9:11}$,\ $A_{9:11}$ && $x_{1:11}$ & \\
$x_{9:12}$,\ $A_{9:12}$ &&  $x_{1:12}$ &\\ \hline
$x_{13}$ &&& $x_{1:13}$ \\
$x_{13:14}$,\ $A_{13:14}$ & &&  $x_{1:14}$ \\
$x_{13:15}$,\ $A_{13:15}$ & &&  $x_{1:15}$ \\
$x_{13:16}$,\ $A_{13:16}$ & &&  $x_{1:16}$ \\ \hline
\end{tabular}
\caption[Algorithm \# 3 example]{Algorithm \# 3 example}\label{algo3ex}

\end{center}
\end{table}

\section{Some complicated equations}
To this end, the Lagrangian is introduced
(see \citet{Lanczos} for the theoretical foundations of 
minimizing functions subject to a constraint).
\bea
L(c,\lambda,\mu,\eta) &=& J(c)  \nonumber \\*
&& + \sum_{i=1}^{n}
\lambda_{i}\left[
\theta_{i}-\theta_{i-1}
-\frac{\tau}{h_{i-1}}C_{T}V(1+K)
\left(T_{i-1} - \theta_{i-1} \right)
\right] \nonumber \\*
&& + \sum_{i=1}^{n}
\mu_{i}\left[
h_{i}-h_{i-1}
-\tau K C_{T}V
\frac{\left(T_{i-1} - \theta_{i-1} \right)}
{\sigma_{i-1}}
-\tau W
\right]  \nonumber \\*
&& + \sum_{i=1}^{n}
\eta_{i}\left[
\sigma_{i}-\sigma_{i-1}
+\frac{\tau}{h_{i-1}}C_{T}V(1+K)
\left( T_{i-1} - \theta_{i-1} \right) \right.
\nonumber \\*
&&
\left. \qquad -\tau \gamma K C_{T}V
\frac{\left(T_{i-1} - \theta_{i-1} \right)}
{\sigma_{i-1}}
-\tau \gamma W
\right]
\label{niceeq}
\eea
where $\lambda=\left(\lambda_{1},\ldots,\lambda_{n}\right)^{t}$,
$\mu=\left(\mu_{1},\ldots,\mu_{n}\right)^{t}$,
$\eta=\left(\eta_{1},\ldots,\eta_{n}\right)^{t}$
are the undetermined Lagrangian 
multipliers in (\ref{niceeq}).   

\be
E=
\left[
\begin{array}{cccccccc}
-I_{3} & 0 & 0 & 0 & \ldots & 0 & 0 &0 \\
D_{1} & -I_{3} & 0 & 0 & \ldots & 0 & 0 &0 \\
0 & D_{2} & -I_{3}  & 0 & \ldots & 0 & 0 &0 \\
\ldots & \ldots & \ldots  & \ldots & \ldots & \ldots & \ldots &\ldots \\
0 & \ldots & \ldots  & \ldots & \ldots & \ldots & D_{n-1} & -I_{3} \\
\end{array}
\right],
\label{Edef}
\ee

By definition,
\bea
\epsilon_{i}^{\theta} &=& \theta_{i} - \overline{\theta}_{i},\label{one}\\* 
\epsilon_{i}^{h} &=& h_{i} - \overline{h}_{i},\label{two}  \\* 
\epsilon_{i}^{\sigma} &=& \sigma_{i} - \overline{\sigma}_{i}. 
\label{three} 
\eea
Now we can still refer to (\ref{Edef}) or (\ref{one})-(\ref{three}).
