\section{Continuous Wave}
Continuous wave (CW) radars will not be considered here and only be mentioned briefly. These bistatic
radars continuosly transmit a wave from one antenna and receive with another. While they have many uses,
their use in weather observation is almost non-existant.
\section{Pulsed Radar}
The basic pulsed radar emits many pulses of duration $\tau$ towards the target by switching the transmitter
on for $\tau$ s and then off to transmit any returns for a significantly longer time compared to $\tau$.
\section{Staggered PRT}
The problem with a single PRT is the so-called Doppler dilemma. A short PRT provides great unambiguous Doppler
velocity before the signal aliases, while a long PRT gives us a long unambiguous range. Neither gives us both.
To fix this problem, some radars will alternate between a long and a short pulse, and use one for determining
purely reflectivity, and both for determining Doppler velocity.

The ratio between both PRTs is chosen and known beforehand, so the way that velocities alias is also known. 
The velocities received from both PRTs can then be used to determine the true velocity, such that the 
unambiguous velocity is $v_a = av_1$.

The disadvantage is a longer scan time, since every sector now has to be scanned twice, once with the short
and once with the long pulse, doubling the time between sweeps. For some applications, this is not an acceptable
trade-off.


\section{Coded Pulse}
\section{Pulse Compression}



