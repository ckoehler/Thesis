This is the background

\section{Pulse Compression}
\label{pc}
The goal for radar signals is high sensitivity, which means a high signal-to-noise ratio (SNR), 
and high resolution. In order to get a high SNR, the more energy can be put on target, the better
the sensitivity. For a radar operating in transmitter saturation, that means the longer the
transmitted pulse is, the more energy is transmitted.  Unfortunately, pulse length is inverse 
proportional to the resolution, so the longer the pulse, the larger the resolution volume will be.
Pulse compression is a way to decouple the relationship between SNR and resolution by controlling 
the duration and bandwidth separately.

The use of matched filter allows pulse compression since its output is the autocorrelation function 
of the transmit signal. This allows the design of waveforms such that they are narrowly focused in a 
narrow main lobe, with low sidelobes to concentrate the signal energy in the main lobe (\cite{Richards:2005}).

\section{Linear Frequency Modulation}
\label{s_lfm}
In Pulse Compression, the frequency inside the pulse is modulated linearly (for Linear FM, or LFM) or
non-linearly over a certain bandwidth. It is defined as:
\begin{equation}
\label{LFM}
x(t) = e^{j \pi \beta t^2 } = e^{j \theta (t)},\ \ 0 \leq t \leq \tau
\end{equation}

The spectrum of such a LFM waveform approaches the shape of a rectangle that becomes better defined
as the $\beta\tau$ ratio increases. The matched filter response of a rectangular waveform is a sinc 
function with Rayleigh resolution of approximately $1/\beta$ (\cite{Richards:2005}).


The result is an increase in the ambiguous range $r_a$.

The LFM effect is a consequence of a Ambiguity Function property, as follow:

If $u(t)$ has an added phase that changes quadratically, i.e. $u^{'}(t) = u(t) e^{j\pi \beta t^2}$, which 
is equivalent to a linear frequency modulation because $\omega(t) = \frac{d}{dt}\phi(t)$, the resulting 
AF is $|\chi(\tau,f_d-k\tau)|$.


TODO: figures


The slow drop-off of peak in the AF makes LFM pulse compression very Doppler tolerant, which make it convenient 
for radar applications that have to deal with high-velocity targets. Add to that the ease of implementation in 
analog hardware and we have a simple way to improve our radar's performance.

\section{Non-linear Frequency Modulation}
Extrapolating from the results above, the general expression for a frequency modulated waveform is as follows.

\begin{equation}
\label{NLFM}
x(t) = e^{j 2 \pi \int_0^t{f(t') dt'}} ,\ \ 0 \leq t \leq \tau,
\end{equation}

where $f(t)$ defines the function of the frequency sweep. In the LFM case, $f(t)=\beta t$, which simplifies Eq
\ref{NLFM} to

\begin{equation}
\label{NLFM_simp}
x(t) = e^{j 2 \pi \int_0^t{f(t') dt'}} =  e^{j 2 \pi \frac{\beta t^2}{2}} = e^{j \pi \beta t^2},
\end{equation}

which validates our initial claim that a quadratic phase modulation corresponds to a linear frequency change.


\section{Pulse Compression with binary codes}
One alternative to \ref{s_lfm} is the use of binary, or in our case, quadrature codes. With it, a subsection of 
each pulse is shifted in phase by $e^{jk \pi}$, where $k=(0, \frac{1}{2}, 1, \frac{3}{2})$. 
\section{Binary Phase Codes compared to Frequency Modulation}

\section{Bandwidth for Binary Phase Codes}
The bandwidth of a binary phase-modulated waveform is the inverse of the length of a subpulse. Given a
pulse with length $\tau$ and $n$ subpulses, the length of a subpulse is $T = \frac{\tau}{n}$, the the bandwidth
of the whole pulse will be $B = \frac{1}{T}$ Hz.

\section{Use of Phase Codes in Profilers}
why are they using it?

\section{Military Radars' use of Frequency Modulation} 
Since LFM is simply taking advantage of a property of the AF, it was the earliest way of getting increased sensitivity.
It's fairly simple to implement in analog hardware and very Doppler tolerant due to the slow decrease in slope.

\section{Amplitude Tapering compared to Non-linear Frequency Modulation}

\section{Doppler Effect for Weather Radars}
One question for pulse compression performance is the effect of a Doppler shift on matched filter in the receiver. Any
received signal that has been affected by a Doppler shift will cause a mismatch in the matched filter of the receiver.
This is especially pronounced the higher the target velocity is. For atmospheric weather radars, we observe targets that
usually move 25 m/s at maximum, causing a Doppler shift of $f_d = \frac{2v}{\lambda} = \frac{50}{0.1} = 500\ Hz$ for S-band.

This mismatch effect on the matched filter has been quantified by Kelly and Wishner as follows (\cite{Kelly:1965}).

For a signal of length T, a target of velocity v, and signal bandwidth B, the return Doppler-shifted signal can be written 
as $T(1-\frac{2v}{c})=T-\Delta T$. If $\Delta T$ is much smaller than $T$, i.e.  $\Delta T << \frac{1}{B}$, the mismatch 
due to the Doppler effect can be neglected. In other words, $\frac{2v}{c}BT << 1$. For a $1\mu s$ pulse at 5 MHz bandwidth 
moving at 30 m/s, $0.001 << 1$ is true, so the Doppler effect can be ignored. We will see this confirmed later in the 
simulations.

The pulse compression ratio, $BT$, is not equal to unity, so the mismatch in the filter is compounded for highly compressed 
pulses, which also makes sense intuitively.
