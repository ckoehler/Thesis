The basic elements of a waveform that can be manipulated are its amplitude, phase, frequency, and timing.  Each of these
components can be designed to adhere to the required specifications, e.g., duty cycle, aliasing velocity and
range, resolution, sensitivity, etc.,  and will be described in more detail in this chapter. Figure \ref{fig:basic_waveform} can be used as a reference.

\begin{figure}[h]
\includegraphics[width=\imgsize]{figures/basic_waveform_f.png}
\caption{A basic waveform showing amplitude, period, phase, pulse length, and PRT. Each component will be explained in
this chapter.}
\label{fig:basic_waveform}
\end{figure}

\section{Pulse Repetition Time}
The pulse repetition time (PRT) is the time between two consecutive pulses and is often denoted as $T_s$.  Another way
to describe the timing of consecutive pulses is the pulse repetition frequency (PRF), which is simply the inverse of the
PRT. Both are used in literature to describe the same concept.

The PRT determines two very important aspects of a radar, the \emph{aliasing range} $r_a$ and the \emph{aliasing
velocity} $v_a$. They are given by the following equations \citep[p. 60-61]{Doviak:2006}.

\begin{equation}
\label{eq:unambrange}
r_\mathrm{a} = \frac{c T_\mathrm{s}}{2},
\end{equation}
where $ c = \SI{3e8}{\metre\per\second}$ is the speed of light, and

\begin{equation}
  \label{eq:unambvelocity}
  v_\mathrm{a} = \pm \frac{\lambda}{4 T_\mathrm{s}},
\end{equation}
where $\lambda$ is the wavelength of the radar.

The aliasing range describes the maximum range of the radar after which range aliasing occurs, i.e., targets that are
farther away than the aliasing range will appear to be closer to the radar than they truly are. Referring to
Figure \ref{fig:rangealiasing}, for example, if $r_\mathrm{a} = \SI{100}{\kilo\metre}$, and there is a target at a distance of
\SI{150}{\kilo\metre} from the radar, it will appear to the radar to be only \SI{50}{\kilo\metre} away.  To increase
$r_\mathrm{a}$, a longer PRT is needed.

\begin{figure}[H]
\includegraphics[width=\imgsizes]{figures/rangealiasing.png}
\caption{Diagram explaining the aliasing range. Returned signals from a target at \SI{150}{\kilo\metre}, which is
outside the aliasing range, fold over ambiguously and appear as a target at \SI{50}{\kilo\metre} w.r.t. the second pulse. }
\label{fig:rangealiasing}

\includegraphics[width=\imgsizes]{figures/velocityaliasing.png}
\caption{Diagram explaining the detected velocity. Targets that move at velocities outside the aliasing velocity will fold over
and cause ambiguity, which may be slower and/or in opposite direction. }
\label{fig:velocityaliasing}
\end{figure}

The aliasing velocity is the maximum velocity of a target that the radar can determine without aliasing to occur. The
principle is similar to that of aliasing range. Figure \ref{fig:velocityaliasing} illustrates an example of a target's
velocity ``wrapping around'' and being detected wrongly. With $v_\mathrm{a}= \SI{50}{\metre\per\second}$, a target moving at
\SI{75}{\metre\per\second} will appear to only be moving at \SI{25}{\metre\per\second} in the opposite direction. The
reversal of the target's direction alone is an indicator of how undesirable velocity aliasing is.


One important thing to notice is that, while the aliasing range is proportional to the PRT, aliasing velocity is
inversely proportional to it.  That means that, if the PRT is long, the aliasing velocity will be low and velocity
aliasing will be more likely. On the other hand, if the PRT is short, the aliasing range will be low as well.  The
resulting trade-off is called the \emph{Doppler dilemma}, which states that, for a given PRT, one can either achieve high
aliasing range or high aliasing velocity, but not both \citep[p.  60-61]{Doviak:2006}

\section{Staggered PRT}
As mentioned earlier, the problem with a single PRT is the so-called Doppler dilemma. A short PRT provides high aliasing
velocity, while a long PRT gives us a long aliasing range. Neither gives us both.  One way around this limitation is by
using two different PRTs, a long and a short one, and combine measurements of both to increase aliasing velocity. The
concept of using two PRTs is shown in Figure \ref{fig:sprt}.  Typically, the long PRT is chosen based on the desired
aliasing range, while the short PRT is chosen such that their combination maximizes the aliasing velocity. The aliasing
velocity of two staggered pulses of different lengths is defined as \citep[p. 173]{Doviak:2006}:

\begin{equation}
  \label{eq:vm}
  v_\mathrm{m} = \pm \frac{\lambda}{4(T_{\mathrm{s2}} - T_{\mathrm{s1}})}
\end{equation}

The disadvantage of this method is a longer scan time, since every radial now has to be scanned twice, once with the short and once
with the long pulse, doubling the time between sweeps. For some applications, this is not an acceptable trade-off.

\begin{figure}[H]
\includegraphics[width=\imgsize]{figures/sprt.png}
\caption{For the Staggered PRT technique, two different PRTs are used to achieve sufficient aliasing range and velocity.}
\label{fig:sprt}
\end{figure}

\section{Inter-pulse Coding}
In order to distinguish return signals from differetn pulses, one can encode the transmit phase of each pulse. That way,
an echo from the first pulse that is received after the second pulse has already been transmitted can be identified to
belong to the first pulse and thus distinguished from echoes caused by the second pulse.

This allows the use of a small PRT, resulting in good aliasing velocity, but also the ability to unfold second, third,
or even fourth trip echoes correctly. One such implementation is called SZ phase codes.  For this technique, every other
pulse is phase shifted by a certain sequence $a_k = e^{j \Psi_k}$. On receive, the signal can be decoded by multiplying
it with the complex conjugate of the coding sequence, $a^*_k$. This will make the first-trip signal coherent, leaving
the rest of the signals with different phase shifts. To make the second-trip echos coherent, the return signal is
multiplied by $a^*_{k-1}$, which will again leave the other trip signals encoded. This can be repeated for the other
trip signals in the same way. In this manner, each trip signal can be made coherent and thus separated from the others.
For more details, refer to \cite{Sachidananda:1999}.


\section{Pulse Length}
The previous techniques were independent of the pulse length. However, pulse length, usually
denoted by $\tau$, does affect the performance of the radar.  For example, by using a longer pulse, the target is
illuminated by more energy, thus increasing the sensitivity of the radar. On the other hand, a longer pulse will also
decrease range resolution, which is defined as

\begin{equation}
\label{eq:rangeres}
\Delta r = \frac{c \tau}{2}.
\end{equation}
However, depending on the application, simply increasing the pulse length to increase sensitivity may not be a desirable
solution.

The bandwidth of pulse is partiallly determined by its shape. Taking the Fourier transform of the pulse yields its frequency
spectrum. The spectrum of a perfectly rectangular pulse, shown in Figure \ref{fig:pulsespectrum}, looks like a sinc
function with sidelobes that extend into infinity, so its bandwidth is infinite bandwidth, making it impossible to
generate in a real system. In practice, Gaussian pulse shapes are used because they can be created more easily and still
have the correct bandwidth requirement. For such a pulse, bandwidth and pulse length are inverses of each other such
that $B=\frac{1}{\tau}$. In this case, then, the aliasing range given in Equation \eqref{eq:rangeres} can be
rewritten as:

\begin{equation}
\label{eq:rangeresbwpulse}
  \Delta r = \frac{c}{2B}
\end{equation}

\begin{figure}[h]
\includegraphics[width=\imgsize]{figures/pulsespectrum.png}
\caption{The frequency spectrum of a rectangular pulse. The sidelobes are cut off in the figure, but extend into
infinity.}
\label{fig:pulsespectrum}
\end{figure}

\section{Transmitters}
\label{s:transmitters}
The transmitter dictates the amplitude and length of the pulse, determining the energy that is
transmitted onto the target.  This energy is usually expressed in terms of the radar's average power,which is an
important metric that relates the amplifier's peak power, pulse length, and pulse repetition time, as described in \cite[p.
74]{Skolnik:2001} and shown as Equation \eqref{eq:avgpower}.

\begin{equation}
  \label{eq:avgpower}
  P_{\mathrm{av}} = \frac{P_\mathrm{t} \tau}{T_\mathrm{s}},
\end{equation}
The goal is to maximize $P_{\mathrm{av}}$ while maintaining $T_\mathrm{s}$ for a desired maximum range.  The most
commonly used transmitters are Klystrons, Magnetrons, Traveling Wave Tubes (TWT), and Solid-State Transistor Amplifiers.
Klystrons are capable of high average and peak power with good gain and efficiency, and wide bandwidth \citep[p.
692]{Skolnik:2001}.  However, to achieve large peak powers, high voltages and X-ray shielding are required, limiting
klystrons to mostly stationary radars that have access to a high voltage power source. Klystrons are Linear-Beam Power
Tubes in which electrons are emitted from a cathode, formed into a beam, and shot towards a collector. Between the
cathode and the collector are RF cavities at which the electrons interact with an input signal by amplifying it
\citep[pp. 694-695]{Skolnik:2001}. The amplified signal is then delivered to the output cavity and further down the
transmit chain of the radar.

TWTs are similar to klystrons in terms of architecture, but power and bandwidth are not as high as those of a
klystron.  Magnetrons are not technically amplifiers but oscillators. They are smaller than klystrons and operate at
lower voltages, but have poor noise and stability performance \citep[p. 693]{Skolnik:2001}.  Lastly, Solid-State
Transistor Amplifiers, which have been described briefly in Section \ref{s:par}, operate at lower
voltages and have much lower gain than klystrons or magnetrons. On the other hand, they are very reliable, easy to
maintain, and have long life. They are also capable of wider bandwidth than klystrons or magnetrons \citep[p.
702]{Skolnik:2001}. If the long duty cycles and related need for longer pulses is acceptable, they are a good choice for
powering radars, especially in situations where size restricts the use of the other transmitters that have been
discussed.


\section{Matched Filter}
\label{s:2pc}
At the heart of pulse compression is the matched filter, which draws its name from the fact that it is matched to the
transmitted waveform. As the waveform passes through the filter, it is compressed in range and will ideally concentrate
most of the energy in a narrow mainlobe with no or low sidelobes. 
The matched filter is designed based on the maximization of the SNR. The following derivation traces the discussion
given in \cite[pp. 24-27]{Levanon:2004}.

Specifically, the SNR to be maximized is given by

\begin{equation}
  \label{eq:mfsnr}
  \left(\frac{S}{N}\right)_{\mathrm{out}} = \frac{|s_\mathrm{o}(t_\mathrm{0})|^2}{n_\mathrm{o}^2},
\end{equation}
where $s_\mathrm{o}$ is the output signal of the filter at a certain delay $t_\mathrm{0}$, and $n_\mathrm{o}$ is the Gaussian output noise. The
variables influencing the output signal are the original signal $s(t)$ with delay $t_\mathrm{0}$. The whole output
signal is then given by:

\begin{equation}
  \label{eq:mfso}
  s_\mathrm{o}(t_\mathrm{0}) = \frac{1}{2 \pi} \int_{-\infty}^{\infty} H(\omega) S(\omega) \exp(j \omega t_\mathrm{0}) d\omega,
\end{equation}
where $S(\omega)$ is the Fourier transform of $s(t)$ and $H(\omega)$ the Fourier transform of $h(t)$, the impulse response
of the matched filter that needs to be derived. The exponential describes the delay $t_\mathrm{0}$ in the frequency domain.
The total output noise power $n_\mathrm{o}^2$ for white noise interference with spectral density $N_\mathrm{0}/2$ is described by:

\begin{equation}
  \label{eq:mfnoise}
  n_\mathrm{o}^2 = \frac{1}{2 \pi}\frac{N_\mathrm{0}}{2} \int_{-\infty}^{\infty} |H(\omega)|^2 d\omega.
\end{equation}

Before continuing this derivation, the \emph{Schwarz inequality} must be mentioned. It states that:

\begin{equation}
  \label{eq:schwarz}
  \left|\int_{-\infty}^{\infty} A(\omega) B(\omega) d\omega \right|^2 \leq 
    \int_{-\infty}^{\infty} |A(\omega)|^2 d\omega \int_{-\infty}^{\infty} |B(\omega)|^2 d\omega,
\end{equation}
for two complex signals $A(\omega)$ and $B(\omega)$. The two sides are exactly equal if and only if the following holds
true: 

\begin{equation}
  \label{eq:schwarzcond}
  A(\omega) = K B^*(\omega),
\end{equation}
where $K$ is an arbitrary constant.
Substituting Equations \eqref{eq:mfso} and \eqref{eq:mfnoise} back into Equation \eqref{eq:mfsnr} and applying the
Schwarz inequality with:

\begin{equation}
  \label{eq:schwarzsubst}
  A(\omega) = H(\omega) , B(\omega) = S(\omega)\exp(j \omega t_\mathrm{0}), 
\end{equation}
yields:

\begin{equation}
\label{eq:mfsnrmax}
  \left(\frac{S}{N}\right)_{\mathrm{out}} \leq \frac{1}{\pi N_\mathrm{0}} \int_{-\infty}^{\infty} |S(\omega)|^2 d\omega
  = \frac{2E}{N_\mathrm{0}},
\end{equation}
where $E$ is the energy of the signal, given by:

\begin{equation}
  \label{eq:energy}
  E = \int_{-\infty}^{\infty} s^2(t) dt = \frac{1}{2 \pi} \int_{-\infty}^{\infty} S^2(\omega) d\omega.
\end{equation}

Substituting \eqref{eq:schwarzsubst} into \eqref{eq:schwarzcond} shows that the maximum output SNR is achieved when the
two sides of Equation \eqref{eq:mfsnrmax} are equal, i.e.:
\begin{equation}
  \label{eq:mffr}
  H(\omega) = K S^*(\omega) \exp(-j \omega t_\mathrm{0}).
\end{equation}
Taking the inverse Fourier transform produces the impulse response of the matched filter:
\begin{equation}
  \label{eq:mfir}
  h(t) = K s^*(t_\mathrm{0}-t)
\end{equation}
This reveals that the impulse response of the matched filter is the delayed mirror image of the conjugate of the
original signal $s(t)$. When the signal and filter are matched in this way, the condition of the Schwarz inequality
in Equation \eqref{eq:schwarzcond} is fulfilled and the SNR at $t=t_\mathrm{0}$ is maximized with
$\mathrm{SNR}=\frac{2E}{N_\mathrm{0}}$. This result shows that the ideal SNR is not dependent on any component of the
pulse but its total energy.

With $h(t)$ known, the output of the matched filter is then 
\begin{equation}
  \label{eq:filterresponse}
  s_\mathrm{o}(t_\mathrm{0}) = s(t) \star h(t) = \int_{-\infty}^{\infty}{s(\tau_\mathrm{d}) K s^*[t_\mathrm{0} - (t -
  \tau_\mathrm{d})] d\tau_\mathrm{d}},
\end{equation}
where $\tau_\mathrm{d}$ is the range delay due to propagation.
Choosing the arbitrary gain constant $K=1$ and the predetermined delay $t_\mathrm{0}=0$, the output can be recognized as the
autocorrelation function (ACF) of $s(t)$:
\begin{equation}
  \label{eq:acf}
  s_\mathrm{o}(0) = \mathrm{ACF}(s(t)) = \int_{-\infty}^{\infty}{s(\tau_\mathrm{d}) s^*(\tau_\mathrm{d}-t)
  d\tau_\mathrm{d}}.
\end{equation}
The concept of the ACF for a rectangular pulse is shown in Figure \ref{fig:acf}.
This is an important function because it describes the output of the radar depending on the waveform $s(t)$ used. It is
the basis of analyzing the performance of radar signals in this work. As discussed before, the best result is achieved
when the filter matches the waveform exactly. If that is not the case and there is a mismatch between the filter and the
signal, there will be some loss called \emph{mismatch loss}. One reason for mismatch loss is the Doppler phase shift applied
to the return signal by a moving target. Adding that phase shift, given by $e^{-j 2 \pi f_\mathrm{d} t}$, where
$f_\mathrm{d}$ is the
Doppler frequency, leads to the main tool of evaluating waveforms in this work, the \emph{Ambiguity
Function}, or AF.


\begin{figure}[h]
\includegraphics[width=\imgsize]{figures/acf.png}
\caption{ACF of a rectangular pulse. Similar to the convolution, the output of the matched filter of a
rectangular pulse is a triangle. The height of the triangle at any given point is determined by the overlap of the
signal with the matched filter response, so that it reaches its maximum when both fully overlap.}
\label{fig:acf}
\end{figure}


\section{Ambiguity Function}
\label{s:af}
In \cite{Rihaczek:1968}, the AF is defined as:

\begin{equation}
\label{eq:AF}
  |\chi(\tau_\mathrm{d},f_\mathrm{d})| = \left| \int_{-\infty}^\infty{s(t) s^*(t+\tau_\mathrm{d}) e^{-j2\pi f_\mathrm{d} t} dt}\right|,
\end{equation}
where $s(t)$ is the complex envelope of the signal. It can clearly be seen that, for $f_\mathrm{d} = 0$, the
AF reduces to the ACF, presented in Section \ref{s:2pc}.


For a simple rectangular pulse of the form $s(t)=\frac{1}{\sqrt{\tau}} \text{rect}(\frac{t}{\tau})$, the AF is \citep{Levanon:2004}:

\begin{align*}
  |\chi(\tau_\mathrm{d},f_\mathrm{d})| &= \left| \int_{-\infty}^\infty{s(t) s^*(t+\tau_\mathrm{d}) e^{j2\pi f_\mathrm{d} t} dt}\right| \\ 
  &= 
  \begin{cases}
    \frac{1}{\tau} \int_{-\frac{\tau}{2} + \tau_\mathrm{d}}^{\frac{\tau}{2}}{e^{j2\pi f_\mathrm{d} t}dt} & \text{for $0
    \leq \tau_\mathrm{d} \leq \tau$}, \\
    \frac{1}{\tau} \int_{-\frac{\tau}{2}}^{\frac{\tau}{2} + \tau_\mathrm{d}}{e^{j2\pi f_\mathrm{d} t}dt} & \text{for
    $-\tau
    \leq \tau_\mathrm{d} < 0$}, \\
    0 & \text{otherwise}. \\
  \end{cases}
\end{align*}

Evaluate the integrals to get:

\begin{align}
  |\chi(\tau_\mathrm{d},f_\mathrm{d})| &= 
  \begin{cases} 
  \left|\left(1-\frac{|\tau_\mathrm{d}|}{\tau}\right) \frac{\sin[\pi \tau f_\mathrm{d}(1-
  \frac{|\tau_\mathrm{d}|}{\tau})]}{\pi \tau f_\mathrm{d}(1-
  \frac{|\tau_\mathrm{d}|}{\tau})}\right| & \text{for $|\tau_\mathrm{d}| \leq \tau$}, \\
  0 & \text{otherwise}.
  \end{cases}
\end{align}
For $\tau_\mathrm{d}=0$, the result is the $\mathrm{sinc}$ function, for $f_\mathrm{d}=0$ it is simply the ACF of a pulse, which is
a triangle.

The ideal shape of the AF depends on the application. One useful AF looks like a thumbtack and thus provides good range
and Doppler resolution.  However, this also means that any moving target will result in a weak or, for the ideal case,
no return signal due to the Doppler shift. A waveform with this characteristic can be used to detect targets whose velocity is already known,
since even a narrowly focused AF covers a certain range of Doppler velocities without much loss. As will be shown in the
simulations, this is the case for weather radar applications.

If a wider range of velocities has to be covered, several
different matched filters can be used, with each one matched to a different Doppler velocity range.  The reader is
referred to Chapter 7.2 of \citep{Levanon:2004} for more information. Alternatively, the AFs of some waveforms are
shaped like a ridge, so there will be a return signal no matter the Doppler shift. Chapter \ref{chap:pulsecompression} will
introduce one such waveform.




