\section{Pulse Repetition Time}
The pulse repetition time (PRT) is the time between two consecutive pulses and determines the unambiguous
range of the radar. The higher the PRT, the longer the radar waits in receive mode to listen for echoes.
A low PRT can lead to aliasing of second trip echoes, i.e. echoes that are caused by the first pulse, but
are received after the second pulse has already been sent. To achieve a high range, we need as long a PRT
as possible.

While the unambiguous range is proportional to the PRT, unambiguous velocity is inversely proportional to it.
That means the higher the PRT, the lower the unambiguous velocity, and the more likely we will see velocity
aliasing.
The resulting problem is called the Doppler dilemma, which states that one can either achieve high unambiguous
range or high unambiguous velocity, but not both.


\section{Staggered PRT}
The problem with a single PRT is the so-called Doppler dilemma. A short PRT provides great unambiguous Doppler
velocity before the signal aliases, while a long PRT gives us a long unambiguous range. Neither gives us both.
To fix this problem, some radars will alternate between a long and a short pulse, and use one for determining
purely reflectivity, and both for determining Doppler velocity.

The ratio between both PRTs is chosen and known beforehand, so the way that velocities alias is also known. 
The velocities received from both PRTs can then be used to determine the true velocity. Given the unambiguous 
velocities ratio $2 v_{a1} = 3 v_{a2}$, the unambiguous velocity is $v_a = 2 v_{a1} = 3 v_{a2}$, and higher than
the unambiguous velocity of a single PRT.

The disadvantage is a longer scan time, since every radial now has to be scanned twice, once with the short
and once with the long pulse, doubling the time between sweeps. For some applications, this is not an acceptable
trade-off.


\section{Pulse Coding}
A coded pulse is in some way modulated, so it's not just rectangular. The idea is to code subsequent pulses
differently, such that the received echoes can be paired up with a certain pulse. That way, an echo from the first
pulse that is received after the second pulse has already been transmitted can be seen to belong to the first pulse
and thus distinguished from echoes caused by the second pulse.

This allows the use of a small PRT, resulting in good unambiguous velocity, but also the ability to unfold second, third,
or even fourth trip echoes correctly. The advantage to the Staggered PRT approach is that the scan time doesn't increase, 
since each radial is still only scanned once.


\section{Pulse Length}
The previous techniques were independent of the pulse's length, denoted by $\tau$. However, the pulse length of a 
rectangular pulse determines the bandwidth of the signal. For a simple pulse, the bandwidth B is generally the inverse of
the pulse length $\tau$. Intuitively, for a larger $\tau$, more energy is put on the target. However, a larger pulse
length also decreases range resolution, so it's not a solution for increasing SNR.


\section{Pulse Compression}
We need some way to decouple the pulse length from the bandwidth to get both good SNR and fine range resolution.
Pulse compression does just that.
In the case of a phase-coded pulse, the bandwidth is determined by the length of each sub-pulse.
For frequency sweeps, or chirps, the bandwidth is determined by the frequency range of the chirp. 



