\section{Pulse Repetition Time}

The pulse repetition time (PRT) is the time between two consecutive pulses and determines the unambiguous
range of the radar which is defined as

\begin{equation}
\label{eq:unambrange}
r_a = \frac{c T_s}{2},
\end{equation}

where $c$ is the speed of light and $T_s$ the PRT \citep[p. 60f]{Doviak:2006}.

The higher the PRT, the longer the radar waits in receive mode to listen for echoes.
A low PRT can lead to aliasing of second trip echoes, i.e. echoes that are caused by the first pulse, but
are received after the second pulse has already been sent. In order to scan out to long range, a long PRT
is needed.

While the unambiguous range is proportional to the PRT, unambiguous velocity is inversely proportional to it, i.e.

\begin{equation}
  \label{eq:unambvelocity}
  v_a = \pm \frac{\lambda}{4 T_s},
\end{equation}
where $\lambda$ is the wavelength of the radar.
That means the higher the PRT, the lower the unambiguous velocity, and the more likely we will see velocity
aliasing.
The resulting problem is called the Doppler dilemma, which states that one can either achieve high unambiguous
range or high unambiguous velocity, but not both \citep[p. 60f]{Doviak:2006}


\section{Staggered PRT}
As mentioned earlier, the problem with a single PRT is the so-called Doppler dilemma. A short PRT provides high
unambiguous Doppler
velocity, while a long PRT gives us a long unambiguous range. Neither gives us both.
To fix this problem, some radars will use two different PRTs, a long and a short one, and combine measurements of 
both to increase unambiguous velocity.
Typically, the long PRT is chosen based on the desired unambiguous range, while at the same time maximizing
the unambiguous velocity of the combination of both, defined as \citep[p. 173]{Doviak:2006}
\begin{equation}
  \label{eq:vm}
  v_m = \pm \frac{\lambda}{4(T_{s2} - T_{s1})}
\end{equation}

The disadvantage is a longer scan time, since every radial now has to be scanned twice, once with the short
and once with the long pulse, doubling the time between sweeps. For some applications, this is not an acceptable
trade-off.

TODO: Loss of sensitivity?

\section{Inter-pulse Coding}
A coded pulse is in some way modulated, so it's not just rectangular. The idea is to code subsequent pulses
differently, such that the received echoes can be paired up with a certain pulse. That way, an echo from the first
pulse that is received after the second pulse has already been transmitted can be seen to belong to the first pulse
and thus distinguished from echoes caused by the second pulse.

This allows the use of a small PRT, resulting in good unambiguous velocity, but also the ability to unfold second, third,
or even fourth trip echoes correctly. For one such implementation, called SZ phase codes, see \cite{Sachidananda:1999}.

\section{Pulse Length}
The previous techniques were independent of the pulse length. However, the pulse length of a 
rectangular pulse determines the bandwidth of the signal. For a simple pulse, the bandwidth $B$ is generally the inverse of
the pulse length $\tau$. Intuitively, by using a longer pulse, the target is illuminated by more energy, thus increasing
the sensitivity of the radar. However, a longer pulse
also decreases range resolution. Depending on the application, such a solution may not be desirable.

\section{Amplifiers}
The amplifier will influence what kind of pulse lengths and pulse repetition times are possible and determine
how much power is put out. The radar's average power is an important metric that relates amplifier peak power, pulse
length, and pulse repetition time, as shown in \cite[p. 74]{Skolnik:2001} and shown as Equation \ref{eq:avgpower}.

\begin{equation}
  \label{eq:avgpower}
  P_{av} = \frac{P_t \tau}{T_s},
\end{equation}

The goal is to output as much power as possible to improve the maximum range of the radar and signal-to-noise ratio.

Another kind of amplifier, the low-noise amplifier (LNA) in the receive chain, also influences the sensitivity of the
radar by amplifying the return signal while at the same time adding as little additional noise as possible.


\section{Pulse Compression}
We need some way to decouple the pulse length from the bandwidth to get both good SNR and fine range resolution.
Pulse compression does just that.
In the case of a phase-coded pulse, the bandwidth is determined by the length of each sub-pulse.
For frequency sweeps, or chirps, the bandwidth is determined by the frequency range of the chirp. 



