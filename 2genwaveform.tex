Before going into more detail about pulse compression, the general components of waveforms and their design
will be discussed. The basic elements of a waveform that can be manipulated are its amplitude, phase,
frequency, and timing. Each of these components can be designed to adhere to the required specifications, and
will be described in more detail in this chapter.

\section{Pulse Repetition Time}

The pulse repetition time (PRT) is the time between two consecutive pulses and is often denoted as $T_s$.
Another way to describe the timing of consecutive
pulses is the pulse repetition frequency (PRF), which is simply the inverse of the PRT. Both are used in literature
to describe the same concept.

The PRT determines two very important aspects of a radar, the \emph{unambiguous range} $r_a$ and the \emph{unambiguous velocity}
$v_a$. They are given by the following equations \citep[p. 60f]{Doviak:2006}.

\begin{equation}
\label{eq:unambrange}
r_a = \frac{c T_s}{2},
\end{equation}
where $ c = \SI{3e8}{\metre\per\second}$ is the speed of light, and

\begin{equation}
  \label{eq:unambvelocity}
  v_a = \pm \frac{\lambda}{4 T_s},
\end{equation}
where $\lambda$ is the wavelength of the radar.

The unambiguous range describes the maximum range of the radar up to which no second trip echoes occur.
A second trip echo is the return signal caused by the first pulse, but is received after the second pulse
has already been sent. For example, if $r_a = \SI{100}{\kilo\metre}$, and there is a target at a distance
\SI{150}{\kilo\metre} from the radar, it will appear to the radar to be only \SI{50}{\kilo\metre} away.
To increase the unambiguous range, then, a long PRT is needed.

The unambiguous velocity is measure of the maximum velocity that the radar can determine without velocity aliasing
to occur. The principle is similar to that of unambiguous range. Velocities ``wrap around'', so with $v_a= \SI{50}{\metre\per\second}$, 
a target moving at \SI{75}{\metre\per\second} will appear to only be moving at \SI{25}{\metre\per\second}.

However, while the unambiguous range is proportional to the PRT, unambiguous velocity is inversely proportional to it.  
That means that, if the PRT is large, the unambiguous velocity will be low and velocity aliasing will become more likely, but
if the PRT is low, unambiguous range is low as well.
The resulting problem is called the Doppler dilemma, which states that one can either achieve high unambiguous
range or high unambiguous velocity, but not both \citep[p. 60f]{Doviak:2006}


\section{Staggered PRT}
As mentioned earlier, the problem with a single PRT is the so-called Doppler dilemma. A short PRT provides high
unambiguous Doppler
velocity, while a long PRT gives us a long unambiguous range. Neither gives us both.
To fix this problem, some radars use two different PRTs, a long and a short one, and combine measurements of 
both to increase unambiguous velocity.
Typically, the long PRT is chosen based on the desired unambiguous range, while at the same time maximizing
the unambiguous velocity of the combination of both, defined as \citep[p. 173]{Doviak:2006}
\begin{equation}
  \label{eq:vm}
  v_m = \pm \frac{\lambda}{4(T_{s2} - T_{s1})}
\end{equation}

The disadvantage is a longer scan time, since every radial now has to be scanned twice, once with the short
and once with the long pulse, doubling the time between sweeps. For some applications, this is not an acceptable
trade-off.

\section{Clutter Filtering}


\section{Inter-pulse Coding}
A coded pulse is in some way modulated, so it is not just rectangular. The idea is to code subsequent pulses
with a different phase, such that the received echoes can be paired up with a certain pulse. That way, an echo from the first
pulse that is received after the second pulse has already been transmitted can be identified to belong to the first pulse
and thus distinguished from echoes caused by the second pulse.

This allows the use of a small PRT, resulting in good unambiguous velocity, but also the ability to unfold second, third,
or even fourth trip echoes correctly. One such implementation is called SZ phase codes.
For this technique, every other pulse is phase shifted by a certain sequence $a_k = e^{j \Psi_k}$. On receive, the signal
can be decoded by multiplying it with the complex conjugate of the coding sequence, $a^*_k$. For more details, refer to \cite{Sachidananda:1999}.

\section{Pulse Length}
The previous techniques were independent of the pulse length. However, pulse length, usually denoted by $\tau$, does affect
the performance of the radar.
For example, by using a longer pulse, the target is illuminated by more energy, thus increasing
the sensitivity of the radar. On the other hand, a longer pulse will
also decrease range resolution, which is defined as
\begin{equation}
\label{eq:rangeres}
\Delta r = \frac{c \tau}{2}.
\end{equation}
However, depending on the application, simply increasing the pulse length to increase sensitivity may not be a desirable solution.

\section{Amplitude}
Together with the pulse length, the amplitude of the pulse determines how much energy is transmitted on target.
It depends on the amplifier, which will also affect what kind of pulse lengths and pulse repetition times are possible.
The radar's average power is an important metric that relates amplifier peak power, pulse
length, and pulse repetition time, as described in \cite[p. 74]{Skolnik:2001} and shown as Equation \ref{eq:avgpower}.

\begin{equation}
  \label{eq:avgpower}
  P_{av} = \frac{P_t \tau}{T_s},
\end{equation}

The goal is to output as much power as possible to improve the maximum range of the radar and signal-to-noise ratio.

Another important component is the low-noise amplifier (LNA) in the receive chain, which also influences the sensitivity of the
radar by amplifying the return signal while at the same time adding as little additional noise as possible. Using better
quality amplifiers in a system increases it price together with its performance, but is not always an option, e.g. for legacy
systems that are already operational or simply budget reasons for new systems.


\section{Pulse Compression}
\label{s:2pc}
At the heart of pulse compression is the matched filter. It is called that because its response is matched to the transmitted
waveform. As the waveform passes through the filter, it is compressed in range and will ideally concentrate most of the energy
in a narrow mainlobe with no or low sidelobes. The process of the return waveform passing through the matched filter can be 
described by the cross correlation function of the filter and the return signal
received back at the radar. Ideally this will be identical to the autocorrelation function, i.e., the matched filter response
and the return signal will be identical. Although this is almost never the case, leading to so-called mismatch loss,
this work will only refer to autocorrelation function (ACF).

More formally, then, the matched filter response is defined as the convolution between the signal $u(t)$ and its impulse response,
$h(t)$ \citep{Levanon:2004}.

\begin{equation}
  \label{eq:filterresponse}
  \text{Output} = u(t) \star h(t) = \int_{-\infty}^{\infty}{u(\tau) h(t-\tau) d\tau}.
\end{equation}

Furthermore, since the filter is matched, and maximizing the SNR, the impulse response becomes

\begin{equation}
  h(t) = K u^*(t_0-t),j
\end{equation}
where K is an arbitrary constant. Choosing $t_0=0, K=1$ and substituting back into Equation \eqref{eq:filterresponse} yields
the autocorrelation function.
\begin{equation}
  \label{eq:acf}
  \text{ACF}(t) = \int_{-\infty}^{\infty}{u(\tau) u^*(\tau-t) d\tau}.
\end{equation}

This is an important function that is the basis of analyzing the performance of radar signals in this work.



