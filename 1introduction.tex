\section{Literature Survey}
There are numerous works available on waveform design, the most comprehensive being \cite{Levanon:2004} without
a doubt. It covers the most common waveforms used today, such as pulses, pulse trains, frequency and phase modulated signals, 
as well as continuous wave signals, each accompanied by comprehensive discussion and examples.
\cite{Richards:2005} also discusses radar waveforms with emphasis on frequency modulation 
and pulse trains, but generally focuses more on radar signal processing.
The paper by \cite{Keeler:1999} has much information for weather radar applications, concentrating on 
filter design in phased array radars.

\section{Using Radar for Observing the Atmosphere}
From hard target observation during WWII and the latter part of the 20th century, radar quickly became 
a great way to observe the atmosphere. As with all remote sensing applications, radar could give advanced warning
and better insights into what is happening in the atmosphere in places that are not possible or difficult to observe
using other means, such as localized weather stations or weather balloons, because this single instrument could 
cover a very large area from many miles away. This made dense networks of instruments unnecessary, drastically
lowering maintenance expenses and increasing the area of observation.

A pulsed radar allows for the retrieval of intensity as well as Doppler shift to determine the radial
velocity of weather targets. 

Some wavelengths perform better for observing weather than others, especially in respect to attenuation 
\citep[p. 448]{Skolnik:2001}. The smaller the wavelength, the more attenuated
the signal becomes as it passes through the atmosphere.  The WSR-88D network radars operate at S-band, which
covers frequencies between 2 and 4 GHz. More specifically, the 88D radars operate at between 2700 and 3000 MHz
\citep[p. 47]{Doviak:2006}.
However, S-band may already be reserved and is generally highly desired for other applications,
like cell phones, so C-band or X-band radars are also used for observing weather. In such cases, the effects of attenuation can be
somewhat mitigated by, e.g. using polarimetric radar measurements \citep{Ryzhkov:2008}. 

One advantage of higher frequency
radars is the need for a smaller antenna size for the same angular resolution. Other components such as waveguides,
amplifiers and support structure are also smaller, so the whole system requires less space and is often cheaper to 
manufacture and easier to deploy.

As technology advances, demands on radar performance increase. Finer resolution offers better detail, while
dual polarization radars provide even more data that can be used to better analyze the atmosphere \citep[p. 242]{Doviak:2006}.

\section{Phased Array Radars}
Phased array radars move away from standard dish antennas to so-called antenna-arrays.
An antenna array is made up of discrete units of transceivers distributed in one or more dimensions. The NWRT Phased Array 
Radar, for example, has 4352 elements distributed across a plane \citep{Zrnic:2007}. Each of the elements is able to transmit a
signal with a certain phase shift. Setting these phase shifts in a certain way will result in the signals of all the
array elements to add up in different ways, steering the radar beam without physically moving the antenna.

This electronic steering has the advantage of being very agile and flexible and reduces overall scan time,
focus on particular targets in space in any order, or reduce data correlation by steering the beam out-of-order across
a storm, a technique called beam multiplexing \citep{Yu:2007}.

For an active array of transmitters, such as solid-state amplifiers, the peak power is generally lower, compared to a
Klystron or magnetron. One way to regain the sensitivity lost due to lower transmit power is to transmit longer pulses 
\citep[p. 702]{Skolnik:2001}. Longer pulses, in turn, decrease range resolution,
so a method is needed to improve the signal-to-noise ratio while keeping the range resolution acceptable.

\section{The Need for Pulse Compression}
Range resolution is defined as
\begin{equation}
\label{eq:rangeres}
\Delta r = \frac{c \tau}{2},
\end{equation}
where $c= \SI{3e8}{\metre\per\second}$ represents the speed of light and $\tau$ is the pulse length. Equation
\ref{eq:rangeres} shows that the longer the pulse, the larger the resolution volume will be. 
Pulse compression is a technique to increase the SNR and resolution by using waveforms that allow the duration
and bandwidth to be controlled separately.

Compression is achieved by using one or more matched filters, which will be introduced in Chapter \ref{genwaveform}.
This allows the design of waveforms such that they are narrowly focused in a 
narrow main lobe, with low sidelobes to concentrate the signal energy in the main lobe \citep{Richards:2005}.
The challenge now becomes to design a waveform that has the needed SNR and at the same time good range resolution, 
all the while keeping down sidelobes, and is a procedure of many trade-offs.
