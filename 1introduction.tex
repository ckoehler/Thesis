There are numerous works available on waveform design, the most comprehensive being \cite{Levanon:2004}.  It covers the
most common waveforms used today, such as pulses, pulse trains, frequency and phase modulated signals, as well as
continuous wave signals, each accompanied by comprehensive discussion and examples.  \cite{Richards:2005} also discusses
radar waveforms with an emphasis on frequency modulation and pulse trains, but generally focuses more on radar signal
processing.  The work of \cite{Keeler:1999} has much information for weather radar applications, concentrating on filter
design in phased array radars.  \cite{Mudukutore:1998} discusses the simulation of pulse compression for weather radars
for phase coded signals and different filters, with a focus on sidelobe reduction.

\section{Using Radar for Observing the Atmosphere}
From hard target observation during WWII and the latter part of the
20th century, radar quickly became a potential way to observe the atmosphere. As with all remote sensing applications,
radar could give advanced warning and better insights into what is happening in the atmosphere in places that are not
accessible or difficult to observe using other means, such as localized weather stations or weather balloons, because
this single instrument could cover a very large area remotely. This made dense networks of instruments unnecessary,
drastically lowering maintenance and operation expenses and increasing the area of observation.

The most basic pulsed radar is able to determine the reflectivity of the atmosphere by measuring the return signal of
the radar beam. Additionally, coherent radars can measure the Doppler shift of the return signal and use it to find the
radial velocity of the target, i.e., the velocity component of the target that is directed right towards or away from
the radar. This is achieved with a stable local oscillator (STALO). The STALO signal is split into an in-phase and a 90
degree phase shifted signal, which are then mixed with the return signal and passed through a set of low pass filters that filter
out the undesired frequency components. The two outputs of the filters are called $I$, the in-phase
component, and $Q$, the quadrature component. 
The phase of the $I$ and $Q$ components is sampled over several pulses to determine the Doppler frequency.


Some wavelengths perform better for observing weather than others, especially with respect to attenuation through
hydrometeors \citep[p. 448]{Skolnik:2001}. Specifically, attenuation due to rain depends on the wavelength, as does
attenuation in clouds \citep[pp. 42-43]{Doviak:2006}.  One advantage of higher frequency radars is the possibility of a smaller
antenna size for the same angular resolution, but the smaller the wavelength, the more attenuated the signal becomes as
it passes through the atmosphere.  The WSR-88D network radars operate at S-band, more specifically between 2700 and 3000
MHz \citep[p. 47]{Doviak:2006}.  However, S-band is highly desired for other applications, like cellular phones, so
C-band or X-band radars are also used for observing weather to avoid interference between different systems operating at
the same frequency band.

In such cases, the effects of attenuation can be somewhat mitigated by, e.g., using an attenuation correction technique
that is based on \cite{Ryzhkov:2008}.  Other components such as waveguides, amplifiers and support structure are also
smaller, so the whole system requires less space and is often less expensive to manufacture and easier to deploy.
Another beneficial cost factor is the increased availability of commercial off-the-shelf (COTS) components due to an
increase in consumer RF devices like wireless routers and other networking equipment. These components are mass produced
in high quantities and available at lower costs to the radar manufacturer. One example of taking advantage of COTS
components for Digital Array Radars is given in \cite{Tarran:2008}.

As technology advances, demands on radar performance increase. Finer resolution offers better detail, while dual
polarization radars provide even more data that can be used to better analyze the atmosphere \citep[p.
242]{Doviak:2006}.  This data is derived from the so-called \emph{backscattering covariance matrix}, which relates the
backscattered electric field to the incident electric field. With two orthogonal polarizations, additional parametric
variables are derived,e.g., $Z_{\mathrm{dr}}$, the differential reflectivity, or $\rho_{\mathrm{hv}}$, the correlation coefficient, and
$\Phi_{\mathrm{dp}}$, the differential phase. They are mostly used for hydrometeor classification and Quantitative Precipitation
Estimation (QPE) \citep{Ryzhkov:2008}.

\section{Phased Array Radars}
\label{s:par}
Phased array radars move away from parabolic dish antennas to so-called antenna-arrays.  An antenna array is made up of
discrete units of antennas distributed in space. Each antenna element may function as a receiver and/or transmitter.
The NWRT Phased Array Radar, for example, has 4352 elements distributed across a plane \citep{Zrnic:2007}. Each of the
elements is able to transmit a signal with a different phase shift. The way the phase shifts will result in the signals
of all the array elements to add constructively, steering the radar beam in a certain direction.

This electronic steering has the advantages of being very agile and flexible and can reduce overall scan time, focus on
particular targets in space in any order, or reduce data correlation by steering the beam out-of-order across a storm, a
technique called beam multiplexing (BMX) \citep{Yu:2007}. BMX provides fast updates of weather information with higher
statistical accuracy because collected sample pairs are independent since they have not been collected sequentially, allowing
the atmosphere to decorrelate before taking the next pair of samples in the same area.

Some phased array radars, so-called \emph{active arrays} do not have just one transmitter. Instead, they use one smaller
amplifier in each element directly. These amplifiers are solid-state amplifiers, whose peak power is generally lower
compared to a klystron or magnetron, because they are made with transistors. As such, they operate at lower voltages,
have a lower gain, and require longer duty cycles, but have higher reliability. To achieve the needed power, many of
them can be operated in parallel and in multiple stages as modules \citep{Skolnik:2001}. So, instead of using a klystron
or magnetron, a phased array radar can incorporate amplifier modules into the antenna element to power each one
separately. The advantage of this kind of modular design is the high reliability and fault tolerance of the
system.  Failure of individual elements still allows the radar to operate, and failed elements can simply be replaced
with a new one.  This lowers maintenance and manufacturing costs because all the identical elements can be mass
produced and are often available commercially off the shelf (COTS).

One way to compensate for the lower duty cycle and peak power of solid-state amplifiers is to use longer pulses
\citep[p. 702]{Skolnik:2001}. However, a longer pulse decreases range resolution, so a method is needed to improve the
signal-to-noise ratio (SNR) while keeping the range resolution acceptable. That method is called pulse compression.


\section{The Need for Pulse Compression}
Pulse compression is a technique to increase the SNR by using waveforms that allow the duration and bandwidth to be
controlled separately \citep[pp. 42-43]{Richards:2005}.  A very early paper by Bell Labs introduces the concept of a
Chirp, or a frequency modulated pulse, for exactly this purpose \citep{Klauder:1960}. Since then, other ways of
achieving the same goal have been found and will be discussed in this work. Compression is achieved by using one or more
matched filters, which will be introduced in Chapter \ref{genwaveform}.  This allows the design of waveforms such that
they are focused in a narrow main lobe, with low sidelobes to concentrate the signal energy in the main lobe
\citep[p. 44]{Richards:2005}. 

As mentioned previously, solid-state amplifiers are economical, but have lower peak power and higher duty cycles, so
long pulses are necessary to achieve the same sensitivity. On the one hand, range resolution is important to detect
small features and would suffer severely from longer pulses.
On the other hand, the SNR is directly related to probability of detection, so it is very important to maintain a sufficient SNR.
Furthermore, since weather signals have a large dynamic range, required for detecting anything from light rain or even
just clear air measurements, to heavy rain and hail, a high SNR is absolutely necessary.  

These requirements for good range resolution and SNR make pulse compression necessary. The different ways of
accomplishing pulse compression are discussed in this thesis.

