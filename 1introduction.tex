There are numerous works available on waveform design, the most comprehensive being \cite{Levanon:2004}.
It covers the most common waveforms used today, such as pulses, pulse trains, frequency and phase modulated signals, 
as well as continuous wave signals, each accompanied by comprehensive discussion and examples.
\cite{Richards:2005} also discusses radar waveforms with an emphasis on frequency modulation 
and pulse trains, but generally focuses more on radar signal processing.
The work of \cite{Keeler:1999} has much information for weather radar applications, concentrating on 
filter design in phased array radars.
\cite{Mudukutore:1998} discusses the simulation of pulse compression for weather radars for phase coded signals and
different filters, with a focus on sidelobe reduction.

\section{Using Radar for Observing the Atmosphere}
From hard target observation during WWII and the latter part of the 20th century, radar quickly became 
a potential way to observe the atmosphere. As with all remote sensing applications, radar could give advanced warning
and better insights into what is happening in the atmosphere in places that are not accessible or difficult to observe
using other means, such as localized weather stations or weather balloons, because this single instrument could 
cover a very large area remotely. This made dense networks of instruments unnecessary, drastically
lowering maintenance and operation expenses and increasing the area of observation.

A pulsed radar allows for the retrieval of intensity as well as Doppler shift to determine the radial
velocity of weather targets. 

Some wavelengths perform better for observing weather than others, especially with respect to attenuation 
\citep[p. 448]{Skolnik:2001}. The smaller the wavelength, the more attenuated
the signal becomes as it passes through the atmosphere.  The WSR-88D network radars operate at S-band, which
covers frequencies between 2 and 4 GHz. More specifically, the 88D radars operate between 2700 and 3000 MHz
\citep[p. 47]{Doviak:2006}.
However, S-band is highly desired for other applications,
like cellular phones. So, C-band or X-band radars are also used for observing weather. In such cases, the effects of attenuation can be
somewhat mitigated by, e.g., using polarimetric radar measurements \citep{Ryzhkov:2008}. 

One advantage of higher frequency
radars is the possibility of a smaller antenna size for the same angular resolution. Other components such as waveguides,
amplifiers and support structure are also smaller, so the whole system requires less space and is often less expensive to 
manufacture and easier to deploy.

As technology advances, demands on radar performance increase. Finer resolution offers better detail, while
dual polarization radars provide even more data that can be used to better analyze the atmosphere \citep[p. 242]{Doviak:2006}.
This data is derived from the so-called \emph{backscattering covariance matrix}, which relates the backscattered electric
field to the incident electric field. The additional information, e.g. $Z_{dr}$, the differential reflectivity, or $\rho_{hv}$, the
correlation coefficient, are determined from that matrix and used for, e.g., hydrometeor classification or better rainfall 
estimation \citep{Ryzhkov:2008}.

\section{Phased Array Radars}
Phased array radars move away from standard dish antennas to so-called antenna-arrays.
An antenna array is made up of discrete units of transceivers distributed in one or more dimensions. The NWRT Phased Array 
Radar, for example, has 4352 elements distributed across a plane \citep{Zrnic:2007}. Each of the elements is able to transmit a
signal with a certain phase shift. Setting these phase shifts in a certain way will result in the signals of all the
array elements to add up constructively, steering the radar beam in a certain direction without physically moving the antenna.

This electronic steering has the advantages of being very agile and flexible and can reduce overall scan time,
focus on particular targets in space in any order, or reduce data correlation by steering the beam out-of-order across
a storm, a technique called beam multiplexing (BMX) \citep{Yu:2007}. BMX provides fast updates of weather information 
with higher statistical accuracy because collected samples are independent since they have not been collected sequentially, 
allowing the atmosphere to decorrelate before taking the next sample in the same area.

For solid-state amplifiers the peak power is generally lower compared to a
klystron or magnetron, because they are made with transistors. As such, they operate at lower voltages and have a lower gain, but
have high reliability. To achieve the needed power, many of them can be operated in parallel and in multiple stages as modules
\citep{Skolnik:2001}. So, instead
of using a klystron or magentron, a phased array radar can incorporate amplifier modules into the transceiver and have each antenna
element powered separately. The advantage of this kind of modular design is the high reliability and fault tolerance of the system. 
Failure of individual elements still allows the radar to operate, and failed elements can be repaired by simply replacing them 
with a new one. This lowers maintenance cost and, because many identical elements are needed, manufacturing cost due to mass
production.


Solid state amplifiers also generally require longer pulse lengths because their peak power capability is lower, requiring
operation at lower duty cycles \citep[p. 702]{Skolnik:2001}. However, a longer pulse decreases range resolution, so
a method is needed to improve the signal-to-noise ratio (SNR) while keeping the range resolution acceptable. 
That method is called pulse compression.


\section{The Need for Pulse Compression}
Pulse compression is a technique to increase the SNR by using waveforms that allow the duration
and bandwidth to be controlled separately.

Compression is achieved by using one or more matched filters, which will be introduced in Chapter \ref{genwaveform}.
This allows the design of waveforms such that they are focused in a 
narrow main lobe, with low sidelobes to concentrate the signal energy in the main lobe \citep{Richards:2005}.
The challenge then becomes to design a waveform that has the needed SNR and at the same time good range resolution, 
while keeping down sidelobes, and is a procedure of many trade-offs.
