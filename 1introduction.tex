There are numerous works available on waveform design, the most comprehensive of which being \cite{Levanon:2004}.  This
work covers the
most common waveforms used today, such as pulses, pulse trains, frequency and phase modulated signals, as well as
continuous wave signals, each accompanied by comprehensive discussion and examples.  \cite{Richards:2005} also discusses
radar waveforms with an emphasis on frequency modulation and pulse trains, but focuses more generally on radar signal
processing.  The work of \cite{Keeler:1999} has much information for weather radar applications, concentrating on filter
design in phased array radars.  \cite{Mudukutore:1998} discusses the simulation of pulse compression for weather radars
for phase coded signals and different filters, with an emphasis on sidelobe reduction.

\section{Using Radar for Observing the Atmosphere}
From hard target observation during WWII and the latter part of the 20th century, radar quickly became a potential way
to observe the atmosphere. As with all remote sensing applications, radar can give advanced warning and better insights
into what is happening in the atmosphere in places that are not accessible or difficult to observe using other means,
such as localized weather stations or weather balloons. This made dense networks of instruments unnecessary, drastically
lowering maintenance and operation expenses and increasing the area of observation.

The most basic pulsed radar is able to determine the reflectivity of the atmosphere by measuring the return signal of
the radar beam. Additionally, coherent radars can measure the Doppler shift of the return signal and use it to find the
radial velocity of the target, i.e., the velocity component of the target along the radar beam. This is normally
achieved with a stable local oscillator (STALO). Simplified, the STALO signal is split into an in-phase and a 90 degree
phase shifted signal, which are then mixed with the return signal and passed through a set of low-pass filters that
remove the undesired frequency components. The two outputs of the filters are called $I$, the in-phase component, and
$Q$, the quadrature component, respectively. It is necessary to measure the phase of the $I$ and $Q$ components over
several pulses to determine the Doppler frequency, due to the long decorrelation time of the signal.


The frequency of the radar, and thus its wavelength, contribute to the performance for weather observation, because some
wavelengths perform better than others, especially with respect to attenuation through hydrometeors \citep[p.
448]{Skolnik:2001}. Specifically, attenuation due to rain depends on the wavelength, as does attenuation in clouds, with
signals of shorter wavelengths being more strongly attenuated \citep[pp. 42-43]{Doviak:2006}.   The WSR-88D network
radars operate at S-band, more specifically between 2700 and 3000 MHz \citep[p. 47]{Doviak:2006}.  However, S-band is
highly desired for other applications, like cellular phones, so C-band or X-band radars are also used for observing
weather to avoid interference between different systems operating at the same frequency band. For radars of shorter
wavelengths, the effects of attenuation can be mitigated by, e.g., using techniques that are based on polarimetric
measurements (\cite{Ryzhkov:2008}, \cite{Brandes:2002}).  

One advantage of higher frequency radars is the possibility of a smaller antenna size
for the same angular resolution. Other components such as waveguides, amplifiers and support structure are also
smaller, so the whole system requires less space and is often less expensive to manufacture and easier to deploy.
Another beneficial cost factor is the increased availability of commercial off-the-shelf (COTS) components due to an
increase in consumer RF devices like wireless routers and other networking equipment. These components are produced
in high quantities and available at lower costs to the radar manufacturer. One example of taking advantage of COTS
components for Digital Array Radars is given in \cite{Tarran:2008}.

As technology advances, demands on radar performance increase as well. Finer resolution offers better detail, while dual
polarization radars provide even more data that can be used to better analyze the atmosphere \citep[p.
242]{Doviak:2006}.  These data are derived from the so-called \emph{backscattering covariance matrix}, which relates the
backscattered electric field to the incident electric field. With two orthogonal polarizations, additional parametric
variables are derived, e.g. $Z_{\mathrm{dr}}$, the differential reflectivity, $\rho_{\mathrm{hv}}$, the correlation
coefficient, or $\Phi_{\mathrm{dp}}$, the differential phase. They are mostly used for hydrometeor classification \citep{Ryzhkov:2008} and
Quantitative Precipitation Estimation (QPE) \citep{Brandes:2002}.

\section{Phased Array Radars}
\label{s:par}
Phased array radars move away from parabolic dish antennas to so-called antenna-arrays.  An antenna array is made up of
discrete antennas distributed in space. Each antenna element may function as a receiver and/or transmitter.  The NWRT
Phased Array Radar, for example, has 4352 elements distributed across a plane \citep{Zrnic:2007}. Each of the elements
is able to transmit a signal with a different phase shift. The distribution of the phase shifts will result in the
signals of all the array elements to add constructively, steering the radar beam electronically in a certain direction.

This electronic steering has the advantages of being very agile and flexible and can reduce overall scan time, focus on
particular targets in space in any order, or reduce data correlation by steering the beam out-of-order across a storm, a
technique called beam multiplexing (BMX) \citep{Yu:2007}. BMX provides fast updates of weather information with higher
statistical accuracy because collected sample pairs are independent since they have not been collected sequentially, allowing
the atmosphere to decorrelate before taking the next pair of samples in the same area.

Some phased array radars, so-called \emph{active arrays}, do not have just one transmitter. Instead, they use one smaller
amplifier for each element. These amplifiers are solid-state amplifiers, whose peak power is generally lower
compared to a klystron or magnetron. As such, they operate at lower voltages,
have a lower gain, and require longer duty cycles, but have higher reliability. To achieve the needed power, many of
them must be operated in parallel and in multiple stages as modules \citep{Skolnik:2001}. So, instead of using a klystron
or magnetron, an active phased array radar incorporates individual amplifier modules into each antenna element.
One major advantage of this kind of modular design is the high reliability and fault tolerance of the
system.  Failure of a small percentage of individual elements still allows the radar to operate, and failed elements can simply be replaced
at a convenient time.  This lowers maintenance and manufacturing costs because the components of the identical elements can be mass
produced and could be available COTS.

One way to compensate for the lower peak power of solid-state amplifiers is to use longer pulses \citep[p.
702]{Skolnik:2001}, given their longer duty cycle. However, a longer pulse decreases range resolution, so a method is
needed to improve the range resolution while keeping the sensitivity acceptable. One such method is called pulse
compression.


\section{The Need for Pulse Compression}
Pulse compression is a technique to increase sensitivity (increase the signal-to-noise ratio (SNR)) by using waveforms
that allow the duration and bandwidth to be controlled separately \citep[pp. 42-43]{Richards:2005}.  An early paper by
Bell Labs introduced the concept of a chirp, or frequency modulated pulse, for exactly this purpose
\citep{Klauder:1960}. Since then, other ways of achieving the same goal have been found and will be discussed in this
work. Compression is achieved by using one or more matched filters, the concept of which will be introduced in Chapter
\ref{genwaveform}.  This allows the design of waveforms such that their energy is focused in a mainlobe with low
sidelobes, concentrating the signal energy as narrowly as possible \citep[p. 44]{Richards:2005}. This is similar to
the main- and sidelobes of an antenna pattern, but in range-time instead.

As mentioned previously, solid-state amplifiers are economical, but have lower peak power and higher duty cycles, so
long pulses are necessary to achieve the same sensitivity. On one hand, range resolution is important to detect small
features and would suffer severely from longer pulses.  On the other hand, the SNR is directly related to the
probability of detection \citep[p. 44]{Skolnik:2001}, so it is very important to maintain a sufficient SNR.
Furthermore, since weather signals have a large dynamic range, required for detecting anything from light rain or even
just clear air measurements, to heavy rain and hail, a high SNR is absolutely necessary.  These requirements for good
range resolution and SNR make pulse compression a welcome technique to improve weather radars. The different ways of
accomplishing pulse compression are discussed in this thesis.

Chapter 2 discusses the general aspects of waveforms and their design and introduces a way to measure their performance.
The topic of Chapter 3 is pulse compression and the various ways of applying modulation to the waveform. Chapter 4
explains the performance metrics used in this work to quantify waveform performance, with Chapter 5 presenting the
results. Chapter 6 provides the conclusion and future work.


