\section{Literary Survey}
There are numerous works available on waveform design, the most comprehensive being \cite{Levanon:2004} without
a doubt. \cite{Richards:2005} also hits on general waveform techniques for radar.
The paper by \cite{Keeler:1999} has much information for weather radar applications, concentrating on phased
array radars.

\section{Using Radar for Observing the Atmosphere}
From only observing hard targets during WWII and the latter part of the 20th century, radar quickly became 
a great way to observe the atmosphere. As with all remote sensing applications, radar could give advanced warning
and better insights into what's happening in the atmosphere in places that are not possible or difficult to observe
using other means.

A basic pulsed radar allows for the retrieval of weather intensity as well as Doppler shift to determine the radial
velocity of the hydrometeors.

Some frequency bands perform better for observing weather than others. The higher the frequency, the more attenuated
the signal becomes as it passes through the atmosphere. The ideal band for weather radars is about 3 GHz, or S-band,
because of its low attenuation. However, S-band may already be reserved and is generally highly desired for other applications,
like cell phones, so C-band or X-band radars are also used for observing weather, where the effects of attenuation can be
mitigated somewhat. One advantage of higher frequency radars is the need for a smaller antenna, which reduces required
space and cost.

\section{Phased Array Radars}
As more and more radar components are pushed into the digital domain, radars have become smaller in size and cheaper to
manufacture. In particular, phased array radars move away from a standard dish antenna to so-called array antennas.
An array antenna is made up of discrete units of transceivers distributed in one or more dimensions. OU's Phased Array 
Radar, for example, has over 4000 elements distributed across a plane. Each of the elements is able to transmit a
signal with a certain phase shift. Setting these phase shifts in a certain way will result in the signals of all the
array elements to add up in different ways, steering the radar beam without physically moving the antenna.

This electronic steering has the advantage of being very fast and flexible and works well to reduce overall scan time,
focus on particular targets in space in any order, or reduce data correlation by steering the beam out-of-order.

Lower power amplifiers in these systems make it necessary to increase pulse length in order to get the same amount of 
energy on target. Longer pulses, in turn, decrease range resolution.

\section{The Need for Pulse Compression}
The goal for radar signals is high sensitivity, which means a high signal-to-noise ratio (SNR), 
and high resolution. In order to get a high SNR, the more energy can be put on target, the better
the sensitivity. For a radar operating in transmitter saturation, that means the longer the
transmitted pulse is, the more energy is transmitted.  Unfortunately, pulse length is inverse 
proportional to the resolution, so the longer the pulse, the larger the resolution volume will be.
Pulse compression is a way to decouple the relationship between SNR and resolution by controlling 
the duration and bandwidth separately.

The use of matched filter allows pulse compression since its output is the autocorrelation function 
of the transmit signal. This allows the design of waveforms such that they are narrowly focused in a 
narrow main lobe, with low sidelobes to concentrate the signal energy in the main lobe (\cite{Richards:2005}).
