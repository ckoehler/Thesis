\section{af.m}
\lstset{
  basicstyle=\scriptsize
}
\begin{lstlisting}[language=Matlab]

function [delay v af] = af(signal, clean_signal, tau, fs, v_max, f_points,
    carrier, full_af) 
  % Ambiguity function calculation
  % ambiguity function is af(t,f) = sum_over_t(u(t) * u'(t-tau) * exp(j*2*pi*f*t))
  %
  % fs = Range dimension sampling frequency
  % signal = the signal we need the AF of.
  % clean_signal = if different from signal, this is the signal we use to create 
  %                the time-shifted version, or u'(t-tau) above.
  % tau = signal length
  
  % if no signal is given, assume that all we have is a clean signal
  % and use that.
  if isempty(signal)
    signal = clean_signal;
    ir=false;
  else
    ir=true;
  end

  if nargin < 8
    full_af=false;
  end


  % speed of light
  c = 3e8;
  % wavelength
  lam = c/carrier;

  m = length(signal);
  m_clean = length(clean_signal);

  af = [];
  
  % convert v_max to a frequency
  f_max = 2*v_max/lam;

  % frequency span
  if full_af
    f = linspace(-f_max,f_max, 2*f_points);
  else
    f = linspace(0,f_max, f_points);
  end

  %f = [1.5917e3 1.5917e3];
  for i=1:length(f)
    dshift = exp(1i*2*pi.*f(i).*(0:m_clean-1)/fs);
    shifted_s = clean_signal.*dshift;

    af(i,:) = abs(xcorr(signal, shifted_s));
    %af(i,:) = abs(conv(signal,fliplr(conj(shifted_s))));
    
  end
  af = af./max(max(af));

  % compute delay
  t = linspace(0, tau, m);
  delay = [-fliplr(t) t(2:end)] * c / 2;

  % convert doppler frequency to velocity
  v = f .* lam ./ 2;
end

\end{lstlisting}



\section{makesignal.m}
\begin{lstlisting}[language=Matlab]
function [signal ir_signal new_tau] = makesignal(amp, phase, freq_mod, imp_resp,
      tau, fs)

  ir_signal = [];

  debug = false;
  if ~debug
    echo makesignal off;
  end

  if ~isempty(imp_resp)
    ir = true;
  else
    ir = false;
  end

  if ~isempty(freq_mod)
    fm = true;
  else
    fm = false;
  end


  % this is how many samples we need to use.
  N = tau*fs;

  % this is the "time" sequence, just a sequence of samples.
  n = 0:N-1;

  if isempty(amp)
    m = length(phase);
    amp = ones(1,m);
  end

  if isempty(phase)
    m = length(amp);
    phase = zeros(1,m);
  end

  signal = amp .* exp(j.*phase);
  
  if debug
    'amp'
    size(amp)
    'phase'
    size(phase)
    m
    N
    fm
    ir
  end

  % expand the signal and impulse response for oversampling.
  tempsignal = kron(signal,ones(1,floor(N/m)));

  if debug
    'tempsignal'
    size(tempsignal)
  end

  diff_length = uint32(N-length(tempsignal));
  signal = [tempsignal zeros(1,diff_length)];
  clear tempsignal;

  if fm
    %freq_mod = (2.*pi.*n./fs.*freq_mod);
    freq_mod = 2.*pi.*cumsum(freq_mod./fs);
  else
    freq_mod = 0;
  end

  if debug
    'freq mod'
    size(freq_mod)
  end

  % reassemble the signal
  signal = signal .* exp(j.*freq_mod);
  signal = signal(1:N-diff_length);
  new_tau = tau;

  if ir
    imp_resp = imp_resp ./ max(imp_resp);
    imp_resp = kron(imp_resp, ones(1,floor(N/m)));
    ir_signal = conv(signal, imp_resp);
    new_tau = tau / length(signal) * length(ir_signal);
  end

end
\end{lstlisting}
