All of the code will be available on GitHub for the foreseeable future: \\
\texttt{http://github.com/ckoehler/Thesis-code}  \\
The two main functions are given here.

\section{af.m}

\lstset{
  basicstyle=\scriptsize
}
\begin{lstlisting}[language=Matlab]

function [delay v af] = af(impulse_response, signal, tau, fs, v_max, f_points,
    carrier, full_af) 
  % Ambiguity function calculation
  % ambiguity function is af(t,f) = sum_over_t(u(t) * u'(t-tau) * exp(j*2*pi*f*t))
  %
  % fs = Range dimension sampling frequency
  % signal = the signal we need the AF of.
  % clean_signal = if different from signal, this is the signal we use to create 
  %                the time-shifted version, or u'(t-tau) above.
  % tau = signal length

  m = length(signal);
  
  % if no signal is given, assume that all we have is a clean signal
  % and use that.
  if isempty(impulse_response)
    ir=false;
    m_ir = m;
  else
    ir=true;
    m_ir = m;
    tau = 2*tau;
  end

  if nargin < 8
    full_af=false;
  end

  % speed of light
  c = 3e8;
  % wavelength
  lam = c/carrier;

  af = [];
  
  % convert v_max to a frequency
  f_max = 2*v_max/lam;

  % frequency span
  if full_af
    f = linspace(-f_max,f_max, 2*f_points+1);
  else
    f = linspace(0,f_max, f_points);
  end

  for i=1:length(f)
    dshift = exp(1i*2*pi.*f(i).*(0:m-1)/fs);
    shifted_s = signal.*dshift;

    if ir
      shifted_s = conv(shifted_s, impulse_response); 
      m_ir = length(shifted_s);
    end

    af(i,:) = abs(xcorr(signal, shifted_s));
    
  end
  af = af./max(max(af));

  % compute delay
  t = linspace(0, tau, m_ir);
  delay = [-fliplr(t) t(2:end)] * c / 2;

  % convert doppler frequency to velocity
  v = f .* lam ./ 2;
end


\end{lstlisting}



\section{makesignal.m}
\begin{lstlisting}[language=Matlab]
function signal = makesignal(amp, phase, freq_mod, tau, fs)

  if ~isempty(freq_mod)
    fm = true;
  else
    fm = false;
  end


  if isempty(amp)
    m = length(phase);
    amp = ones(1,m);
  end

  if isempty(phase)
    m = length(amp);
    phase = zeros(1,m);
  end

  signal = amp .* exp(j.*phase);
  
  if fm
    freq_mod = 2.*pi.*cumsum(freq_mod./fs);
  else
    freq_mod = 0;
  end

  % reassemble the signal
  signal = signal .* exp(j.*freq_mod);

end

\end{lstlisting}
