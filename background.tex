This is the background

\section{Orthogonal Coding}
Orthogonal coding is used in polarimetric radars to reduce the linear depolarization ratio between H and V-channels.
According to \cite{Chandrasekar:2008}, orthogonal coding foo.
Each channel is encoded with one of a pair of codes that are said to be orthogonal. That means that when the pair is 
multiplied together, the product is zero, effectively eliminating the cross products of (some calculation).

\section{Pulse Compression}
\label{pc}
The goal for radar signals is high sensitivity, which means a high signal-to-noise ratio (SNR), and high resolution. In order to get a high SNR,
the more energy can be put on target, the better the sensitivity. For a radar operating in transmitter saturation, that means the longer the
transmitted pulse is, the more energy is transmitted. 
Unfortunately, pulse length is inverse proportional to the resolution, so the longer the pulse, the larger the resolution volume will be.
Pulse compression is a way to decouple the relationship between SNR and resolution by controlling the duration and bandwidth separately.

The use of matched filter allows pulse compression since its output is the autocorrelation function of the transmit signal. This allows the design of
waveforms such that they are narrowly focused in a narrow main lobe, with low sidelobes to concentrate the signal energy in the main lobe (\cite{Richards:2005}).

\section{Pulse Compression with Frequency Modulation}
\label{pcwithfreq}
In Pulse Compression, the frequency inside the pulse is modulated linearly (for Linear FM, or LFM) or non-linearly over a certain bandwidth. It is defined as:
\begin{equation}
\label{LFM}
x(t) = e^{j \pi \beta t^2 / \tau} = e^{j \theta (t)},\ \ 0 \leq t \leq \tau
\end{equation}

The spectrum of such a LFM waveform approaches the shape of a rectangle that is becomes better defined as the $\beta\tau$ ratio increases. The matched 
filter response of a rectangular waveform is a sinc function with Rayleigh resolution of approximately $1/\beta$ (\cite{Richards:2005}).


The result is an increase in the ambiguous range $r_a$.

\section{Pulse Compression with binary codes}
One alternative to \ref{pcwithfreq} is the use of binary, or in our case, quadrature codes. With it, a subsection of each pulse is
shifted in phase by $e^{jk \pi}$, where $k=(0, \frac{1}{2}, 1, \frac{3}{2})$. The performance of each code is measured by calculating
the Integrated Sidelobe Level (ISL), which is defined as (\cite{Keeler:1999}):

\begin{equation}
\label{ISL}
ISL = 10 \log \left(\frac{\sum\limits_{j} |m_k|^2}{\sum\limits_{k} |s_j|^2}\right)
\end{equation}

where $s_j$ is the power of the $j^{th}$ sidelobe, and $m_k$ the power of the $k^{th}$ mainlobe.

The higher the ISL, the better the performance of the code.

\section{Pule Compression and Velocity Fields}
For larger pulse width, the returned Doppler phase shift influences the carefully calculated phase shifts of the pulse and can lead to 
greatly increased range sidelobes. 

One solution could be to determine the Doppler shift with a short pulse and apply some sort of phase correction to the following long pulse
in order to negate the effects of the non-zero phase shifts returned by moving targets.


