This is the background

\section{Orthogonal Coding}
Orthogonal coding is used in polarimetric radars to reduce the linear depolarization ratio between H and V-channels.
According to \cite{Chandrasekar:2008}, orthogonal coding foo.
Each channel is encoded with one of a pair of codes that are said to be orthogonal. That means that when the pair is 
multiplied together, the product is zero, effectively eliminating the cross products of (some calculation).


\section{Pulse Compression with Frequency Modulation}
\label{pcwithfreq}
In Pulse Compression, the frequency inside the pulse is modulated linearly (for Linear FM) or non-linearly over a certain bandwidth.
The result is an increase in the ambiguous range $r_a$.

\section{Pulse Compression with binary codes}
One alternative to \ref{pcwithfreq} is the use of binary, or in our case, quadrature codes. With it, a subsection of each pulse is
shifted in phase by $e^{jk \pi}$, where $k=(0, \frac{1}{2}, 1, \frac{3}{2})$. The performance of each code is measured by calculating
the Integrated Sidelobe Level (ISL), which is defined as (\cite{Keeler:1999}):

\begin{equation}
\label{ISL}
ISL = 10 \log \left(\frac{\sum\limits_{j} |s_j|^2}{\sum\limits_{k} |m_k|^2}\right)
\end{equation}

where $s_j$ is the power of the $j^{th}$ sidelobe, and $m_k$ the power of the $k^{th}$ mainlobe.

The lower the ISL, the better the performance of the code.


