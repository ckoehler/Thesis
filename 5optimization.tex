In this section the setup and results of the simulations will be presented. First, the way the simulations were
conducted is explained. Afterwards , the results for various waveforms will be shown.

\section{Setup and Configuration}
All simulations were performed with the MATLAB\textsuperscript{\textregistered} software package.  The general algorithm
to compute the AF is presented, with the complete code given in Appendix \ref{app:code}.


\subsection{Generating Waveforms}
The simulations are done in multiple parts. First, a waveform is generated based on a given amplitude, phase, frequency
modulation, and impulse response that is applied to the signal.  The returned waveform is generated at a proper sampling
frequency in order to avoid aliasing.

If an impulse response was supplied, both the distorted and clean waveform will be returned, as well as the new pulse
length $\tau$ which will be twice the original $\tau$. That lengthening is due to the convolution of the
original waveform $s(t)$ with the impulse response $h(t)$.

\subsection{Calculating the Ambiguity Function}
Once the waveform has been returned, the AF is generated and plotted in a surface plot.  In pseudo-code, given the
complex waveform \verb|signal| and an array of Doppler frequencies \verb|doppler_frequencies|, the normalized AF is
computed as follows.

\lstset{
  basicstyle=\footnotesize
}
\begin{lstlisting}[language=Ruby]
af = []
for dfreq in doppler_freqencies do
  doppler_shift = exp(-j*2*pi*dfreq)
  shifted_signal = signal*doppler_shift
  af << abs(xcorr(signal, shifted_signal))
end
return (af / max(af))

\end{lstlisting}





\section{Rectangular Pulse}
A rectangular pulse is the simplest waveform for a radar. The following figures show the AF for pulses of varying
lengths so that the effect of pulse length on Doppler resistance can be examined. Figure \ref{fig:simplepulse-15us} shows
two slices of the AF of a \SI{15}{\micro\second} pulse.  There is no significant change in the AF across the Doppler axis, so the different
Doppler cuts are identical, which means that for weather radar applications, the commonly encountered Doppler shifts do
not cause significant distortion to the signal for short pulses.

Similarly, Figure \ref{fig:simplepulse-200us} shows the AF of a much longer \SI{200}{\micro\second} pulse. Comparing it
to the shorter pulse, it looks very similar. There is some distortion at higher Doppler velocities for the longer pulse,
about \SI{-7}{\decibel}, which is significant but only occurs at high velocities and long pulses.
Additionally, pulse lengths of \SI{200}{\micro\second} are not common. For comparison, the two pulse lengths used in the WSR-88D radar
network are \SI{1.57}{\micro\second} and \SI{4.57}{\micro\second} \citep[p. 47]{Doviak:2006}.




\begin{figure}[H]
\includegraphics[width=\imgsizes]{figures/pulse-15us.png}
\caption{ AF slices of a rectangular pulse of length 15 $\mu s$. There is no significant Doppler mismatch for typical velocities of
weather phenomena.}
\label{fig:simplepulse-15us}
\includegraphics[width=\imgsizes]{figures/pulse-200us.png}
\caption{ AF slices of a rectangular pulse of length 200 $\mu s$. The effect on Doppler
performance for typical velocities of weather phenomena is about \SI{7}{\decibel} loss, which is significant. }
\label{fig:simplepulse-200us}
\end{figure}






\section{Barker Code}
In this section, a Barker 13 code of different pulse lengths will be examined. Figure \ref{fig:barker13-15us}
shows two slices of the AF of a \SI{15}{\micro\second} pulse that has been phase coded with a length 13 Barker
code. As for a rectangular pulse, there is no Doppler distortion for a short pulse, so this effect can be neglected.
The zero-Doppler cut shows the typical shape of the Barker code and sidelobes of about \SI{-22}{\decibel} \citep[p.
351]{Skolnik:2001}. The range resolution is approximately \SI{110 }{\metre} at the \SI{3}{\decibel} point.

Comparing Figure \ref{fig:barker13-15us} to Figure \ref{fig:barker13-200us} shows that there is no significant
difference in the Doppler dimension either, so the pulse length does not affect the shape of the AF for a Barker 13 code. 
The only difference is the range resolution, which is larger due to the increased pulse length, as expected.

\begin{figure}[H]
\includegraphics[width=\imgsize]{figures/barker-15us.png}
\caption{ AF slices of the Barker 13 code of length 15 $\mu s$. The sidelobes are all at the same level, which is too high to
achieve sufficient sensitivity for typical weather applications. }
\label{fig:barker13-15us}
\end{figure}

The high sidelobes of the Barker code make it not suitable for typical weather applications, although it could still
be used in applications that don't require the high dynamic range of weather observations, such as clear air or
convective boundary layer (CBL) observations.


\begin{figure}[H]
\includegraphics[width=\imgsize]{figures/barker-200us.png}
\caption{ AF slices of the Barker 13 code of length 200 $\mu s$. There is no significant distortion for high Doppler
velocities, similar to the shorter Barker 13 waveform. }
\label{fig:barker13-200us}
\end{figure}



\section{Linear Frequency Modulation}
In this section, the results of frequency modulation with and without amplitude tapering will be presented. The
bandwidth is limited to \SI{5}{\mega\hertz}.


\subsection{Linear Frequency Modulation}
The frequency sweep function of an LFM waveform is shown in Figure \ref{fig:lfm-fmf}. It is a linear sweep from \SI{-2.5}{\mega\hertz}
to \SI{2.5}{\mega\hertz}, over a total of \SI{5}{\mega\hertz}.

\begin{figure}[H]
\includegraphics[width=\imgsize]{figures/lfm-fmf.png}
\caption{Frequency of an LFM waveform, sweeping over \SI{5}{\mega\hertz} for the duration of the pulse
$\tau=\SI{15}{\micro\second}$.}
\label{fig:lfm-fmf}
\end{figure}


Figures \ref{fig:lfm-15us} and \ref{fig:lfm-200us} each show two AF slices of two LFM waveforms with pulse lengths of
\SI{15}{\micro\second} and \SI{200}{\micro\second}, respectively.  Comparing the two AFs shows that in the
Doppler dimension, the pulse length has little effect for weather radar applications. There is some shifting of the
mainlobe, but it is negligible.


\begin{figure}[H]
\includegraphics[width=\imgsizes]{figures/lfm-15us.png}
\caption{ AF slices of an LFM waveform of length 15 $\mu s$. Again, Doppler mismatch is negligible for typical weather-related
velocities.  }
\label{fig:lfm-15us}
\includegraphics[width=\imgsizes]{figures/lfm-200us.png}
\caption{ AF slices of an LFM waveform of length 200 $\mu s$. Increasing the pulse length only affects the Doppler
mismatch, but has no effect on range resolution. }
\label{fig:lfm-200us}
\end{figure}

One observation of significance is the range resolution of both signals. As expected, the sidelobes are approximately
\SI{-13}{\decibel} in both AFs, but also, the shapes are the same for each case, which means that the
pulse length has no effect on the range resolution, which is approximately \SI{30 }{\metre}. Instead, range
resolution is determined by the sweep bandwidth, not the pulse length, as is expressed in Equation
\eqref{eq:lfm-rangeres}.


\subsection{Linear Frequency Modulation with Amplitude Tapering}
LFM has the disadvantage of very high sidelobes at about \SI{-13}{\decibel}.  These can be mitigate with amplitude
tapering. In this work, a Kaiser window is applied to demonstrate the effects of amplitude tapering.  Based on Equation
\eqref{eq:kaiser2}, the parameter $\beta$ is varied from 0 to 20 in approximately 0.2 increments, and the AF generated
and evaluated for each by quantifying its performance according to the metrics described in Section
\ref{s:quantifications}. The results are shown in Figure \ref{fig:lfm-isl-15us-opti}, which shows the same optimizations
for both the whole AF and just the zero-Doppler slice. There is no significant difference, so using the whole AF for
calculations does not adversely affect the calculations. Thus, to be comprehensive, the rest of the optimizations in
this work are made across the whole AF.

As expected, the ISL dramatically decreases, as does the PSL. At the same time, the range resolution decreases from an
initial \SI{40}{\metre} to about \SI{200}{\metre}. The PPR may be the limiting metric here because the tapered signal
quickly loses half of its power with only minor amplitude tapering and eventually decreases to about \SI{-7}{\decibel}
of the original amplitude, which means the signal loses a significant amount of its power. This, in turn, means reduced
sensitivity.


\begin{figure}[H]
\includegraphics[width=\imgsize]{figures/lfm-isl-15us-opti.png}
\caption{Optimization results for LFM with Kaiser amplitude tapering and $\tau=\SI{15}{\micro\second}$. Tapering the amplitude is very effective for
sidelobe reduction, but loss in power and thus sensitivity becomes significant quickly. Both results for the whole AF as
well as just the zero-Doppler slice are shown. }
\label{fig:lfm-isl-15us-opti}
\end{figure}


Clearly, a trade-off must be made. If the half power point is selected, the waveform would have an ISL of about
\SI{-36}{\decibel}, PSL of \SI{-47}{\decibel}, and a range resolution of \SI{68}{\metre}.
The corresponding zero Doppler cut of the AF at this point is shown in Figure \ref{fig:lfm-isl-15us-opti-slice18}. It
clearly shows the sidelobes at about \SI{-50}{\decibel}, which means this waveform would be adequate for typical weather
observations if a \SI{3}{\decibel} loss in power is acceptable. 


\begin{figure}[H]
\includegraphics[width=\imgsize]{figures/lfm-isl-15us-opti-slice18.png}
\caption{Zero Doppler cut of the tapered LFM AF at the half power point. Sidelobes are reduced significantly while range
resolution is reduced to \SI{68}{\metre}, which is well sufficient for typical storm observations.}
\label{fig:lfm-isl-15us-opti-slice18}
\end{figure}


\section{Non-Linear Frequency Modulation}
As described in Section \ref{s:nlfm}, the frequency of a frequency modulated signal can be swept non-linearly as well.
In order to find more useful waveforms for weather radar applications, this work evaluates several different sweep
functions described by Equation \eqref{eq:nlfm-sweep} below. The PPR here is always 0 dB because no amplitude tapering is
applied yet, so there is no loss of sensitivity. In this study, the general form of the function describing the frequency sweep is:

\begin{equation}
\label{eq:nlfm-sweep}
  f(t) = a t^7 + t,
\end{equation}
where $a$ varies from 0 to 10. Whole functions for different values of $a$ are shown in Figure \ref{fig:nlfm-functions}.
Each function is applied over the pulse length $\tau$ of the signal.

\begin{figure}[H]
\includegraphics[width=\imgsize]{figures/nlfm-functions.png}
\caption{Different functions describing the non-linear frequency sweeps. AFs for each are generated and evaluated for
performance.}
\label{fig:nlfm-functions}
\end{figure}

For each function, a waveform and AF are generated and then evaluated according to the optimizations
described in Section \ref{s:quantifications}. It is then determined which sweep function is best suited for weather
radar applications, and further changes to the setup are made to combine it with amplitude tapering to try and achieve
even better performance.


\subsection{Parameterized NLFM Results}
\label{s:nlfm-only-results}
Figure \ref{fig:nlfm-side-15us-opti} shows the results for varying NLFM waveforms in regards to ISL, PSL, and range
resolution. For larger $a$, the ISL also increases, with the maximum ISL being around \SI{-30}{\decibel}. The PSL
decreases rapidly first to about \SI{-26}{\decibel} and then slowly decreases to about \SI{-40}{\decibel}, where it
stays fairly constant. The range resolution gets progressively worse, starting at about \SI{30}{\metre} and increasing
to about \SI{300}{\metre}, meaning the mainlobe broadens.  This also explains the decreasing and eventually constant
PSL, because the AF is essentially one large mainlobe with only minimal sidelobes, an example of which is shown in
Figure \ref{fig:nlfm-side-15us-10}.

\begin{figure}[H]
\includegraphics[width=\imgsize]{figures/nlfm-side-15us-10.png}
\caption{Zero-Doppler slice of the NLFM AF with parameter $a=10$. The mainlobe is so wide that the sidelobes are minimal
compared to it.}
\label{fig:nlfm-side-15us-10}
\end{figure}

\begin{figure}[H]
\includegraphics[width=\imgsize]{figures/nlfm-side-15us-opti.png}
\caption{Optimization results for NLFM waveforms. Sidelobe reduction is between \SI{-20}{\decibel} and
\SI{-40}{\decibel} with only a modest decrease in range resolution.}
\label{fig:nlfm-side-15us-opti}
\end{figure}

To get a better picture of the NLFM waveforms, consider Figure \ref{fig:nlfm-side-15us-0D-1}, showing the zero-Doppler
slice of the NLFM waveform generated with the sweep function with $a=2.9293$. The mainlobe is approximately
\SI{50}{\metre} at the \SI{3}{\decibel} point. The sidelobes are better than those of the LFM waveform at slightly
higher than \SI{-30}{\decibel}.  In comparison, Figure \ref{fig:nlfm-side-15us-0D-2} shows the NLFM waveform generated
with the sweep function with $a=6.46$.  The sidelobes are a little lower at \SI{-40}{\decibel} and the range
resolution at the \SI{3}{\decibel} point is still only about \SI{70}{\metre}, making it a good waveform for weather
radar applications.

\begin{figure}[H]
\includegraphics[width=\imgsizes]{figures/nlfm-side-15us-0D-1.png}
\caption{Zero-Doppler slice of the NLFM AF with $a=2.92$. Sidelobes are only slightly lower than plain LFM with
good range resolution.}
\label{fig:nlfm-side-15us-0D-1}
\end{figure}
\begin{figure}[H]
\includegraphics[width=\imgsizes]{figures/nlfm-side-15us-0D-2.png}
\caption{Zero-Doppler slice of the NLFM AF with $a=6.46$. Sidelobes are reduced by an additional \SI{10}{\decibel} with
a slight decrease in range resolution.}
\label{fig:nlfm-side-15us-0D-2}
\end{figure}


\subsection{Parameterized NLFM with Amplitude Tapering}
In order to improve the performance of the varying NLFM waveforms, Kaiser windows are applied to taper the amplitude.
The X and Y axes of the following figures show the parameters of the Kaiser window, $\beta$, and NLFM sweep function,
$a$, respectively. For each combination of $\beta$ and $a$, a waveform and AF are generated and then evaluated.  The
results are presented in Figures \ref{fig:nlfm-kaiser-15us-opti-isl}-\ref{fig:nlfm-kaiser-15us-opti-res}.

Looking at the ISL in Figure \ref{fig:nlfm-kaiser-15us-opti-isl}, it can be seen that the waveform that is most
aggressively tapered generally has better ISL performance. The same is true for the NLFM parameter $a$, i.e., for higher
$a$, ISL is lower.  Next, the PSL shown in Figure \ref{fig:nlfm-kaiser-15us-opti-psl} closely mirrors the shape of the
ISL optimization, so the same discussion applies.

\begin{figure}[H]
\includegraphics[width=\imgsizes]{figures/nlfm-kaiser-15us-opti-isl.png}
\caption{ISL optimization results for NLFM combined with Kaiser amplitude tapering. Amplitude tapering is very effective
for reducing the ISL, especially with larger $a$.}
\label{fig:nlfm-kaiser-15us-opti-isl}
\includegraphics[width=\imgsizes]{figures/nlfm-kaiser-15us-opti-psl.png}
\caption{PSL optimization results for NLFM combined with Kaiser amplitude tapering. Amplitude tapering with large $a$ is again very effective
for reducing the PSL.}
\label{fig:nlfm-kaiser-15us-opti-psl}
\end{figure}


The PPR in Figure \ref{fig:nlfm-kaiser-15us-opti-ppr} is the same for each waveform, because it only depends on the
shape of the amplitude tapering function determined by the Kaiser parameter. As described previously, the power
decreases rapidly as $\beta$ in the Kaiser window function increases.

Similar to the NLFM-only case in Section \ref{s:nlfm-only-results}, the range resolution decreases slightly as the NLFM
parameter $a$ increases. The same happens for an increasing Kaiser parameter $\beta$.  The effect of both together,
however, is compounded such that a large $\beta$ and large $a$ result in a decrease in range resolution that is much
larger than a mere linear superposition of both.


\begin{figure}[H]
\includegraphics[width=\imgsizes]{figures/nlfm-kaiser-15us-opti-ppr.png}
\caption{PPR optimization results for NLFM combined with Kaiser amplitude tapering. Since the PPR is only affected by
the Kaiser window parameter $\beta$, the PPR does not change with $a$ and is only shown with respect to $\beta$.}
\label{fig:nlfm-kaiser-15us-opti-ppr}
\includegraphics[width=\imgsizes]{figures/nlfm-kaiser-15us-opti-res.png}
\caption{Resolution optimization results for NLFM combined with Kaiser amplitude tapering. Notable is the compounded
decrease of the range resolution when combining NLFM with tapering.}
\label{fig:nlfm-kaiser-15us-opti-res}
\end{figure}




Figure \ref{fig:nlfm-kaiser-slices-15us} shows zero-Doppler cuts of a few waveforms with varying
values for $a$ and $\beta$. Similar to the untapered NLFM waveforms, the top left figure shows sidelobes of about
\SI{-40}{\decibel} with moderate tapering, with range resolution of about \SI{180}{\metre}.  With more aggressive
tapering, sidelobes decrease to about \SI{-70}{\decibel} while sacrificing range resolution, which is about
\SI{280}{\metre}, shown on the top right,

The bottom two figures show very low sidelobes, but fairly low range resolution at \SI{300}{\metre} and \SI{400}{\metre}
on the left and right, respectively. These are not ideal for weather radar applications because even though the
sidelobes are sufficiently low, range resolution is not fine enough.

\begin{figure}[H]
\includegraphics[width=\imgsize]{figures/nlfm-kaiser-slices-15us.png}
\caption{NLFM waveform zero-Doppler cuts with varying $a$ and $\beta$ parameters and pulse length
$\tau=\SI{15}{\micro\second}$. More aggressive tapering lowers
sidelobes but widens the mainlobe. Increasing $a$ has the same effect.}
\label{fig:nlfm-kaiser-slices-15us}
\end{figure}





\section{Effects of Impulse Response on the Ambiguity Function}
\label{s:ireffectsopti}
As discussed in Section \ref{s:ireffects}, as the return signal passes through the receive chain, it will be distorted
by the system impulse response. The effects of that on various waveforms are presented here. 
A couple of assumptions have been made. First, since the exact impulse response is unknown, a Rayleigh-shaped
approximation is used for $g(t)$.  Second, it is assumed that the impulse response is as long as the pulse, in this case
$\tau = \SI{15}{\micro\second}$, which is very likely not accurate. However, doing this will exaggerate the effect and
make it more easily visible.

\subsection{Rectangular Pulse}
Figure \ref{fig:pulsewir-15us} shows the AF of a waveform that has been distorted by a system with the impulse response shown in Figure
\ref{fig:tf1}. The goal of this section is to study the effects of a non-linear response of the system to pulse compression.
\begin{figure}[H]
\includegraphics[width=\imgsizes]{figures/tf1.png}
\caption{Rayleigh-shaped impulse response. It is applied to the signal over the signal's duration, one of the
assumptions made in this simulation.}
\label{fig:tf1}
\end{figure}

\begin{figure}[H]
\includegraphics[width=\imgsizes]{figures/pulsewoir-15us.png}
\caption{ An ideal AF for a rectangular pulse of length $\tau = \SI{15}{\micro\second}$. This is a reference to compare
to the AF that has been distorted by a system impulse response.}
\label{fig:pulsewoir-15us}
\includegraphics[width=\imgsizes]{figures/pulsewir-15us.png}
\caption{ AF of a rectangular pulse with system distortion. The asymmetry of the impulse response is reflected
in the asymmetric distortion in the AF. }
\label{fig:pulsewir-15us}
\end{figure}


As expected there is again no significant distortion in the Doppler dimension, which is the same as the simple
rectangular pulse. The only effect is in the range delay dimension. There, the asymmetric shape of the impulse response
is reflected in the range delay of the AF due to the waveform passing through the system. This distortion is not
desirable and should be corrected.

\subsection{Barker Code}
Figure \ref{fig:barkerwir-15us} shows the AF of a Barker 13 phase code with pulse length $\tau=\SI{15}{\micro\second}$.
Similar to a rectangular pulse, the asymmetry of the impulse response is reflected in the AF. Unlike the rectangular
pulse, the range delay is affected more by the distortion, severely impacting the range resolution.

\begin{figure}[H]
\includegraphics[width=\imgsizes]{figures/barker-15us-af.png}
\caption{ AF of a Barker 13 phase code without system distortion. Given as a reference to compare to the AF that
has been distorted by a system impulse response.}
\label{fig:barker-15us-af}
\includegraphics[width=\imgsizes]{figures/barkerwir-15us.png}
\caption{ AF of a Barker 13 phase code with system distortion. The asymmetry of the impulse response is reflected
in the asymmetric distortion in the AF. }
\label{fig:barkerwir-15us}
\end{figure}


\subsection{Frequency Modulation}
Figures \ref{fig:lfmwir-15us} and \ref{fig:nlfmwir-15us} show the AFs of an LFM and NLFM waveform, respectively, both
with pulse lengths $\tau=\SI{15}{\micro\second}$. The distortion of the impulse response smoothes out the sidelobes on
one side of the AF in the range delay direction, due to the impulse response being asymmetric. The peak sidelobe level
on either side of the mainlobes shows no significant difference, though they are lower than without impulse response
distortion. For the LFM waveform, the smoothing almost eliminates the sidelobes, whereas they are still clearly visible
for the NLFM waveform.

\begin{figure}[H]
\includegraphics[width=\imgsizes]{figures/lfm-15us-af.png}
\caption{ AF of an LFM waveform without system distortion. Given as a reference to compare to the AF that
has been distorted by a system impulse response.}
\label{fig:lfm-15us-af}
\includegraphics[width=\imgsizes]{figures/lfmwir-15us.png}
\caption{ AF of an LFM waveform with system distortion. The asymmetry of the impulse response is reflected
in the asymmetric distortion in the AF. }
\label{fig:lfmwir-15us}
\end{figure}




\begin{figure}[H]
\includegraphics[width=\imgsizes]{figures/nlfm-15us-af.png}
\caption{ AF of a NLFM waveform without system distortion. Given as a reference to compare to the AF that
has been distorted by a system impulse response.}
\label{fig:nlfm-15us-af}
\includegraphics[width=\imgsizes]{figures/nlfmwir-15us.png}
\caption{ AF of a NLFM waveform with system distortion. The asymmetry of the impulse response is reflected
in the asymmetric distortion in the AF. }
\label{fig:nlfmwir-15us}
\end{figure}

Generally, a system impulse response has no significant effect on the Doppler distortion for weather radar applications.
However, it does affect the range delay by smoothing out the waveform similar to amplitude tapering, lowering range
resolution. The distortion is not symmetric because the system impulse response is not, which is the most significant
result presented here. Despite the benefits of smoothing and sidelobe reduction caused by the impulse response, the
asymmetry is highly undesirable and should be corrected. One way of doing that is to predistort the waveform before
transmitting it. Figure \ref{fig:predistortblock} shows a block diagram explaining the concept of predistortion. Before
passing into the matched filter, whose impulse response is denoted by $h(t)$, the waveform passes through the other
parts of the receive chain, whose impulse response is given by $g(t)$, distorting the received waveform. To correct for
that effect, the waveform is distorted with the inverse of the system impulse response $g^{-1}(t)$ before it leaves the
transmitter chain, so that, on receive, the effect of $g(t)$ is cancelled out, because $g(t) * g^{-1}(t) =
\delta$, where $\delta$ is the Dirac delta function.

\begin{figure}[H]
\includegraphics[width=\imgsize]{figures/predistortblock.png}
\caption{Block diagram describing the concept of predistortion. The waveform is distorted by the inverse of the system
impulse response $g^{-1}(t)$ before passing into the matched filter, whose response is denoted as $h(t)$.}
\label{fig:predistortblock}
\end{figure}
