\section{Setup and Configuration}


\subsection{Generating Waveforms}
The simulation is done in multiple parts. First, a waveform is generated based on a given
amplitude, phase, frequency modulation, and impulse response that is applied to the signal.
The returned waveform is properly sampled based on the desired signal bandwidth to avoid aliasing.

If an impulse response was supplied, we get both the distorted and clean waveform back, as well as
the new pulse length $\tau$ which will be approximately twice the original $\tau$. That lengthening
is due to the convolution of the original waveform $u(t)$ with the impulse response $h(t)$.

\subsection{Calculate the Ambiguity Function}
Once we have the waveform, we generate the AF and plot it as a surface plot. There are three components
to the AF: 

\begin{enumerate}
\item the original waveform, $u(t)$
\item the shifted complex conjugate of the waveform, $u^*(t+\tau)$
\item the Doppler shift, $e^{j2\pi f_d t}$
\end{enumerate}
These components are multiplied and the result summed up. To do that, we convert $u(t)$ into a diagonal
matrix, so that each data point of the signal is on its own row. The shifted complex conjugate is also 
a matrix, where each row contains the flipped signal shifted a little further. If we ignore the Doppler
shift for a moment and multiply what we have so far and sum over it, we get the auto-correlation of $u(t)$,
which is easy to verify.

Next we calculate the Doppler shifts for a given range of frequencies over the time $t$. The multiplication of 
$f_d$ and $t$ results in a matrix where each row corresponds to a certain Doppler shift. The first row at 
$f_d = 0$ gives us a vector of ones, as we would expect, since there is no Doppler shift and the signal is
unchanged.

Finally, we multiply all three components together and sum to get the AF in a 2D matrix that we can visualize
as a surface plot.
Note that the original signal $u(t)$ could be distorted by an arbitrary impulse response if we decide to include
that aspect in the simulation, whereas the shifted complex conjugate part is always based on the clean, original
signal.








\section{Simple pulse}
A pulse is the simplest waveform for a radar. These graphs show the AF for pulses of varying lengths so we can 
evaluate the effect of pulse length on Doppler resistance. As we predicted in \ref{s:dopplereffect}, if the 
pulse length is much smaller than the inverse of the bandwidth, the Doppler effect 

\begin{figure}[H]
\includegraphics[width=\imgsize]{figures/pulse-1us.png}
\caption{ Simple pulse of length 1 $\mu s$ }
\label{fig:simplepulse-1us}
\end{figure}

\begin{figure}[H]
\includegraphics[width=\imgsize]{figures/pulse-200us.png}
\caption{ Simple pulse of length 200 $\mu s$ }
\label{fig:simplepulse-200us}
\end{figure}









\section{Phase-coded Barker code}
Next we will examine a length 13 Barker code of different pulse lengths. Again we notice that
Doppler performance gets worse with a longer pulse.

\begin{figure}[H]
\includegraphics[width=\imgsize]{figures/barker-1us.png}
\caption{ Barker 13 of length 1 $\mu s$ }
\label{fig:barker13-1us}
\end{figure}

\begin{figure}[H]
\includegraphics[width=\imgsize]{figures/barker-200us.png}
\caption{ Barker 13 of length 200 $\mu s$ }
\label{fig:barker13-200us}
\end{figure}











\section{Linear Frequency Modulated waveform}
Here we will look at a linearly frequency modulated waveform. As mentioned before, LFM is a property of the
AF and thus very simple simple to implement in analog hardware. Also, the figures below show that the Doppler
performance of an LFM waveform is very high, especially for our weather radar application. There is no visible
difference between the AFs in the Doppler dimension.

The frequency modulation function is shown in figure \ref{fig:lfm-fmf}.

\begin{figure}[H]
\includegraphics[width=\imgsize]{figures/lfm-fmf.png}
\caption{Linear frequency modulation function}
\label{fig:lfm-fmf}
\end{figure}

\begin{figure}[H]
\includegraphics[width=\imgsize]{figures/lfm-15us.png}
\caption{ LFM of length 15 $\mu s$ }
\label{fig:lfm-15us}
\end{figure}

\begin{figure}[H]
\includegraphics[width=\imgsize]{figures/lfm-200us.png}
\caption{ LFM of length 200 $\mu s$ }
\label{fig:lfm-200us}
\end{figure}











\section{Linear Frequency Modulated waveform with amplitude weighting}
The AF of a LFM waveform has fairly high sidelobes that we can try to mitigate with amplitude weighting.
For that, we'll apply a Hamming window to the amplitude. As expected, the sidelobes decrease, but the main
lobe width increases.

\begin{figure}[H]
\includegraphics[width=\imgsize]{figures/lfm-15us-hamming.png}
\caption{ LFM of length 15 $\mu s$ with Hamming amplitude weighting }
\label{fig:lfm-15us-hamming}
\end{figure}


\begin{figure}[H]
\includegraphics[width=\imgsize]{figures/lfm-200us-hamming.png}
\caption{ LFM of length 200 $\mu s$ with Hamming amplitude weighting }
\label{fig:lfm-200us-hamming}
\end{figure}












\section{Quadratic Frequency Modulated waveform}
Here we will look at a quadradicly frequency modulated waveform. The figures below show that the Doppler
performance of a QFM waveform is very high, especially for our weather radar application. There is no visible
difference between the AFs in the Doppler dimension.
The frequency modulation function is shown in figure \ref{fig:qfm-fmf}.

\begin{figure}[H]
\includegraphics[width=\imgsize]{figures/qfm-fmf.png}
\caption{Linear frequency modulation function}
\label{fig:qfm-fmf}
\end{figure}

\begin{figure}[H]
\includegraphics[width=\imgsize]{figures/qfm-15us.png}
\caption{ QFM of length 15 $\mu s$ }
\label{fig:qfm-15us}
\end{figure}

\begin{figure}[H]
\includegraphics[width=\imgsize]{figures/qfm-200us.png}
\caption{ QFM of length 200 $\mu s$ }
\label{fig:qfm-200us}
\end{figure}











\section{Quadratic Frequency Modulated waveform with amplitude weighting}
Similar to the LFM waveforms, we can decrease sidelobes by applying a Hamming window to the waveform amplitude
at the expense of a wider main lobe.

\begin{figure}[H]
\includegraphics[width=\imgsize]{figures/qfm-15us-hamming.png}
\caption{ QFM of length 15 $\mu s$ with Hamming amplitude weighting }
\label{fig:qfm-15us-hamming}
\end{figure}

\begin{figure}[H]
\includegraphics[width=\imgsize]{figures/qfm-200us-hamming.png}
\caption{ QFM of length 200 $\mu s$ with Hamming amplitude weighting }
\label{fig:qfm-200us-hamming}
\end{figure}











\section{Hann-shaped Frequency Modulated waveform}
The following waveform is frequency modulated according to the Hann function, 
shown in figure \ref{fig:hannfm-fmf}.

\begin{figure}[H]
\includegraphics[width=\imgsize]{figures/hannfm-fmf.png}
\caption{Hann frequency modulation function}
\label{fig:hannfm-fmf}
\end{figure}


\begin{figure}[H]
\includegraphics[width=\imgsize]{figures/hannfm-15us.png}
\caption{ Hann-shaped FM of length 15 $\mu s$}
\label{fig:hannfm-15us}
\end{figure}

\begin{figure}[H]
\includegraphics[width=\imgsize]{figures/hannfm-200us.png}
\caption{ Hann-shaped FM of length 200 $\mu s$}
\label{fig:hannfm-200us}
\end{figure}










\section{Non-Linear Frequency Modulation}
In this case, we vary the frequency of the signal non-linearly. Specifically, we let

\begin{equation}
  f(t) = a x^7 + x,
\end{equation}

and vary $a$ from 0 to 10. The two extremes for $a=0$ and $a=10$ are shown in figure \ref{fig:nlfm-functions}. 
Each function is applied over the pulse length $\tau$ of the signal.

\begin{figure}[H]
\includegraphics[width=\imgsize]{figures/nlfm-functions.png}
\caption{Functions describing the non-linear frequency sweeps}
\label{fig:nlfm-functions}
\end{figure}

For each function a waveform and ambiguity functions is generated and then evaluated according to the optimizations described
in section \ref{s:quantifications}. We can then determine which waveform is best suited for our application in weather radars
or further change the setup to combine it with, e.g. amplitude weighting to achieve even better performance.







\section{Effects of Impulse Response on the Ambiguity Function}

As the signal returns to the radar and passes through the receive chain, it will be distorted by the
impulse response of the filter and amplifier. Figure \ref{fig:pulsewoir-200us} shows a standard pulse
that is undistorted. 

\begin{figure}[!ht]
\includegraphics[width=\imgsize]{figures/pulsewoir-200us.png}
\caption{ Pulse without IR distortion}
\label{fig:pulsewoir-200us}
\end{figure}


On the other hand, figure \ref{fig:pulsewir-200us} shows the AF for a pulse that
has been distorted by the transfer function shown in figure \ref{fig:tf1}. 

\begin{figure}[!ht]
\includegraphics[width=\imgsize]{figures/pulsewir-200us.png}
\caption{ Pulse with IR distortion}
\label{fig:pulsewir-200us}
\end{figure}

\begin{figure}[!ht]
\includegraphics[width=\imgsize]{figures/tf1.png}
\caption{Transfer Function}
\label{fig:tf1}
\end{figure}

The distortion lengthened the signal and smoothed it out a bit.


