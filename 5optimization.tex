In this section the setup and results of the simulations will be presented. First, the way the 
simulations were conducted is explained. After that, the results for various waveforms will be 
presented.

\section{Setup and Configuration}
All simulations were performed with the MATLAB\textsuperscript{\textregistered} software package.
The general algorithm to compute the ambiguity function is presented, with the complete code
given in Appendix \ref{app:code}.


\subsection{Generating Waveforms}
The simulation is done in multiple parts. First, a waveform is generated based on a given
amplitude, phase, frequency modulation, and impulse response that is applied to the signal.
The returned waveform is generated at a proper sampling frequency in order to avoid aliasing.

If an impulse response was supplied, both the distorted and clean waveform will be returned, as well as
the new pulse length $\tau$ which will be approximately twice the original $\tau$. That lengthening
is due to the convolution of the original waveform $u(t)$ with the impulse response $h(t)$.

\subsection{Calculating the Ambiguity Function}
Once the waveform has been returned, the AF is generated and plotted in a surface plot. 
In pseudo-code, given the complex signal \emph{signal} and an array of Doppler frequencies \emph{doppler\_frequencies},
the normalized AF is computed as follows.

\lstset{
  basicstyle=\footnotesize
}
\begin{lstlisting}[language=Ruby]
af = []
for dfreq in doppler_freqencies do
  doppler_shift = exp(-j*2*pi*dfreq)
  shifted_signal = signal*doppler_shift
  af << abs(xcorr(signal, shifted_signal))
end
return (af / max(af))

\end{lstlisting}









\section{Simple pulse}
A rectangular pulse is the simplest waveform for a radar. These graphs show the AF for pulses of varying lengths so 
that the effect of pulse length on Doppler resistance can be examined. Figure \ref{fig:simplepulse-1us} shows the AF
of a \SI{1}{\micro\second} pulse. Across the delay axis, the characteristic triangle shape can be seen. Across the 
Doppler axis, there is no visible change in the AF. The different Doppler cuts are identical, so for weather radar 
applications, the commonly encountered Doppler shifts do not cause any distortion to the signal for short pulses.

Similarly, Figure \ref{fig:simplepulse-200us} shows the AF of a much longer, \SI{200}{\micro\second} pulse. Comparing
it to the shorter pulse, it looks very similar. However, unlike for a short pulse, there is some distortion at higher
Doppler velocities. This distortion is visibly minimal, so the impact on weather radar application is negligible because,
first, the distortion doesn't occur until higher velocity weather targets, and second, pulse lengths of \SI{200}{\micro\second}
are not common. For comparison, the two pulse lengths used in the WSR-88D radar network are \SI{1.57}{\micro\second} and 
\SI{4.57}{\micro\second} \citep[p. 47]{Doviak:2006}.




\begin{figure}[H]
\includegraphics[width=\imgsize]{figures/pulse-1us.png}
\caption{ Simple pulse of length 1 $\mu s$ }
\label{fig:simplepulse-1us}
\end{figure}

\begin{figure}[H]
\includegraphics[width=\imgsize]{figures/pulse-200us.png}
\caption{ Simple pulse of length 200 $\mu s$ }
\label{fig:simplepulse-200us}
\end{figure}









\section{Barker code}
Next, a length 13 Barker code of different pulse lengths will be examined. Figures \ref{fig:barker13-1us} and
\ref{fig:barker13-1us-0D} show a \SI{1}{\micro\second} pulse that has been phase coded with a length 13 Barker 
code. As for a rectangular pulse, there is no visible degradation of the signal with increasing Doppler shifts.
The zero-Doppler cut shows the typical shape of the Barker code and sidelobes of about
\SI{-22}{\decibel} \citep[p. 351]{Skolnik:2001}.

Comparing these images to Figures \ref{fig:barker13-200us} and \ref{fig:barker13-200us-0D} shows little difference
in the Doppler dimension, confirming again that for weather radar applications, the Doppler distortions can be 
neglected. The only difference is the expected loss of range resolution caused by the longer pulse.

\begin{figure}[H]
\includegraphics[width=\imgsize]{figures/barker-1us.png}
\caption{ Barker 13 of length 1 $\mu s$ }
\label{fig:barker13-1us}
\includegraphics[width=\imgsize]{figures/barker-1us-0D.png}
\caption{Zero Doppler cut of the AF}
\label{fig:barker13-1us-0D}
\end{figure}

\begin{figure}[H]
\includegraphics[width=\imgsize]{figures/barker-200us.png}
\caption{ Barker 13 of length 200 $\mu s$ }
\label{fig:barker13-200us}
\includegraphics[width=\imgsize]{figures/barker-200us-0D.png}
\caption{Zero Doppler Cut of the AF}
\label{fig:barker13-200us-0D}
\end{figure}











\section{Linear Frequency Modulation}
Here the results of different variations of linear frequency modulation will be presented, specifically LFM with
and without amplitude tapering.

\subsection{Linear Frequency Modulation}
In this section the performance results of a linearly frequency modulated waveform will be presented.

The frequency modulation function is shown in figure \ref{fig:lfm-fmf}. It is a linear sweep from \SI{-2.5}{\mega\hertz} to
\SI{2.5}{\mega\hertz}, over a total of \SI{5}{\mega\hertz}.

\begin{figure}[H]
\includegraphics[width=\imgsize]{figures/lfm-fmf.png}
\caption{Linear frequency modulation function}
\label{fig:lfm-fmf}
\end{figure}


Figures \ref{fig:lfm-15us} and \ref{fig:lfm-200us} show the AFs of two LFM waveforms, of \SI{15}{\micro\second} and
\SI{200}{\micro\second} lengths, respectively.
Comparing the two for differences shows once again that in the Doppler dimension, the pulse length has
little effect for weather radar applications. 

One observation of significance is the range resolution of both signals.
Figures \ref{fig:lfm-15us-0d} and \ref{fig:lfm-200us-0d} show the zero-Doppler cut of both signals.
As expected, the sidelobes are around -13 dB in both AFs, but also, the shapes are approximately the same
for each case, which means that the pulse length has little effect on the range resolution. Instead, range
resolution is determined by the sweep bandwidth, not the pulse length. 

\begin{figure}[H]
\includegraphics[width=\imgsize]{figures/lfm-15us.png}
\caption{ LFM of length 15 $\mu s$ }
\label{fig:lfm-15us}
\includegraphics[width=\imgsize]{figures/lfm-200us.png}
\caption{ LFM of length 200 $\mu s$ }
\label{fig:lfm-200us}
\end{figure}



\begin{figure}[H]
\includegraphics[width=\imgsize]{figures/lfm-15us-0D.png}
\caption{ LFM of length 15 $\mu s$, zero Doppler cut. }
\label{fig:lfm-15us-0d}
\includegraphics[width=\imgsize]{figures/lfm-200us-0D.png}
\caption{ LFM of length 200 $\mu s$, zero Doppler cut. }
\label{fig:lfm-200us-0d}
\end{figure}


\subsection{Linear Frequency Modulation with Amplitude Tapering}
LFM has the disadvantage of very high sidelobes at about \SI{-13}{\decibel}.  
These can be mitigate with amplitude tapering, so a Kaiser window is applied to the amplitude. 

Based on Equation \eqref{eq:kaiser2}, the parameter $\beta$ is
varied from 0 to 50, and the AF generated and evaluated for each by quantifiying its performance according to the metrics
described in Section \ref{s:quantifications}. The results are shown in figure \ref{fig:lfm-isl-15us-opti}.

As expected, the ISL dramatically decreases, as does the MSL. At the same time, the range resolution decreases from
an initial 40 m to about 200 m. The PPR may be the limiting metric here because the tapered signal quickly
loses half of its energy with only minor amplitude tapering and eventually decreases to about \SI{20}{\percent} of
the original amplitude, which means the signal loses \SI{80}{\percent} of its energy. This, in turn, means reduced sensitivity.



\begin{figure}[H]
\includegraphics[width=\imgsize]{figures/lfm-isl-15us-opti.png}
\caption{Optimization results for LFM with Kaiser amplitude tapering}
\label{fig:lfm-isl-15us-opti}
\end{figure}


Clearly, a tradeoff must be made. If the half power point is selected, the waveform would have an ISL of about \SI{-20}{\decibel}, 
MSL of \SI{-33}{\decibel}, and a range resolution of \SI{68}{\metre}.

The corresponding zero Doppler cut of the AF at this point is shown in figure \ref{fig:lfm-isl-15us-opti-slice13}. It 
clearly shows the sidelobes at about -70 dB, which means this waveform would be adequate for typical weather observations.


\begin{figure}[H]
\includegraphics[width=\imgsize]{figures/lfm-isl-15us-opti-slice13.png}
\caption{Zero Doppler cut of the AF at the half power point.}
\label{fig:lfm-isl-15us-opti-slice13}
\end{figure}






\section{Non-Linear Frequency Modulation}
As described in Section \ref{s:nlfm}, the frequency of a frequency modulated signal can be swept non-linearly as well. In
order to find more useful waveforms for weather radar applications, this work evaluates several different sweep functions 
described by Equation \eqref{eq:nlfm-sweep} below. The general form is

\begin{equation}
\label{eq:nlfm-sweep}
  f(t) = a t^7 + t,
\end{equation}
where $a$ varies from 0 to 10. Whole functions for different values of $a$ are shown in figure \ref{fig:nlfm-functions}. 
Each function is applied over the pulse length $\tau$ of the signal.

\begin{figure}[H]
\includegraphics[width=\imgsize]{figures/nlfm-functions.png}
\caption{Functions describing the non-linear frequency sweeps}
\label{fig:nlfm-functions}
\end{figure}

For each function a waveform and ambiguity functions are generated and then evaluated according to the optimizations described
in Section \ref{s:quantifications}. It can then be determined which sweep function is best suited for weather radar applications, 
or further changes to the setup can be made to combine it with, e.g. amplitude tapering to achieve even better performance.


\subsection{Parameterized NLFM Results}
\label{s:nlfm-only-results}
Figure \ref{fig:nlfm-side-15us-opti} shows the results for varying NLFM waveforms in regards to ISL, MSL, and range resolution.
We find that the closer the frequency modulation function approaches an S-curve, the better the ISL becomes, though it is never
fantastic. The maximum ISL seems to be around \SI{-15}{\decibel}. The maximum sidelobe level decreases rapidly first to about 
\SI{-13}{\decibel} and then slowly decreases to about \SI{-20}{\decibel}, where it stays fairly constant. The range resolution
gets progressively worse, starting at about \SI{30}{\metre} and increasing to about \SI{210}{\metre}, meaning the mainlobe broadens.
This also explains the decreasing and eventually constant MSL, because the whole AF is one big mainlobe with only minimal sidelobes.

\begin{figure}[H]
\includegraphics[width=\imgsize]{figures/nlfm-side-15us-opti.png}
\caption{Optimization results for NLFM}
\label{fig:nlfm-side-15us-opti}
\end{figure}

To get a better picture about NLFM waveforms, consider Figure \ref{fig:nlfm-side-15us-0D-1}, showing the zero-Doppler slice of
the NLFM waveform generated with the sweep function with $a=2.9293$. The mainlobe is fairly wide, though only about \SI{50}{\metre}
at the \SI{3}{\decibel} point. The sidelobes are better than those of the LFM waveform at slightly higher than \SI{-30}{\decibel}.
In comparison, Figure \ref{fig:nlfm-side-15us-0D-2} shows the NLFM waveform generated with the sweep function with $a=6.46$. 
Altough the sidelobes are a little lower at \SI{-40}{\decibel} and the range resolution at the \SI{3}{\decibel} point is still 
only about \SI{70}{\metre}, the mainlobe quickly becomes very wide, making it not ideal for weather radar applications.

\begin{figure}[H]
\includegraphics[width=\imgsize]{figures/nlfm-side-15us-0D-1.png}
\caption{Zero-Doppler slice for AF with $a=2.92$}
\label{fig:nlfm-side-15us-0D-1}
\includegraphics[width=\imgsize]{figures/nlfm-side-15us-0D-2.png}
\caption{Zero-Doppler slice for AF with $a=6.46$}
\label{fig:nlfm-side-15us-0D-2}
\end{figure}


\subsection{Parameterized NLFM with Amplitude Tapering}
In order to improve the performance of the varying NLFM waveforms, Kaiser windows are applied to taper the amplitude.
The results are presented in Figures \ref{fig:nlfm-kaiser-15us-opti-isl}-\ref{fig:nlfm-kaiser-15us-opti-res}.

Looking at the ISL in figure \ref{fig:nlfm-kaiser-15us-opti-isl}, it can be seen that the waveform that is most aggressively 
tapered generally has better ISL performance. The same is true for the NLFM parameter $a$. 
Next, the MSL shown in figure \ref{fig:nlfm-kaiser-15us-opti-msl} closely mirrors the shape of the ISL optimization,
so the same discussion applies to it.

\begin{figure}[H]
\includegraphics[width=\imgsize]{figures/nlfm-kaiser-15us-opti-isl.png}
\caption{ISL optimization results for NLFM combined with Kaiser amplitude tapering}
\label{fig:nlfm-kaiser-15us-opti-isl}
\includegraphics[width=\imgsize]{figures/nlfm-kaiser-15us-opti-msl.png}
\caption{MSL optimization results for NLFM combined with Kaiser amplitude tapering}
\label{fig:nlfm-kaiser-15us-opti-msl}
\end{figure}


The PPR in Figure \ref{fig:nlfm-kaiser-15us-opti-ppr} is the same for each waveform, because it only depends on 
the shape of the amplitude tapering function determined by the Kaiser parameter. As described previously, the power decreases rapidly
as $\beta$ in the Kaiser window function increases.

Similar to the NLFM-only case in Section \ref{s:nlfm-only-results}, the range resolution
decreases slightly as the NLFM parameter $a$ increases. The same happens for an increasing Kaiser parameter $\beta$.
The effect of both together, however, is compounded such that a large $\beta$ and large $a$ result in a decrease in range
resolution that is much larger than a mere linear superposition of both individually.


\begin{figure}[H]
\includegraphics[width=\imgsize]{figures/nlfm-kaiser-15us-opti-ppr.png}
\caption{PPR optimization results for NLFM combined with Kaiser amplitude tapering}
\label{fig:nlfm-kaiser-15us-opti-ppr}
\includegraphics[width=\imgsize]{figures/nlfm-kaiser-15us-opti-res.png}
\caption{Resolution optimization results for NLFM combined with Kaiser amplitude tapering}
\label{fig:nlfm-kaiser-15us-opti-res}
\end{figure}



To give some examples, Figure \ref{fig:nlfm-kaiser-slices-15us} shows zero-Doppler cuts of a few waveforms with
varying values for $a$ and $\beta$. Similar to the untapered NLFM waveforms, the top left figure shows sidelobes of about
\SI{-40}{\decibel} with only mild tapering, with range resolution of about \SI{180}{\metre}. 
With more aggressive tapering, sidelobes decrease to about \SI{-70}{\decibel}
while sacrificing range resolution, which is about \SI{280}{\metre}, shown on the top right,

The bottom two figures show very low sidelobes, but fairly low range resolution at \SI{300}{\metre} and \SI{400}{\metre} on 
the left and right, respectively. These are not ideal for weather radar applications because even though the sidelobes are
sufficiently low, range resolution is not fine enough.

\begin{figure}[H]
\includegraphics[width=\imgsize]{figures/nlfm-kaiser-slices-15us.png}
\caption{NLFM waveform zero-Doppler cuts with varying a and $\beta$ parameters}
\label{fig:nlfm-kaiser-slices-15us}
\end{figure}





\section{Effects of Impulse Response on the Ambiguity Function}
As discussed in Section \ref{s:ireffects}, as the return signal passes through the receive chain, it will be distorted by the
impulse response of the filter and amplifier. The effects of that on the waveform are presented here.
To provide a baseline, Figure \ref{fig:pulsewoir-15us} shows a rectangular pulse that is undistorted as a reference.

\begin{figure}[!ht]
\includegraphics[width=\imgsize]{figures/pulsewoir-15us.png}
\caption{ Rectangular pulse without IR distortion}
\label{fig:pulsewoir-15us}
\end{figure}


Figure \ref{fig:pulsewir-15us} shows the AF for a pulse that
has been distorted by the transfer function shown in figure \ref{fig:tf1}. Since the exact shape of the impulse response of 
the PX1000 mobile radar is still unknown, it is approximated with a Rayleigh-shaped function.

\begin{figure}[!ht]
\includegraphics[width=\imgsize]{figures/tf1.png}
\caption{Impulse Response}
\label{fig:tf1}
\end{figure}

\begin{figure}[!ht]
\includegraphics[width=\imgsize]{figures/pulsewir-15us.png}
\caption{ Pulse with IR distortion}
\label{fig:pulsewir-15us}
\end{figure}


The result is not unexpected. First, there is again
no visible distortion in the Doppler dimension, which is the same as the simple rectangular pulse. The only effect is in the 
range delay dimension. There, the asymmetric shape of the impulse response is reflected in the range delay of the AF. This
distortion is not desirable and should be corrected.


