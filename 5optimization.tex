\section{Setup and Configuration}


\subsection{Generating Waveforms}
The simulation is done in multiple parts. First, a waveform is generated based on a given
amplitude, phase, frequency modulation, and impulse response that is applied to the signal.
The returned waveform is properly sampled based on the desired signal bandwidth to avoid aliasing.

If an impulse response was supplied, we get both the distorted and clean waveform back, as well as
the new pulse length $\tau$ which will be approximately twice the original $\tau$. That lengthening
is due to the convolution of the original waveform $u(t)$ with the impulse response $h(t)$.

\subsection{Calculate the Ambiguity Function}
Once we have the waveform, we generate the AF and plot it as a surface plot. There are three components
to the AF: 

\begin{enumerate}
\item the original waveform, $u(t)$
\item the shifted complex conjugate of the waveform, $u^*(t+\tau)$
\item the Doppler shift, $e^{j2\pi f_d t}$
\end{enumerate}
These components are multiplied and the result summed up. To do that, we convert $u(t)$ into a diagonal
matrix, so that each data point of the signal is on its own row. The shifted complex conjugate is also 
a matrix, where each row contains the flipped signal shifted a little further. If we ignore the Doppler
shift for a moment and multiply what we have so far and sum over it, we get the auto-correlation of $u(t)$,
which is easy to verify.

Next we calculate the Doppler shifts for a given range of frequencies over the time $t$. The multiplication of 
$f_d$ and $t$ results in a matrix where each row corresponds to a certain Doppler shift. The first row at 
$f_d = 0$ gives us a vector of ones, as we would expect, since there is no Doppler shift and the signal is
unchanged.

Finally, we multiply all three components together and sum to get the AF in a 2D matrix that we can visualize
as a surface plot.
Note that the original signal $u(t)$ could be distorted by an arbitrary impulse response if we decide to include
that aspect in the simulation, whereas the shifted complex conjugate part is always based on the clean, original
signal.








\section{Simple pulse}
A pulse is the simplest waveform for a radar. These graphs show the AF for pulses of varying lengths so we can 
evaluate the effect of pulse length on Doppler resistance. As we predicted in \ref{s:dopplereffect}, if the 
pulse length is much smaller than the inverse of the bandwidth, the Doppler effect 

\begin{figure}[H]
\includegraphics[width=\imgsize]{figures/pulse-1us.png}
\caption{ Simple pulse of length 1 $\mu s$ }
\label{fig:simplepulse-1us}
\end{figure}

\begin{figure}[H]
\includegraphics[width=\imgsize]{figures/pulse-200us.png}
\caption{ Simple pulse of length 200 $\mu s$ }
\label{fig:simplepulse-200us}
\end{figure}









\section{Phase-coded Barker code}
Next we will examine a length 13 Barker code of different pulse lengths. Again we notice that
Doppler performance gets worse with a longer pulse.

\begin{figure}[H]
\includegraphics[width=\imgsize]{figures/barker-1us.png}
\caption{ Barker 13 of length 1 $\mu s$ }
\label{fig:barker13-1us}
\end{figure}

\begin{figure}[H]
\includegraphics[width=\imgsize]{figures/barker-200us.png}
\caption{ Barker 13 of length 200 $\mu s$ }
\label{fig:barker13-200us}
\end{figure}











\section{Linear Frequency Modulation}
\subsection{Linear Frequency Modulation}
Here we will look at a linearly frequency modulated waveform. As mentioned before, LFM is a property of the
AF and thus very simple simple to implement in analog hardware. Also, the figures below show that the Doppler
performance of an LFM waveform is very high, especially for our weather radar application. There is no visible
difference between the AFs in the Doppler dimension.

The frequency modulation function is shown in figure \ref{fig:lfm-fmf}.

\begin{figure}[H]
\includegraphics[width=\imgsize]{figures/lfm-fmf.png}
\caption{Linear frequency modulation function}
\label{fig:lfm-fmf}
\end{figure}


Comparing the two AFs to determine the difference that the pulse length makes, we see that in the Doppler
direction, the waveform does not change significantly. To get a better understanding of the difference 
in range resolution, let's look at the zero Doppler cut in figures \ref{fig:lfm-15us-0d} and \ref{fig:lfm-200us-0d}.

We see the expected shape with sidelobes around -13 dB in both AFs, and the shapes are approximately the same
for each case, which means that the pulse length has little effect on the range resolution, which is determined
by the bandwidth swept instead. These figures confirm that observation.

\begin{figure}[H]
\includegraphics[width=\imgsize]{figures/lfm-15us.png}
\caption{ LFM of length 15 $\mu s$ }
\label{fig:lfm-15us}
\includegraphics[width=\imgsize]{figures/lfm-200us.png}
\caption{ LFM of length 200 $\mu s$ }
\label{fig:lfm-200us}
\end{figure}



\begin{figure}[H]
\includegraphics[width=\imgsize]{figures/lfm-15us-0D.png}
\caption{ LFM of length 15 $\mu s$, zero Doppler cut. }
\label{fig:lfm-15us-0d}
\includegraphics[width=\imgsize]{figures/lfm-200us-0D.png}
\caption{ LFM of length 200 $\mu s$, zero Doppler cut. }
\label{fig:lfm-200us-0d}
\end{figure}


\subsection{Linear Frequency Modulation with Amplitude Weighting}
The AF of a LFM waveform has fairly high sidelobes that we can try to mitigate with amplitude weighting.
For that, we'll apply a Kaiser window to the amplitude. We vary the parameter for the Kaiser function
from 0 to 50, generate the AF for each, and evaluate it. The results are shown in figure \ref{fig:lfm-isl-15us-opti}.
As expected, the ISL dramatically decreases, as does the MSL. At the same time, the range resolution decreases from
an initial 60 m to about 400 m. The pulse power ratio may be the limiting metric here. The weighted signal quickly
loses half of its energy with only minor amplitude weighting and eventually decreases to about 0.2, which means
the signal loses 80\% of its energy. 



\begin{figure}[H]
\includegraphics[width=\imgsize]{figures/lfm-isl-15us-opti.png}
\caption{Optimization results for LFM with Kaiser amplitude weighting}
\label{fig:lfm-isl-15us-opti}
\end{figure}


Clearly, a tradeoff must be made. If we choose the half power point, we end up with an ISL of about -22 dB, 
MSL of -35 dB, and a range resolution of 151 m.

The corresponding zero Doppler cut of the AF at this point is shown in figure \ref{fig:lfm-isl-15us-opti-slice13}.
The sidelobes are fairly low at about -70 dB with acceptable range resolution and may be a good candidate to use.


\begin{figure}[H]
\includegraphics[width=\imgsize]{figures/lfm-isl-15us-opti-slice13.png}
\caption{Zero Doppler cut of the AF at the half power point.}
\label{fig:lfm-isl-15us-opti-slice13}
\end{figure}






\section{Non-Linear Frequency Modulation}
In this case, we vary the frequency of the signal non-linearly. Specifically, we let

\begin{equation}
  f(t) = a t^7 + t,
\end{equation}

and vary $a$ from 0 to 10. The two extremes for $a=0$ and $a=10$ are shown in figure \ref{fig:nlfm-functions}. 
Each function is applied over the pulse length $\tau$ of the signal.

\begin{figure}[H]
\includegraphics[width=\imgsize]{figures/nlfm-functions.png}
\caption{Functions describing the non-linear frequency sweeps}
\label{fig:nlfm-functions}
\end{figure}

For each function a waveform and ambiguity functions is generated and then evaluated according to the optimizations described
in section \ref{s:quantifications}. We can then determine which waveform is best suited for our application in weather radars
or further change the setup to combine it with, e.g. amplitude weighting to achieve even better performance.


\subsection{Parameterized NLFM Results}
\label{s:nlfm-only-results}
Figure \ref{fig:nlfm-side-15us-opti} shows the results for varying NLFM waveforms in regards to ISL, MSL, and range resolution.
We find that the closer the frequency modulation function approaches an S-curve, the better the ISL becomes, though it's never
fantastic. The maximum ISL seems to be around -13 dB. The maximum sidelobe level decreases rapidly first to about -15 dB and
mostly decreases to about -20 dB, where it stays constant. The range resolution gets progressively coarser, starting at about 50 m
and increasing to about 500 m, meaning the mainlobe broadens. This also explains the decreasing and eventually constant MSL,
because the whole AF is taken up by the mainlobe with only minimal sidelobes.

\begin{figure}[H]
\includegraphics[width=\imgsize]{figures/nlfm-side-15us-opti.png}
\caption{Optimization results for NLFM}
\label{fig:nlfm-side-15us-opti}
\end{figure}




\subsection{Parameterized NLFM with Amplitude Weighting}
Next we try to optimize varying NLFM waveforms combined with Kaiser windowing. First,
look at the ISL in figure \ref{fig:nlfm-kaiser-15us-opti-isl}. Intuitively we find that the waveform that is most aggressively 
weighted generally has better ISL performance. The same seems true for the NLFM parameter. The ideal range for the NLFM parameter
looks to be around 3 all the way up to 10, with as low a Kaiser parameter as we can live with.

Next, let's look at the MSL in figure \ref{fig:nlfm-kaiser-15us-opti-msl}. The shape closely mirrors that of the ISL optimization,
so the same discussion applies to it.

\begin{figure}[H]
\includegraphics[width=\imgsize]{figures/nlfm-kaiser-15us-opti-isl.png}
\caption{ISL optimization results for NLFM combined with Kaiser amplitude weighting}
\label{fig:nlfm-kaiser-15us-opti-isl}
\includegraphics[width=\imgsize]{figures/nlfm-kaiser-15us-opti-msl.png}
\caption{MSL optimization results for NLFM combined with Kaiser amplitude weighting}
\label{fig:nlfm-kaiser-15us-opti-msl}
\end{figure}


The pulse power ration in figure \ref{fig:nlfm-kaiser-15us-opti-msl} is the same for each waveform, because it only depends on 
the shape of the amplitude weighting function determined by the Kaiser parameter. As described previously, the power decreases rapidly
with more aggressive weighting.

The range resolution result also makes sense. Similar to the NLFM-only case in section \ref{s:nlfm-only-results}, the range resolution
decreases slightly with increasing NLFM parameter. The same happens for an increasing Kaiser parameter. Interestingly, the effect of both
seems to be compounded such that aggressive windowing and high NLFM parameter result in a very large decrease in range resolution.


\begin{figure}[ht]
\includegraphics[width=\imgsize]{figures/nlfm-kaiser-15us-opti-ppr.png}
\caption{PPR optimization results for NLFM combined with Kaiser amplitude weighting}
\label{fig:nlfm-kaiser-15us-opti-ppr}
\includegraphics[width=\imgsize]{figures/nlfm-kaiser-15us-opti-res.png}
\caption{Resolution optimization results for NLFM combined with Kaiser amplitude weighting}
\label{fig:nlfm-kaiser-15us-opti-res}
\end{figure}





\section{Effects of Impulse Response on the Ambiguity Function}

As the signal returns to the radar and passes through the receive chain, it will be distorted by the
impulse response of the filter and amplifier. Figure \ref{fig:pulsewoir-200us} shows a standard pulse
that is undistorted. 

\begin{figure}[!ht]
\includegraphics[width=\imgsize]{figures/pulsewoir-200us.png}
\caption{ Pulse without IR distortion}
\label{fig:pulsewoir-200us}
\end{figure}


On the other hand, figure \ref{fig:pulsewir-200us} shows the AF for a pulse that
has been distorted by the transfer function shown in figure \ref{fig:tf1}. 

\begin{figure}[!ht]
\includegraphics[width=\imgsize]{figures/pulsewir-200us.png}
\caption{ Pulse with IR distortion}
\label{fig:pulsewir-200us}
\end{figure}

\begin{figure}[!ht]
\includegraphics[width=\imgsize]{figures/tf1.png}
\caption{Transfer Function}
\label{fig:tf1}
\end{figure}

The distortion lengthened the signal and smoothed it out a bit.


