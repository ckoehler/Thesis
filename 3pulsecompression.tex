Having looked at general waveform design, several techniques of pulse compression are considered next. As previously
mentioned, the goal is to increase sensitivity without reducing the range resolution and thus control the sensitivity
and bandwidth separately.  To that end, different types of modulation are examined that allow the signal to be designed
to the desired specifications.


\section{Amplitude Modulation}
\label{s:ampmod}
One way to suppress range sidelobes is to taper the amplitude of the signal such that less energy is transmitted towards
the edges of the signal in order to reduce so-called \emph{spectral leakage} caused by working with finite sequences of
a signal \citep{Harris:1978}. In terms of bandwidth, the discontinuities of finite signals introduce sharp edges
that, similar to a perfectly rectangular pulse, increase the bandwidth needed to properly represent the signal. Since the
bandwidth of any system is limited, amplitude tapering can help condition the signal to match the bandwidth requirements
of the system.  However, tapering has at least two disadvantages.  First, less energy is illuminating the target because
the amplifier does not transmit as much power for an amplitude-tapered waveform as a rectangular pulse. This results in
a lower SNR, which has a negative impact on detection. Second, the mainlobe is usually broadened, resulting in a decrease in
spatial resolution.

In this work, the Kaiser window is focused upon because of its flexibility and the fact that its shape can be changed based
on only one parameter, $\alpha$.  Evaluating different waveforms tapered by the Kaiser window with different values of
$\alpha$ allows the amplitude tapering to be optimized for weather radar applications.  The Kaiser window is defined as:

\begin{equation}
  \label{eq:kaiser}
  w_n = \frac{I_0\left(\pi \alpha \sqrt{1- \left( \frac{n}{N/2} \right) ^2} \right)} { I_0(\pi \alpha)}, 0 \leq |n| \leq N/2,
\end{equation}
where $I_0$ is the zeroth order Modified Bessel function of the first kind, $N$ is the length of the sequence, and
$\alpha$ is the parameter that determines the shape of the window \citep{Harris:1978}.
For the sake of simplicity, the function argument is redefined as $\beta = \pi \alpha$, resulting in the function:

\begin{equation}
  \label{eq:kaiser2}
  w_n = \frac{I_0\left(\beta \sqrt{1- \left( \frac{n}{N/2} \right) ^2} \right)} { I_0(\beta)}, 0 \leq |n| \leq N/2,
\end{equation}
This work will use $\beta$ as the parameter describing the window shape.  Figure \ref{fig:kaiserparams} shows the Kaiser
Window function for varying values of the parameter $\beta$.

\begin{figure}[H]
\includegraphics[width=\imgsizes]{figures/kaiserparams.png}
\caption{For amplitude tapering, a Kaiser window is applied to the discussed waveforms.
  This figure shows the Kaiser window function for varying values of its parameter $\beta$.}
\label{fig:kaiserparams}
\includegraphics[width=\imgsizes]{figures/fftspec.png}
\caption{Frequency spectrum of a \SI{15}{\micro\second} pulse that has been tapered with various windows. The spectrum
of a rectangular pulse is given as a reference.}
\label{fig:fftspec}
\end{figure}


Figure \ref{fig:fftspec} shows the frequency spectrum of a rectangular pulse compared to pulses that have been tapered
with different Kaiser windows. The spectrum of the rectangular pulse is a sinc function with sidelobes extending into
infinity. Applying a window lowers the sidelobes and broadens the mainlobe. As mentioned previously, tapering can also
limit the bandwidth of the signal.

To make use of amplitude tapering, one important consideration is the choice of amplifier used in the system. Most
transmitters operate most efficiently in saturation \citep[p. 692]{Skolnik:2001}. For amplitude
tapering to work, the transmitter must operate linearly, which means it is not operating at peak efficiency. This loss
in power will result in decreased sensitivity, so the trade-offs of this method must be considered.



\section{Frequency Modulation}
Frequency modulation is generally divided into two areas, Linear Frequency
Modulation (LFM) and Non-Linear Frequency Modulation (NLFM). This section will
examine how both schemes work and compare them to amplitude modulation, discussed above.


\subsection{Linear Frequency Modulation}
\label{s:lfm}
For LFM, the frequency of the pulse is simply modulated linearly over a certain bandwidth $B$. This is called a chirp
and is defined by:

\begin{equation}
\label{eq:LFM}
s(t) = \exp(-j \pi k t^2 ) ,\ \ 0 \leq t \leq \tau,
\end{equation}
where $k= \pm \frac{B}{\tau}$ \citep[p.57]{Levanon:2004}. The real part of the waveform is shown in Figure
\ref{fig:lfm-waveform}.

\begin{figure}[h]
\includegraphics[width=\imgsize]{figures/lfm-waveform-5us.png}
\caption{Varying the frequency linearly over a certain bandwidth for the duration of the pulse is called Linear
Frequency Modulation. This plot shows the frequency increase over the pulse length $\tau=\SI{5}{\micro\second}$.}
\label{fig:lfm-waveform}
\end{figure}
A quadratic phase modulation is the equivalent of a linear frequency modulation, because the instantaneous frequency is
the derivative of the phase. Thus, for a phase $\phi = 2 \pi (f_\mathrm{0} t + k t^2 / 2)$, the instantaneous frequency
is \citep{Klauder:1960}:
\begin{equation}
  f_\mathrm{i} = \frac{1}{2 \pi} \frac{d \phi}{dt} = f_\mathrm{0} + k t,
\end{equation}
where $f_\mathrm{0}$ is the carrier frequency which, in these examples, is $0$.

The maximum theoretical range resolution of an LFM waveform is only dependent on the bandwidth $B$, not the pulse length $\tau$,
and is expressed as \citep[p. 196]{Richards:2005}:
\begin{equation}
  \label{eq:lfm-rangeres}
  \Delta r = \frac{c}{2 B}.
\end{equation}
If $B=\frac{1}{\tau}$, this result is consistent with a traditional pulsed radar, as shown in Equation
\eqref{eq:rangeresbwpulse}.

As seen in Figure \ref{fig:lfm-full}, the slow drop-off of the peak in the AF, also called a ridge, makes LFM pulse
compression very Doppler tolerant and thus convenient for radar applications that have to deal with targets whose
velocity is unknown. The mismatch in range caused by the shifted ridge is called \emph{range-Doppler coupling} and can
be corrected by alternating between an up- and a down-chirp and averaging the resulting range of both to get the true
range \citep[p. 343]{Skolnik:2001}. However, since the range mismatch is in range-time and changes the location of range
gates, the averaging process to get the true range will result in range gate misalignments for distributed targets and
introduce inaccuracies.

The Doppler tolerance of the LFM waveform is a consequence of a property of the AF. The AF of a waveform that has a linear
frequency shift applied to it, i.e., $s'(t) = s(t) \exp(-j \pi k t^2)$, is given as follows:

\begin{align*}
  |\chi(\tau_\mathrm{d},f_\mathrm{d})|
  &= \left| \int_{-\infty}^\infty{s(t) \exp(-j \pi k t^2) s^*(t+\tau_\mathrm{d}) \exp[j \pi k (t + \tau_\mathrm{d})^2]
  \exp(-j2\pi f_\mathrm{d} t) dt} \right| \\
  &= \left| \exp(j \pi k \tau_\mathrm{d}^2)\int_{-\infty}^\infty{s(t) s^*(t+\tau_\mathrm{d})
  \exp[j 2 \pi (  k \tau_\mathrm{d} - f_\mathrm{d}) t] dt} \right| \\
  &= \left| \exp(j \pi k \tau_\mathrm{d}^2) ~ \chi'(\tau_\mathrm{d}, k \tau_\mathrm{d} - f_\mathrm{d} ) \right| \\
  &= \left| \chi'(\tau_\mathrm{d}, k \tau_\mathrm{d} - f_\mathrm{d} ) \right|
\end{align*}
This shows that, for a signal modulated with a quadratic phase shift, the Doppler dimension changes based on the range
delay, causing the shifted ridge.

\begin{figure}[H]
\includegraphics[width=\imgsizes]{figures/lfm-full-15us.png}
\caption{AF for an up-chirp LFM with $\tau=\SI{15}{\micro\second}$ for a \SI{9550}{\mega\hertz} radar. LFM waveforms are very Doppler
tolerant as seen by the ridge. Notice the extreme radial velocities and the comparatively small range mismatch.}
\label{fig:lfm-full}
\includegraphics[width=\imgsizes]{figures/lfm-full-downchirp-15us.png}
\caption{AF for a down-chirp LFM with $\tau=\SI{15}{\micro\second}$ for a \SI{9550}{\mega\hertz} radar. Chirping down
instead of up changes the direction of the ridge and makes range corrections possible.}
\label{fig:lfm-full-down}
\end{figure}



\subsection{Non-linear Frequency Modulation}
\label{s:nlfm}
Equation \eqref{eq:LFM} is a special case that describes
linear frequency modulation. A more general expression for a frequency modulated waveform is:

\begin{equation}
\label{eq:NLFM}
s(t) =  \exp(-j \Phi(t)) = \exp(-j 2 \pi \int_0^t{f(x) dx}) ,\ \ 0 \leq t \leq \tau,
\end{equation}
where $f(x)$ defines the function of the frequency sweep \citep[p. 88]{Levanon:2004}. 
In the LFM case, $f(x)=kx$, which simplifies Equation \eqref{eq:NLFM} to:

\begin{equation}
\label{eq:NLFM_simp}
s(t) = \exp(j 2 \pi \int_0^t{f(x) dx}) =  \exp(j 2 \pi \frac{k t^2}{2}) = \exp(j \pi k t^2),
\end{equation}
which indicates that a quadratic phase modulation corresponds to a linear frequency change.  

However, $f(x)$ can be any function that describes the sweep pattern within the frequency band.  In this work, various
sweep patterns are investigated and their performance evaluated. The results are presented in Chapter
\ref{chap:optimization}.


\subsection{Early Use of Frequency Modulation} 
Since LFM is simply taking advantage of a property of the AF, as described in Section \ref{s:lfm}, it was one the
earliest ways of improving sensitivity \citep[p. 57]{Levanon:2004} (see also \cite{Klauder:1960}).  It is simple to implement in analog hardware by
using a voltage controlled oscillator (VCO) and is Doppler tolerant due to the slow decrease in slope, making it a
popular choice for early radars that had to detect fast moving targets. With more processing being done in digital
systems, other forms of pulse compression are becoming increasingly more common because they are easier to implement in
digital systems.


\section{Phase Modulation}
Similar to amplitude modulation and frequency modulation, waveforms can be modulated in phase as well. A pulse $s(t)$ of
length $\tau$ is divided into $M$ different parts, and each part is assigned a different phase \citep[p. 100]{Levanon:2004}.
The set of phases associated with the different parts of the pulse is called the \emph{phase code} of
$s(t)$. There is an unlimited number of phase codes and much work has gone into finding codes that improve the
performance of waveform. Only a few different types will be discussed here.  The reader is encouraged to refer to
\cite{Levanon:2004} for a much more comprehensive treatment of phase codes.


\subsection{Binary Codes}
Binary codes vary the phase of the signal in increments of $\pi$.  The most commonly known phase code is the Barker code
\citep[pp. 273-287]{Barker:1953}, which is believed to exist up to a length of 13 \citep[p. 106]{Levanon:2004}.  The
Barker 13 code looks like +1 +1 +1 +1 +1 -1 -1 +1 +1 -1 +1 -1 +1 . It is called a ``perfect" code because its ACF has a
main lobe of height 13 and sidelobes with heights of no more than 1, as depicted in Figure \ref{fig:barker13-acf}.

\begin{figure}[h]
\includegraphics[width=\imgsize]{figures/barker13-acf.png}
\caption{ACF of the Barker code with length 13. Barker codes are ``perfect'', meaning they have minimal
sidelobes.}
\label{fig:barker13-acf}
\end{figure}

\subsection{$n$-ary Codes}
\label{s:narycodes}
Instead of varying the phase of the signal between only 0 and $\pi$, they can be varied arbitrarily for any value
between 0 and $2\pi$ in an attempt to gain a better performing code. These types of codes are also known as
\emph{polyphase codes}. Examples, including a table of polyphase Barker codes up to length 45, are given in \cite[pp.
110-112]{Levanon:2004}.

Other polyphase codes include Frank codes and their derivatives, like P1, P2, and Px codes \citep[pp. 113-122]{Levanon:2004}, which
are derived from sampling the phase of an LFM waveform and thus called \emph{chirplike}.


\subsection{Bandwidth for Phase Codes}
The bandwidth of a phase-modulated waveform is approximately the inverse of the length of a subpulse \citep[p.
350]{Skolnik:2001}. Given a pulse with length $\tau$ and $M$ subpulses, the length of a subpulse is $T =
\frac{\tau}{M}$ and the bandwidth of the whole pulse will be approximately $B = \frac{1}{T}$ Hz. This is a great example about how
pulse compression allows the decoupling of pulse length and signal bandwidth.

