
\section{Frequency Modulation}
\subsection{Linear Frequency Modulation}
\label{s_lfm}
In Pulse Compression, the frequency inside the pulse is modulated linearly (for Linear FM, or LFM) or
non-linearly over a certain bandwidth. It is defined as:
\begin{equation}
\label{LFM}
x(t) = e^{j \pi \beta t^2 } = e^{j \theta (t)},\ \ 0 \leq t \leq \tau
\end{equation}

The spectrum of such a LFM waveform approaches the shape of a rectangle that becomes better defined
as the $\beta\tau$ ratio increases. The matched filter response of a rectangular waveform is a sinc 
function with Rayleigh resolution of approximately $1/\beta$ (\cite{Richards:2005}).


The result is an increase in the ambiguous range $r_a$.

The LFM effect is a consequence of a Ambiguity Function property, as follow:

If $u(t)$ has an added phase that changes quadratically, i.e. $u^{'}(t) = u(t) e^{j\pi \beta t^2}$, which 
is equivalent to a linear frequency modulation because $\omega(t) = \frac{d}{dt}\phi(t)$, the resulting 
AF is $|\chi(\tau,f_d-k\tau)|$.


TODO: figures


The slow drop-off of peak in the AF makes LFM pulse compression very Doppler tolerant, which make it convenient 
for radar applications that have to deal with high-velocity targets. Add to that the ease of implementation in 
analog hardware and we have a simple way to improve our radar's performance.

\subsection{Non-linear Frequency Modulation}
Extrapolating from the results above, the general expression for a frequency modulated waveform is as follows.

\begin{equation}
\label{NLFM}
x(t) = e^{j 2 \pi \int_0^t{f(t') dt'}} ,\ \ 0 \leq t \leq \tau,
\end{equation}

where $f(t)$ defines the function of the frequency sweep. In the LFM case, $f(t)=\beta t$, which simplifies Eq
\ref{NLFM} to

\begin{equation}
\label{NLFM_simp}
x(t) = e^{j 2 \pi \int_0^t{f(t') dt'}} =  e^{j 2 \pi \frac{\beta t^2}{2}} = e^{j \pi \beta t^2},
\end{equation}

which validates our initial claim that a quadratic phase modulation corresponds to a linear frequency change.

\subsection{Military Radars' use of Frequency Modulation} 
Since LFM is simply taking advantage of a property of the AF, it was the earliest way of getting increased sensitivity.
It's fairly simple to implement in analog hardware and very Doppler tolerant due to the slow decrease in slope.


\section{Amplitude Modulation}

\section{Phase Modulation}
\subsection{Binary Codes}
One alternative to \ref{s_lfm} is the use of binary, or in our case, quadrature codes. With it, a subsection of 
each pulse is shifted in phase by $e^{jk \pi}$, where $k=(0, \frac{1}{2}, 1, \frac{3}{2})$. 
\subsubsection{Binary Phase Codes compared to Frequency Modulation}

\subsubsection{Bandwidth for Binary Phase Codes}
The bandwidth of a binary phase-modulated waveform is the inverse of the length of a subpulse. Given a
pulse with length $\tau$ and $n$ subpulses, the length of a subpulse is $T = \frac{\tau}{n}$, the the bandwidth
of the whole pulse will be $B = \frac{1}{T}$ Hz.

\subsubsection{Use of Phase Codes in Profilers}
why are they using it?


\subsection{Amplitude Tapering compared to Non-linear Frequency Modulation}
