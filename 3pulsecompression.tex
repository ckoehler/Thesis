\section{Amiguity Function}
The main tool to determine waveform performance is the Ambiguity Function (AF) (\cite{Levanon:2004}). It is defined as:

\begin{equation}
\label{eq:AF}
  |\chi(\tau,f_d)| = | \int_{-\infty}^\infty{u(t) u^*(t+\tau) e^{j2\pi f_d t} dt}|,
\end{equation}

where $u(t)$ is the complex envelope of the signal, $\tau$ is the delay on the delay axis, and $f_d$
is the Doppler frequency shift in the Doppler axis.
We can clearly see that, for $f_d = 0$, the AF reduces to the Autocorrelation Function (ACF).


For a simple rectangular pulse of the form $u(t)=\frac{1}{\sqrt{T}} rect(\frac{t}{T})$, the AF is:
\begin{align*}
  |\chi(\tau,f_d)| &= | \int_{-\infty}^\infty{u(t) u^*(t+\tau) e^{j2\pi f_d t} dt}| \\ 
  &= 
  \begin{cases}
    \frac{1}{T} \int_{-\frac{T}{2} + \tau}^{\frac{T}{2}}{e^{j2\pi f_d t}dt} & \text{for $0 \leq \tau \leq T$}, \\
    \frac{1}{T} \int_{-\frac{T}{2}}^{\frac{T}{2} + \tau}{e^{j2\pi f_d t}dt} & \text{for $-T \leq \tau < 0$}, \\
    0 & \text{otherwise}. \\
  \end{cases}
\end{align*}

Evaluate the integral:

\begin{align*}
  \frac{1}{T} \int_{-\frac{T}{2} + \tau}^{\frac{T}{2}}{e^{j2\pi f_d t}dt}
  &= foo \\ 
  &= bar 
\end{align*}

\section{Effects of amplifier impulse response on the Ambiguity function}

Next we will examine the effects of the amplifier in the receive chain on the pulse.
Recall the general AF (\ref{eq:AF}) and let the return signal part of it pass through the system
with the impulse response $h(t)$:

\begin{align*}
   \chi(t,f_d) &= |\int_{-\infty}^\infty{u(t) [(u^*(t+\tau) e^{j2\pi f_d t})*h(t)]\ dt}| \\
   &= |\int_{-\infty}^\infty{u(t) \int_{-\infty}^{\infty}{u^*(t'+\tau) e^{j2\pi f_d t'}h^*(t'-t) dt'}\ dt}| \\
   \iff \chi(t',f_d) &= |\int_{-\infty}^\infty{u^*(t'+\tau) e^{j2\pi f_d t'}\int_{-\infty}^{\infty}{u(t) h^*(t'-t) dt}\ dt'}| 
\end{align*}

We can easily see that applying the impulse response to the return signal is equivalent to applying it to the original pulse $u(t)$. 

\section{Waveform optimizations}
\subsection{Integrated Sidelobe Level}
The performance of each code is measured by calculating the Integrated Sidelobe Level (ISL), which is defined as (\cite{Keeler:1999}):

\begin{equation}
\label{ISL}
ISL = 10 \log \left(\frac{\sum\limits_{j} |m_k|^2}{\sum\limits_{k} |s_j|^2}\right)
\end{equation}

where $s_j$ is the power of the $j^{th}$ sidelobe, and $m_k$ the power of the $k^{th}$ mainlobe.

The higher the ISL, the better the performance of the code.

\subsection{3dB Beamwidth}
Another metric to evaluate the performance of a waveform is the 3dB waveform in units of meters, i.e. the spatial delay.
The lower the value, the finer the resolution of the radar.


\subsection{Maximum Sidelobe Level}
Although the ISL is a good way to quantify waveform performance, it can be misleading. Sidelobes are very undesirable, so a waveform
that has overall low sidelobes except for one very dominant one will have a low ISL, but may not be performant enough for actual use due
to the outlier sidelobe.
