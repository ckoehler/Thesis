
\section{Amplitude Modulation}
One way to bring down undesired sidelobes is to weight the amplitude of the signal such that less energy is transmitted
towards the edges of the signal. However, this has a couple of downsides. First, not as much energy is transmitted onto the
target because the amplifier doesn't put out as much power for a weighted amplitude waveform as it would for a 
rectangular window. This results in a lower SNR and can impact rate of detection. Second, the mainlobe is broadened,
resulting in a decrease in spatial resolution.


\section{Frequency Modulation}
\subsection{Linear Frequency Modulation}
\label{s_lfm}
In Pulse Compression, the frequency inside the pulse is modulated linearly (for Linear FM, or LFM) or
non-linearly over a certain bandwidth. It is defined as:
\begin{equation}
\label{LFM}
x(t) = e^{j \pi \beta t^2 } = e^{j \theta (t)},\ \ 0 \leq t \leq \tau
\end{equation}

The spectrum of such a LFM waveform approaches the shape of a rectangle that becomes better defined
as the $\beta\tau$ ratio increases. The matched filter response of a rectangular waveform is a sinc 
function with Rayleigh resolution of approximately $1/\beta$ (\cite{Richards:2005}, p. 191).


The result is an increase in the ambiguous range $r_a$.

The LFM effect is a consequence of a Ambiguity Function property, as follow:

If $u(t)$ has an added phase that changes quadratically, i.e. $u^{'}(t) = u(t) e^{j\pi \beta t^2}$, which 
is equivalent to a linear frequency modulation because $\omega(t) = \frac{1}{2 \pi}\frac{d}{dt}\phi(t)$, the resulting 
AF is $|\chi(\tau,f_d-k\tau)|$.


TODO: figures


The slow drop-off of peak in the AF makes LFM pulse compression very Doppler tolerant, which make it convenient 
for radar applications that have to deal with targets whose velocity is unknown. Add to that the ease of implementation in 
analog hardware and we have a simple way to improve our radar's performance.

\subsection{Amplitude Weighting compared to Non-linear Frequency Modulation}
It seems like tapering the amplitude of a signal to reduce sidelobes delivers the same result as simply sweeping the frequency
in a non-linear manner. The reason for this is easily explained. The sidelobes appear in the ACF, which is the inverse
Fourier transform of the power spectrum. Thus modifying the power spectrum, e.g. by amplitude weighting, will change the 
shape of the ACF.

Alternatively, the spectrum can be shaped by varying the frequency sweep in a non-linear fashion and spend more time at frequencies
that should be enhanced (\cite{Levanon:2004}, 87). This is called Non-Linear Frequency Modulation, or NLFM.

\subsection{Non-linear Frequency Modulation}
Extrapolating from the results above, the general expression for a frequency modulated waveform is as follows.

\begin{equation}
\label{NLFM}
x(t) = e^{j 2 \pi \int_0^t{f(t') dt'}} ,\ \ 0 \leq t \leq \tau,
\end{equation}

where $f(t)$ defines the function of the frequency sweep. In the LFM case, $f(t)=\beta t$, which simplifies Eq
\ref{NLFM} to

\begin{equation}
\label{NLFM_simp}
x(t) = e^{j 2 \pi \int_0^t{f(t') dt'}} =  e^{j 2 \pi \frac{\beta t^2}{2}} = e^{j \pi \beta t^2},
\end{equation}

which validates our initial claim that a quadratic phase modulation corresponds to a linear frequency change.
In this case, $f(t)$ can be any desired function that describes the way the frequency band should be sweeped.
We will later experiment with different functions and examine the way it affects the ambiguity function.

\subsection{Early Use of Frequency Modulation} 
Since LFM is simply taking advantage of a property of the AF, it was one the earliest ways of getting increased sensitivity.
It's fairly simple to implement in analog hardware and very Doppler tolerant due to the slow decrease in slope, making it
a popular choice for early radars that had to detect fast moving targets. With the move of a lot of components into the digital
domain, other forms of pulse compression are becoming increasingly more common because they are easier to implement in digital
systems.


\section{Phase Modulation}
\subsection{Binary Codes}
Binary codes vary the phase of the signal in increments of $\pi$. 
The most commonly known phase code is a Barker code, which is believed to exist up to a length of 13. The Barker 13 code
looks like +1 +1 +1 +1 +1 −1 −1 +1 +1 −1 +1 −1 +1. It's called a "perfect" code because its autocorrelation has a main lobe 
of 13 and sidelobes of no more than 1.

\subsection{n-ary Codes}
Instead of varying the phase of the signal between only 0 and $\pi$, they can be varied arbitrarily between 0 and $2\pi$ in
an attempt to gain a better performing code.


\subsubsection{Binary Phase Codes compared to Frequency Modulation}

\subsubsection{Bandwidth for Binary Phase Codes}
The bandwidth of a binary phase-modulated waveform is the inverse of the length of a subpulse. Given a
pulse with length $\tau$ and $n$ subpulses, the length of a subpulse is $T = \frac{\tau}{n}$, the bandwidth
of the whole pulse will be $B = \frac{1}{T}$ Hz.

\subsubsection{Use of Phase Codes in Profilers}
why are they using it?


