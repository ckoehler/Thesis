Having looked at general waveform design, several techniques of pulse compression are considered next. As previously
mentioned, the goal is to increase sensitivity without reducing the range resolution and thus control the sensitivity
and bandwidth separately.  To that end, different types of modulation are examined that allow the signal to be designed
to the desired specifications.


\section{Amplitude Modulation}
\label{s:ampmod}
One way to suppress the sidelobes is to taper the amplitude of the signal such that less energy is transmitted towards
the edges of the signal in order to reduce so-called \emph{spectral leakage} caused by working with finite sequences of
signal data \citep{Harris:1978}. In terms of bandwidth, discontinuities introduce sharp edges that, similar to a
perfectly rectangular pulse, increase the bandwidth needed to properly process the signal. Since the bandwidth of any
system is limited, amplitude tapering can help condition the signal to match the bandwidth requirements of the system. However, this has at least
two disadvantages.  First, less energy is illuminating the target because the amplifier does not transmit as much power
for an amplitude-tapered waveform as a rectangular pulse. This results in a lower SNR, which has a negative
impact on detection. Second, the mainlobe is broadened, resulting in a decrease in spatial resolution.

In this work, the Kaiser window is chosen because its shape can be changed based on only one parameter $\alpha$.
Evaluating different waveforms tapered by the Kaiser window with different values of $\alpha$ allows the amplitude
tapering to be optimized for weather application.  The Kaiser window is defined as

\begin{equation}
  \label{eq:kaiser}
  w_n = \frac{I_0\left(\pi \alpha \sqrt{1- \left( \frac{n}{N/2} \right) ^2} \right)} { I_0(\pi \alpha)}, 0 \leq |n| \leq N/2,
\end{equation}
where $I_0$ is the zeroth order Modified Bessel function of the first kind, $N$ is the length of the sequence, and
$\alpha$ is the parameter that determines the shape of the window \citep{Harris:1978}.
For the sake of simplicity, the parameter is redefined as $\beta = \pi \alpha$, resulting in the function

\begin{equation}
  \label{eq:kaiser2}
  w_n = \frac{I_0\left(\beta \sqrt{1- \left( \frac{n}{N/2} \right) ^2} \right)} { I_0(\beta)}, 0 \leq |n| \leq N/2,
\end{equation}
This work will use $\beta$ as the parameter describing the window shape.  Figure \ref{fig:kaiserparams} shows the Kaiser
Window function for varying values of the parameter $\beta$.

\begin{figure}[H]
\includegraphics[width=\imgsize]{figures/kaiserparams.png}
\caption{For amplitude tapering, a Kaiser window is applied to the discussed waveforms.
  This figure shows the Kaiser window function for varying values of its parameter $\beta$.}
\label{fig:kaiserparams}
\includegraphics[width=\imgsizes]{figures/fftspec.png}
\caption{Frequency spectrum of a \SI{15}{\micro\second} pulse that has been tapered with various windows. The spectrum
of a rectangular pulse is given as a reference.}
\label{fig:fftspec}
\end{figure}


Figure \ref{fig:fftspec} shows the frequency spectrum of a rectangular pulse compared to pulses that have been tapered
with different Kaiser windows. The spectrum of the rectangular pulse is a sinc function with sidelobes extending into
infinity. Applying a window lowers the sidelobes and, with sufficiently aggressive tapering, removes them completely. It
also broadens the mainlobe. As mentioned previously, tapering can also limit the bandwidth of the signal.

To make use of amplitude tapering, one important consideration is the choice of amplifier used in the system. Most
transmitters operate most efficiently in saturation, i.e., completely on \citep[p. 692]{Skolnik:2001}. For amplitude
tapering to work, the transmitter must operate linearly, which means it is not operating at peak efficiency. This loss
in power will result in decreased sensitivity, so again the trade-offs of this method must be considered.



\section{Frequency Modulation}
Frequency modulation is generally divided into two areas, Linear Frequency
Modulation (LFM) and Non-Linear Frequency Modulation (NLFM). This section will
examine how both schemes work by comparing them to amplitude modulation, discussed above.


\subsection{Linear Frequency Modulation}
\label{s:lfm}
For LFM, the frequency of the pulse is simply modulated linearly
over a certain bandwidth $B$. This called a chirp and defined by:

\begin{equation}
\label{eq:LFM}
x(t) = e^{-j \pi k t^2 } ,\ \ 0 \leq t \leq \tau,
\end{equation}
where $k= \pm \frac{B}{\tau}$ \citep[p.57]{Levanon:2004}. The real part of the waveform is shown in Figure
\ref{fig:lfm-waveform}.

\begin{figure}[h]
\includegraphics[width=\imgsize]{figures/lfm-waveform-5us.png}
\caption{Varying the frequency linearly over a certain bandwidth for the duration of the pulse is called Linear
Frequency Modulation and results in a waveform like this.}
\label{fig:lfm-waveform}
\end{figure}

The LFM effect is a consequence of a property of the AF.
This property states that given a signal $u(t)$, if $u(t)$ has an added phase that changes quadratically,
i.e., $u'(t) = u(t) e^{-j\pi k t^2}$, caused by a certain Doppler shift of frequency $f_d$, this phase change is
equivalent to a linear frequency modulation because the instantaneous frequency $\omega(t)$ is related to the derivative of the
instantaneous phase $\phi(t)$ like  $\omega(t) = \frac{1}{2 \pi}\frac{d}{dt}\phi(t)$. The AF of $u'(t)$
is $|\chi(\tau,f_d-k\tau)|$ \citep[p. 38]{Levanon:2004}.


\begin{figure}[H]
\includegraphics[width=\imgsize]{figures/lfm-full-15us.png}
\caption{LFM waveforms are very Doppler tolerant as seen by the ridge. Notice the large radial velocities and the
comparatively small range mismatch in the AF.}
\label{fig:lfm-full}
\end{figure}


As seen in Figure \ref{fig:lfm-full}, the slow drop-off of the peak in the AF, also called a ridge, makes LFM pulse
compression very Doppler tolerant and thus convenient for radar applications that have to deal with targets whose
velocity is unknown. The mismatch in range caused by the shifted ridge is called \emph{range-doppler coupling} and can
be corrected by alternating between an up and a down-chirp and averaging the resulting range of both to get the true
range \citep[p. 343]{Skolnik:2001}.

The maximum range resolution of a LFM waveform is now only dependent on the bandwidth $B$, not the pulse length $\tau$,
which is expressed as \citep[p. 196]{Richards:2005}:
\begin{equation}
  \label{eq:lfm-rangeres}
  \Delta r = \frac{c}{2 B}.
\end{equation}
This result will be confirmed in the simulation later, where it is shown that the pulse length does not change the shape
of the mainlobe in the AF.

\subsection{Non-linear Frequency Modulation}
\label{s:nlfm}
Equation \eqref{eq:LFM} is a special case that describes
linear frequency modulation. A more general expression for a frequency modulated waveform is

\begin{equation}
\label{eq:NLFM}
x(t) =  e^{-j \Phi(t)} = e^{-j 2 \pi \int_0^t{f(x) dx}} ,\ \ 0 \leq t \leq \tau,
\end{equation}
where $f(x)$ defines the function of the frequency sweep \citep[p. 88]{Levanon:2004}. 

In the LFM case, $f(x)=kx$, which simplifies Equation \eqref{eq:NLFM} to:

\begin{equation}
\label{eq:NLFM_simp}
x(t) = e^{j 2 \pi \int_0^t{f(x) dx}} =  e^{j 2 \pi \frac{k t^2}{2}} = e^{j \pi k t^2},
\end{equation}
which indicates that a quadratic phase modulation corresponds to a linear frequency change.  

However, $f(x)$ can be any function that describes the sweep pattern within the frequency band.  In this work, various
sweep patterns are investigated and their performance evaluated. The results are presented in Chapter
\ref{chap:optimization}.

\subsection{Amplitude Tapering Compared to Frequency Modulation}
Tapering the amplitude of a signal to reduce sidelobes
delivers the same result as simply sweeping the frequency non-linearly. This relationship can be explained by
considering the ACF of the signal. The sidelobes appear in the ACF, which is the inverse Fourier transform of the power
spectrum. Thus modifying the power spectrum, e.g., by amplitude tapering, will likewise change the shape of the ACF.

From this follows that alternatively, the spectrum can be shaped by varying the frequency sweep non-linearly 
to spend more time at frequencies that should be enhanced \citep[p. 87]{Levanon:2004}. This is the concept of 
NLFM and will be considered in the next section.


\begin{figure}[h]
\includegraphics[width=\imgsize]{figures/nlfmbwspectrum.png}
\caption{Frequency spectrum comparison of NLFM and amplitude tapering. It shows that applying a window to the amplitude
has a similar effect to chirping the frequency of the pulse non-linearly.}
\label{fig:nlfmbwspectrum}
\end{figure}


Figure \ref{fig:nlfmbwspectrum} shows the results of a NLFM pulse and a tapered LFM pulse. An untapered LFM pulse
is given for comparison. Both NLFM and tapering changes the shape of the mainlobe and sidelobes. Optimizing the shape of
the NLFM is one of the challenges of waveform design.


\subsection{Early Use of Frequency Modulation} 
Since LFM is simply taking advantage of a property of the AF, as described in Section \ref{s:lfm}, it was one the
earliest ways of improving sensitivity \citep[p. 57]{Levanon:2004}.  It is simple to implement in analog hardware
by using a voltage controlled oscillator (VCO) and very Doppler tolerant due to the slow decrease in slope, making it a
popular choice for early radars that had to detect fast moving targets. With the move of a lot of components into the
digital domain, other forms of pulse compression are becoming increasingly more common because they are easier to
implement in digital systems.


\section{Phase Modulation}
Similar to amplitude modulation and frequency modulation, waveforms can be modulated in phase as well. A pulse $u(t)$ of
length $\tau$ is divided up into $M$ different parts, and each part is assigned a different phase \citep[ch.
6]{Levanon:2004}. The set of phases associated with the different parts of the pulse is called the \emph{phase code} of
$u(t)$. There is an unlimited amount of phase codes and much work has gone into finding codes that improve the
performance of waveform. Only a few different types will be discussed here.  The reader is encouraged to refer to
\cite{Levanon:2004} for a much more comprehensive treatment of phase codes.


\subsection{Binary Codes}
Binary codes vary the phase of the signal in increments of $\pi$.  The most commonly known phase code is the Barker
code, which is believed to exist up to a length of 13 \citep[p. 106]{Levanon:2004}.  The Barker 13 code looks like +1 +1
+1 +1 +1 -1 -1 +1 +1 -1 +1 -1 +1. It is called a ``perfect" code because its ACF has a main lobe of 13 and
sidelobes of no more than 1, as depicted in Figure \ref{fig:barker13-acf}.

\begin{figure}[h]
\includegraphics[width=\imgsize]{figures/barker13-acf.png}
\caption{ACF of the Barker code with length 13. Barker codes are ``perfect'', meaning they have minimal
sidelobes.}
\label{fig:barker13-acf}
\end{figure}

\subsection{$n$-ary Codes}
\label{s:narycodes}
Instead of varying the phase of the signal between only 0 and $\pi$, they can be varied arbitrarily
between 0 and $2\pi$ in an attempt to gain a better performing code. These types of codes are also known as
\emph{polyphase codes}. Examples, including a table of polyphase Barker codes up to length 45, are given in \cite[p.
109ff]{Levanon:2004}.

Other polyphase codes include Frank codes and their derivatives, like P1, P2, and Px codes \citep{Levanon:2004}, which
are derived from sampling the phase of a LFM waveform and thus called \emph{chirplike}.


\subsection{Bandwidth for Phase Codes}
The bandwidth of a phase-modulated waveform is approximately the inverse of the length of a subpulse \citep[p.
350]{Skolnik:2001}. Given a pulse with length $\tau$ and $n$ subpulses, the length of a subpulse is $T =
\frac{\tau}{n}$, the bandwidth of the whole pulse will be about $B = \frac{1}{T}$ Hz. This is a great example about how
pulse compression allows the decoupling of pulse length and signal bandwidth.

