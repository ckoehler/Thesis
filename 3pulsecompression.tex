Having looked at general waveform design, several techniques of pulse compression are considered next. As previously mentioned, 
the goal is to increase sensitivity without increasing the pulse length and thus control the bandwidth and length separately.
To that end, different types of modulation are examined that allow the signal to be designed to the desired specifications.


\section{Amplitude Modulation}
\label{s:ampmod}
One way to suppress the sidelobes is to weight the amplitude of the signal such that less energy is transmitted
towards the edges of the signal \citep{signalbook}. However, this has a couple of downsides. First, less energy is illuminating the
target because the amplifier does not transmit as much power for an amplitude-weighted waveform compared to a 
rectangular pulse. This results in a lower SNR, which has a negative impact on detection. Second, the mainlobe is broadened,
resulting in a decrease in spatial resolution.

TODO: Hardware considerations, amplifier reqs, sensitivity? trade offs?


\section{Frequency Modulation}
Frequency modulation is generally divided into two areas. There is Linear Frequency Modulation (LFM) and everything else, 
commonly referred to as Non-Linear Frequency Modulation (NLFM). This section will examine how both work and compare to 
amplitude modulation, discussed previously in Section \ref{s:ampmod}.


\subsection{Linear Frequency Modulation}
\label{s:lfm}
For LFM, the frequency of the pulse is simply modulated linearly over a certain bandwidth $B$.
It is defined as:

\begin{equation}
\label{eq:LFM}
x(t) = e^{-j \pi \beta t^2 } ,\ \ 0 \leq t \leq \tau
\end{equation}

The LFM effect is a consequence of a property of the ambiguity function, which will be more closely examined in
Section \ref{s:af}. This property states that given a signal $u(t)$, 
if $u(t)$ has an added phase that changes quadratically, i.e. $u^{'}(t) = u(t) e^{j\pi \beta t^2}$, caused by a
certain Doppler shift of frequency $f_d$, this phase change 
is equivalent to a linear frequency modulation because $\omega(t) = \frac{1}{2 \pi}\frac{d}{dt}\phi(t)$. The resulting 
AF is $|\chi(\tau,f_d-k\tau)|$ \citep[ch. 3]{Levanon:2004}.


TODO: add figure of LFM AF.

The slow drop-off of peak in the AF makes LFM pulse compression very Doppler tolerant, which make it convenient
for radar applications that have to deal with targets whose velocity is unknown. The ease of implementation in 
analog hardware results in a simple way to improve a radar's performance.

The maximum range resolution of a LFM waveform is now only dependent on the bandwidth $B$, not the pulse length $\tau$. 
It is expressed as
\begin{equation}
  \label{eq:lfm-rangeres}
  \Delta r_{max} = \frac{c}{2 B}
\end{equation}

This result will be confirmed in the simulation letter, where it is shown that the pulse length does not change the
shape of the zero-Doppler cut of the ambiguity function.

\subsection{Amplitude Weighting compared to Non-linear Frequency Modulation}
Tapering the amplitude of a signal to reduce sidelobes delivers the same result as simply sweeping the frequency
in a non-linear manner. This relationship can be explained by considering the autocorrelation function (ACF) of
the signal. The sidelobes appear in the ACF, which is the inverse
Fourier transform of the power spectrum. Thus modifying the power spectrum, e.g. by amplitude weighting, will likewise 
change the shape of the ACF.

Alternatively, the spectrum can be shaped by varying the frequency sweep in a non-linear fashion to spend more time at frequencies
that should be enhanced \citep[p. 87]{Levanon:2004}. This is called Non-Linear Frequency Modulation, or NLFM, which will be considered
next.

\subsection{Non-linear Frequency Modulation}
Equation \eqref{eq:LFM} is a special case for linear frequency modulation. A more general expression for a frequency modulated 
waveform is

\begin{equation}
\label{eq:NLFM}
x(t) =  e^{-j \Phi(t)} = e^{-j 2 \pi \int_0^t{f(x) dx}} ,\ \ 0 \leq t \leq \tau,
\end{equation}

where $f(x)$ defines the function of the frequency sweep \citep[p. 88]{Levanon:2004}. 

In the LFM case, $f(x)=\beta x$, which simplifies Equation \eqref{eq:NLFM} to

\begin{equation}
\label{eq:NLFM_simp}
x(t) = e^{j 2 \pi \int_0^t{f(x) dx}} =  e^{j 2 \pi \frac{\beta t^2}{2}} = e^{j \pi \beta t^2},
\end{equation}

which indidates that a quadratic phase modulation corresponds to a linear frequency change.
In this case, $f(t)$ can be any desired function that describes the sweep pattern within the frequency band.
In this work, various sweep patterns have been investigated and the results will be presented in Chapter \ref{chap:optimization}.

\subsection{Early Use of Frequency Modulation} 
Since LFM is simply taking advantage of a property of the AF, as described in Section \ref{s:lfm}, it was one the earliest
ways of improving sensitivity.
It is fairly simple to implement in analog hardware and very Doppler tolerant due to the slow decrease in slope, making it
a popular choice for early radars that had to detect fast moving targets. With the move of a lot of components into the digital
domain, other forms of pulse compression are becoming increasingly more common because they are easier to implement in digital
systems.


\section{Phase Modulation}
Similar to amplitude modulation and frequency modulation, waveforms can be modulated in phase as well. A pulse $u(t)$ of length $\tau$ is
divided up into $M$ different parts, and each part is assigned a different phase \citep[ch. 6]{Levanon:2004}. The set of phases 
associated with the different parts of the pulse is called the \emph{phase code} of $u(t)$. There is an unlimited amount of phase codes
and much work has gone into finding codes that improve the performance of waveform. Only a few different types will be discussed here.
The reader is encouraged to refer to \cite{Levanon:2004} for a much more comprehensive treatment of phase codes.


\subsection{Binary Codes}
Binary codes vary the phase of the signal in increments of $\pi$. 
The most commonly known phase code is a Barker code, which is believed to exist up to a length of 13 \citep[p. 106]{Levanon:2004}.
The Barker 13 code looks like +1 +1 +1 +1 +1 -1 -1 +1 +1 -1 +1 -1 +1. It's called a ``perfect" code because its autocorrelation has a main lobe 
of 13 and sidelobes of no more than 1, as depicted in Figure \ref{fig:barker13-acf}.

\begin{figure}[h]
\includegraphics[width=\imgsize]{figures/barker13-acf.png}
\caption{Autocorrelation function of the Barker code with length 13.}
\label{fig:barker13-acf}
\end{figure}

\subsection{$n$-ary Codes}
\label{s:narycodes}
Instead of varying the phase of the signal between only 0 and $\pi$, they can be varied arbitrarily between 0 and $2\pi$ in
an attempt to gain a better performing code. These types of codes are also known as \emph{polyphase codes}. Examples, including
a table of polyphase Barker codes up to length 45, are given in \cite[p. 109ff]{Levanon:2004}.

Other polyphase codes include Frank codes and their derivatives, like P1, P2, and Px codes \citep{Levanon:2004}, which are derived
from sampling the phase of a LFM waveform and thus called \emph{chirplike}.


\subsection{Bandwidth for Phase Codes}
The bandwidth of a phase-modulated waveform is the inverse of the length of a subpulse. Given a
pulse with length $\tau$ and $n$ subpulses, the length of a subpulse is $T = \frac{\tau}{n}$, the bandwidth
of the whole pulse will be $B = \frac{1}{T}$ Hz. This is the important property at the heart of pulse compression
that allows the decoupling of pulse length and signal bandwidth.

