This chapter examines the different approaches to quantifying waveform performance in the context of weather radar
applications, as well as associated limitations.

\section{Quantifications}
\label{s:quantifications}
In order to evaluate the pulse compression techniques examined in this work, several different metrics are used to
quantify their performance. Together they can assist in creating and evaluating waveforms that perform optimally for
weather radars.

\subsection{3 dB Range Resolution}
The \SI{3}{\decibel} range resolution, given in units of meters, is the range resolution of the mainlobe
\SI{3}{\decibel} down from the peak.  Lower value means finer resolution of the waveform. The factors that influence
range resolution are mainly pulse length and bandwidth, as described in Equations \eqref{eq:rangeres} and
\eqref{eq:rangeresbwpulse}. For pulse length, the shorter the pulse, the better the resolution; for bandwidth, the
higher the bandwidth the better the resolution. Amplitude tapering or NLFM can also change the resolution because they
can affect the bandwidth of the waveform.


\subsection{Integrated Sidelobe Level}
The performance of each code is measured by calculating the Integrated Sidelobe Level (ISL), which is defined as
\citep{Keeler:1999}:

\begin{equation}
\label{ISL}
\text{ISL} = 10 \log \left(\frac{\sum\limits_{j} |s_j|^2}{\sum\limits_{k} |m_k|^2}\right)
\end{equation}
where $s_j$ is the amplitude of the $j^{th}$ sidelobe data point, and $m_k$ the amplitude of the $k^\mathrm{th}$
mainlobe data point. The ISL is thus the ratio between the area under the sidelobes and the area under the mainlobe, as
depicted in Figure \ref{fig:barker13-isl}.

\begin{figure}[H]
\includegraphics[width=\imgsize]{figures/barker13-isl.png}
\caption{The ACF of a Barker 13 code. The ISL describes the ratio between the area under the sidelobes (red) and the area under the
mainlobe (blue).}
\label{fig:barker13-isl}
\end{figure}

Lower ISL means better performance of the code. In this work, the ISL is calculated across the whole AF, but limited in
the Doppler dimension for values $v \in \left[-50, +50\right]\ \si{\metre\per\second}$ as the main focus is weather
applications and encountering targets that move faster than this is rare. 

\subsection{Peak Sidelobe Level}
The Peak Sidelobe Level (PSL) is the ratio between
the highest sidelobe and the mainlobe in the plane of the AF and complements the ISL. It is given by
\citep{Keeler:1999}:

\begin{equation}
\label{PSL1}
\text{PSL} = 20 \log \left(\frac{\max(s_j)}{\max(m_k)}\right).
\end{equation}
Although the ISL is a good way to quantify waveform performance, it can be misleading. Sidelobes are very undesirable,
so a waveform that has overall low sidelobes except for one very dominant one may have a low ISL, but may not be
performing well enough for actual use due to the outlier sidelobe. 

\subsection{Pulse Power Ratio}
For amplitude tapering, another factor needs to be taken into consideration, called the Pulse Power Ratio (PPR). For a
rectangular pulse, \SI{100}{\percent} of the signal power is transmitted. For an amplitude-tapered pulse, however, that
is not the case. The PPR is defined as follows:


\begin{equation}
\label{PPR}
\mathrm{PPR} = 10 \log \left(\frac{\int_{\tau} w^2(t)dt}{A_\mathrm{p}^2 \tau}\right), 
\end{equation}
where $w(t)$ is the tapering function applied to the pulse, integrated over the pulse length $\tau$, and $A_\mathrm{p}$ is the
original amplitude.  This metric indicates how much power is lost when amplitude tapering is applied compared to a
rectangular pulse with equal pulse length, and can help quantify the loss in sensitivity caused by tapering.


\section{Limitations}
As mentioned before, the core of pulse compression is the matched filtering. Any kind of moving target, including weather,
will introduce distortions to the waveform that cause a mismatch between the filter and the waveform. Depending on the
waveform, this mismatch can be quite severe and greatly impact the performance of the weather radar. 
Besides weather, other sources of distortion limit the performance of the waveform. In this section, another limitation
that will be examined is the effect of a system impulse response on the waveform.


\subsection{Doppler Effect for Weather Radars}
\label{s:dopplereffect}
One concern for pulse compression performance is the effect of a Doppler shift on the matched filter in the receiver. A
received signal that has been affected by a Doppler shift causes a mismatch in the matched filter processing of the
receiver.  This is especially pronounced at high target velocities and is dependent on the used waveform.

This mismatch effect for a pulse waveform on the matched filter can be quantified as follows. Referring back to Equation
\eqref{eq:pulseaf}, consider the Doppler dimension only, i.e. $\tau_\mathrm{d} = 0$. This is the familiar
$\mathrm{sinc}$ function:

\begin{equation}
\label{eq:pulseafsinc}
  |\chi(0,f_\mathrm{d})| = \left| \frac{\sin(\pi f_\mathrm{d} \tau ) }{\pi
  f_\mathrm{d} \tau} \right| \text{, for } -\tau \leq \tau_\mathrm{d} \leq \tau.
\end{equation}
This function has a value of $0$ exactly when $f_\mathrm{d}$ is a multiple of $1/\tau$, as shown in Figure
\ref{fig:sincnulls}. Thus, if a Doppler frequency is much less than the inverse of the pulse length, i.e., $f_\mathrm{d}
\ll 1/\tau$, its effect on the waveform can be neglected.

\begin{figure}[h!]
\includegraphics[width=\imgsize]{figures/sincnulls.png}
\caption{The $\mathrm{sinc}$ function describes the AF of a pulse in the Doppler dimension. It is $0$ when
$f_\mathrm{d}$ is a multiple of $1/\tau$.}
\label{fig:sincnulls}
\end{figure}

For atmospheric weather radars, such as the WSR-88D with a wavelength of \SI{10}{\centi\metre}, and a target velocity of \SI{50}{\metre\per\second}, a Doppler frequency of $|f_\mathrm{d}| =
\left|\frac{2v}{\lambda}\right| = \frac{100}{0.1} = \SI{1}{\kilo\hertz}$ is observed. That is very low compared to the first null occurring
at $1/\tau$, where $\tau=\SI{4.57}{\micro\second}$, the long pulse of a WSR-88D radar \citep[p.  47]{Doviak:2006}. In
this case, $1/\tau = 1/\SI{4.57}{\micro\second} = \SI{218.82}{\kilo\hertz}$. Thus a sufficiently short pulse is very
Doppler tolerant, although that Doppler tolerance decreases with increased pulse length, as the optimization
calculations will show.



\subsection{Effects of Amplifier Impulse Response on the Ambiguity Function}
\label{s:ireffects}

Next, the effects of the components in the transmit and receive chains on the pulse will be examined. For
simplification, these
components, excluding the matched filter, are represented by some equivalent system with
impulse response $g(t)$, in the case of the receive chain, and $g'(t)$, in the case of the transmit chain. A block
diagram is shown in Figure \ref{fig:distortblock}.  Starting with Equation \eqref{eq:AF} describing the
AF, the signal is convolved with $g(t)$, the impulse response of the system. The AF then becomes:

\begin{align*}
\label{eq:afwithir}
   \chi(\tau_\mathrm{d},f_\mathrm{d}) &= \left|\int_{-\infty}^\infty{\left[(s(t) \exp(-j2\pi
   f_\mathrm{d} t)) * g(t)\right] s^*(t+\tau_\mathrm{d}) \ dt}\right|
\end{align*}
The results for various waveforms are presented in Section \ref{s:ireffectsopti}.

\begin{figure}[h!]
\includegraphics[width=\imgsize]{figures/distortblock.png}
\caption{Block diagram showing the radar TX and RX chains. The components are represented by equivalent systems with
impulse responses $g(t)$ for the RX chain and $g'(t)$ for the TX chain. The matched filter impulse response is denoted
as $h(t)$.}
\label{fig:distortblock}
\end{figure}

