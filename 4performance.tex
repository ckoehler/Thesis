This chapter examines the different approaches to quantifying waveform performance in the context of weather radar
applications, as well as associated limitations.

\section{Quantifications}
\label{s:quantifications}
In order to evaluate the pulse compression techniques examined in this work, several different metrics are used to
quantify their performance. Together they can assist in creating and evaluating waveforms that perform optimally for
weather radars.

\subsection{3 dB Range Resolution}
The \SI{3}{\decibel} range resolution, given in units of meters, is the range resolution of the mainlobe
\SI{3}{\decibel} down from the peak.  Lower value means finer resolution of the waveform. The factors that influence
range resolution are mainly pulse length and bandwidth, as described in Equations \eqref{eq:rangeres} and
\eqref{eq:rangeresbwpulse}. For pulse length, the shorter the pulse, the better the resolution; for bandwidth, the
higher the bandwidth the better the resolution. Amplitude tapering or NLFM can also change the resolution because they
affect the bandwidth, as explained in Section \ref{s:taperingandfm}.


\subsection{Integrated Sidelobe Level}
The performance of each code is measured by calculating the Integrated Sidelobe Level (ISL), which is defined as
\citep{Keeler:1999}:

\begin{equation}
\label{ISL}
\text{ISL} = 10 \log \left(\frac{\sum\limits_{j} |s_j|^2}{\sum\limits_{k} |m_k|^2}\right)
\end{equation}
where $s_j$ is the amplitude of the $j^{th}$ sidelobe, and $m_k$ the amplitude of the $k^\mathrm{th}$ mainlobe.

TODO: plot of ISL.
Lower ISL means better performance of the code. In this work, the ISL is calculated across the whole AF, but limited in
the Doppler dimension for values $v \in \left[-50, +50\right]\ \si{\metre\per\second}$ as the main focus is weather
application and encountering targets that move faster than this is rare. 

\subsection{Peak Sidelobe Level}
The Peak Sidelobe Level (PSL) is the ratio between
the highest sidelobe and the mainlobe in the plane of the AF and complements the ISL. It is given by
\citep{Keeler:1999}:

\begin{equation}
\label{PSL1}
\text{PSL} = 20 \log \left(\frac{\max(s_j)}{\max(m_k)}\right).
\end{equation}
Although the ISL is a good way to quantify waveform performance, it can be misleading. Sidelobes are very undesirable,
so a waveform that has overall low sidelobes except for one very dominant one may have a low ISL, but may not be
performing well enough for actual use due to the outlier sidelobe. 

\subsection{Pulse Power Ratio}
For amplitude tapering, another factor needs to be taken into consideration, called the Pulse Power Ratio (PPR). For a
rectangular pulse, \SI{100}{\percent} of the signal power is transmitted. For an amplitude-tapered pulse, however, that
is not the case. The PPR is defined as follows:


\begin{equation}
\label{PPR}
\mathrm{PPR} = 10 \log \left(\frac{\int_{\tau} w^2(t)dt}{A_\mathrm{p}^2 \tau}\right), 
\end{equation}
where $w(t)$ is the tapering function applied to the pulse, integrated over the pulse length $\tau$, and $A_\mathrm{p}$ is the
original amplitude.  This metric indicates how much energy is lost when amplitude tapering is applied compared to a
rectangular pulse with equal pulse length, and can help quantify the loss in sensitivity caused by tapering.


\section{Limitations}
As mentioned before, the core of pulse compression is the match filtering. Any kind of moving target, including weather,
will introduce distortions to the waveform that cause a mismatch between the filter and the waveform. Depending on the
waveform, this mismatch can be quite severe and greatly impact the performance of the weather radar. 
Besides weather, other sources of distortion limit the performance of the waveform. In this section, another limitation
that will be examined is the effect of a system impulse response on the waveform.


\subsection{Doppler Effect for Weather Radars}
\label{s:dopplereffect}
One concern for pulse compression performance is the effect of a Doppler shift on the matched filter in the receiver. A
received signal that has been affected by a Doppler shift causes a mismatch in the matched filter processing of the
receiver.  This is especially pronounced at high target velocities. For atmospheric weather radars, such as the WSR-88D,
the aliasing velocity is \SI{32.5}{\metre\per\second}, based on $\lambda = \SI{0.1}{\metre}$ and a maximum
PRT $T_\mathrm{s} = \SI{1300}{\hertz}$, not taking into account any signal processing techniques described previously \citep[p.
47]{Doviak:2006}.  At that velocity, a Doppler frequency of $f_\mathrm{d} = \frac{2v}{\lambda} = \frac{65}{0.1} = \SI{650}{Hz}$
is observed, which is very low compared to the carrier frequency of \SI{3000}{\mega\hertz}.

This mismatch effect on the matched filter has been quantified in \cite{Kelly:1965} as follows. For
a signal of length $\tau$, a target of velocity $v$, and signal bandwidth $B$, the return Doppler-shifted signal can be
expressed as 
\begin{equation}
  \tau(1-\frac{2v}{c})=\tau-\Delta \tau
\end{equation}
If $\Delta \tau$ is much smaller than $\tau$, i.e.,  $\Delta \tau \ll \frac{1}{B}$, the mismatch due to the Doppler effect can be
neglected. In other words, $\frac{2v}{c}B\tau \ll 1$. For a \SI{1}{\micro\second} pulse at \SI{5}{\mega\hertz} bandwidth
moving at \SI{30}{\metre\per\second}, $0.001 \ll 1$ is true, so the Doppler effect can be ignored.  The higher the pulse
compression ratio $B\tau$ is, the more mismatch will be caused by the Doppler shift.


\subsection{Effects of Amplifier Impulse Response on the Ambiguity Function}
\label{s:ireffects}

Next, the effects of the amplifier in the receive chain on the pulse will be examined.  Starting with Equation
\eqref{eq:AF} describing the AF, the return signal is convolved with $h(t)$, the impulse response of the system. The AF
then becomes:

\begin{align*}
\label{eq:afwithir}
   \chi(\tau_\mathrm{d},f_\mathrm{d}) &= \left|\int_{-\infty}^\infty{s(t) \left[(s^*(t+\tau_\mathrm{d}) e^{-j2\pi
   f_\mathrm{d} t}) \star h(t)\right]\ dt}\right| \\
   &= \left|\int_{-\infty}^\infty{s(t) \int_{-\infty}^{\infty}{s^*(t'+\tau_\mathrm{d}) e^{-j2\pi f_\mathrm{d} t'}h^*(t'-t) dt'}\ dt}\right| \\
   &= \left|\int_{-\infty}^\infty{s^*(t'+\tau_\mathrm{d}) e^{-j2\pi f_\mathrm{d} t'}\int_{-\infty}^{\infty}{s(t)
   h^*(t'-t) dt}\ dt'}\right| \\
   &= \left|\int_{-\infty}^\infty{[s(t') \star h(t')] s^*(t'+\tau_\mathrm{d}) e^{-j2\pi f_\mathrm{d} t'} \ dt'}\right| 
\end{align*}

It can be seen that applying the impulse response to the return signal is equivalent to applying it to the original
pulse $s(t)$.  This makes it easy in simulation to apply an impulse response to the calculations in order to evaluate
its effects on the AF.  The results of a Rayleigh-shaped impulse response on a rectangular pulse are presented in
Section \ref{s:ireffectsopti}.

