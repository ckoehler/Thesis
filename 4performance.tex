This chapter examines the different approaches to quantifying waveform performance in the context of
weather radar applications, as well as limitations associated with them.

\section{Quantifications}
\label{s:quantifications}

In order to evaluate the pulse compression techniques examined in this work, several different
metrics are used to quantify their performance. Together they can assist in creating waveforms that
perform optimally for weather radars.

\subsection{Ambiguity Function}
\label{s:af}
The main metric to determine waveform performance is the Ambiguity Function (AF) \citep{Levanon:2004}. It is defined as:

\begin{equation}
\label{eq:AF}
  |\chi(\tau,f_d)| = \left| \int_{-\infty}^\infty{u(t) u^*(t+\tau) e^{-j2\pi f_d t} dt}\right|,
\end{equation}

where $u(t)$ is the complex envelope of the signal, $\tau$ is the delay due to propagation, and $f_d$
is the Doppler frequency in the Doppler axis.
We can clearly see that, for $f_d = 0$, the AF reduces to the Autocorrelation Function (ACF).


For a simple rectangular pulse of the form $u(t)=\frac{1}{\sqrt{T}} rect(\frac{t}{T})$, the AF is \citep{Levanon:2004}:

\begin{align*}
  |\chi(\tau,f_d)| &= \left| \int_{-\infty}^\infty{u(t) u^*(t+\tau) e^{j2\pi f_d t} dt}\right| \\ 
  &= 
  \begin{cases}
    \frac{1}{T} \int_{-\frac{T}{2} + \tau}^{\frac{T}{2}}{e^{j2\pi f_d t}dt} & \text{for $0 \leq \tau \leq T$}, \\
    \frac{1}{T} \int_{-\frac{T}{2}}^{\frac{T}{2} + \tau}{e^{j2\pi f_d t}dt} & \text{for $-T \leq \tau < 0$}, \\
    0 & \text{otherwise}. \\
  \end{cases}
\end{align*}

Evaluate the integral to get:

\begin{equation}
  |\chi(\tau,v)| = \left|\left(1-\frac{|\tau|}{T}\right) \frac{\sin[\pi T v(1- \frac{|\tau|}{T})]}{\pi T v(1- \frac{|\tau|}{T})}\right|, |\tau| \leq T, \text{zero elsewhere}
\end{equation}

For $\tau=0$, the result is the sinc function, for $v=0$ it is simply the autocorrelation function of a pulse, which is a triangle.


The ideal AF shape depends on the application. One useful AF looks like a thumbtack and provides good range and Doppler resolution.
However, this also means that any kind of Doppler shift will result in a weak or, for the ideal case, no return signal. This waveform
can be used to detect targets whose velocity is already known, or is in a known, narrow range, as is the case for weather applications.
Thumbtack shaped AFs can be used for different Doppler shifts by using several different filters, each matched to a different Doppler shift.
The reader is referred to Chapter 7.2 of \citep{Levanon:2004} for more information.


\subsection{3dB Range Resolution}
Another metric to evaluate the performance of a waveform is the 3dB range resolution in units of meters.
Lower value means finer resolution of the waveform. The factors that influence range resolution are mainly pulse
length and amplitude tapering. For pulse length, the shorter the pulse, the better the resolution. 
Amplitude tapering or non-linear frequency modulation will also change the resolution.



\subsection{Integrated Sidelobe Level}
The performance of each code is measured by calculating the Integrated Sidelobe Level (ISL), which is defined as (\cite{Keeler:1999}):

\begin{equation}
\label{ISL}
\text{ISL} = 10 \log \left(\frac{\sum\limits_{j} |s_j|^2}{\sum\limits_{k} |m_k|^2}\right)
\end{equation}
where $s_j$ is the power of the $j^{th}$ sidelobe, and $m_k$ the power of the $k^{th}$ mainlobe.

Lower ISL means better performance of the code. In this work, the ISL is only calculated with $v \in \left[-50, +50\right]\ \si{\metre\per\second}$
as the main focus is weather application.


\subsection{Maximum Sidelobe Level}
Although the ISL is a good way to quantify waveform performance, it can be misleading. Sidelobes are very undesirable, so a waveform
that has overall low sidelobes except for one very dominant one will have a low ISL, but may not be performing well enough for actual use due
to the outlier sidelobe.

This metric determines the highest sidelobe level in the plane of the ambiguity function so it complements the ISL.


\subsection{Pulse Power Ratio}
For amplitude tapering, another factor needs to be taken into consideration, i.e. Pulse Power Ratio. For a rectangular pulse, 100\% of the signal
power is transmitted. For a amplitude-tapered pulse, however, that is not the case. The Pulse Power Ratio is defined as follows:


\begin{equation}
\label{PPR}
\text{PPR} = \frac{\int_{\tau} w(t)dt}{A_p \tau}, 
\end{equation}
where $w(t)$ is the tapering function applied to the pulse, integrated over the pulse length $\tau$, and $A_p$ is the original amplitude.

This metric represents the fraction of energy that is utilized when amplitude tapering is applied.


\section{Limitations}
There are a some limitations that need to be taken into consideration, e.g. whether 
waveforms used in weather radar need to be very Doppler tolerant. This is highly specific to weather radars and can simplify the task 
of waveform design. Another limitation is the effect of the amplifier impulse response on the signal.


\subsection{Doppler Effect for Weather Radars}
\label{s:dopplereffect}
One concern for pulse compression performance is the effect of a Doppler shift on matched filter in the receiver. A
received signal that has been affected by a Doppler shift causes a mismatch in the matched filter processing of the receiver.
This is especially pronounced at high target velocities. For atmospheric weather radars, such as the WSR-88D, the maximum
unambiguous velocity is \SI{32.5}{\metre\per\second}, based on $\lambda = \SI{0.1}{\metre}$ and a maximum PRT $T_s = \SI{1300}{\hertz}$,
not taking into account any signal processing techniques described previously \citep[p. 47]{Doviak:2006}.
At that velocity, a Doppler shift of $f_d = \frac{2v}{\lambda} = \frac{65}{0.1} = \SI{650}{Hz}$ is
observed.

This mismatch effect on the matched filter has been quantified by Kelly and Wishner as follows \citep{Kelly:1965}.

For a signal of length $T$, a target of velocity $v$, and signal bandwidth $B$, the return Doppler-shifted signal can be expressed 
as 
\begin{equation}
  T(1-\frac{2v}{c})=T-\Delta T
\end{equation}
If $\Delta T$ is much smaller than $T$, i.e.  $\Delta T \ll \frac{1}{B}$, the mismatch 
due to the Doppler effect can be neglected. In other words, $\frac{2v}{c}BT \ll 1$. For a \SI{1}{\micro\second} pulse at \SI{5}{\mega\hertz} bandwidth 
moving at \SI{30}{\metre\per\second}, $0.001 \ll 1$ is true, so the Doppler effect can be ignored. This will be confirmed later in the 
simulations.

The pulse compression ratio, $BT$, is not equal to unity, so the mismatch in the filter is compounded for highly compressed 
pulses.

\subsection{Effects of Amplifier Impulse Response on the Ambiguity Function}
\label{s:ireffects}

Next, the effects of the amplifier in the receive chain on the pulse will be examined.
Starting with Equation \eqref{eq:AF}, substitute the return signal with $h(t) \star u(t-\frac{2r}{c})$, where $h(t)$ is the impulse
response of the system. The AF then becomes

\begin{align*}
   \chi(\tau,f_d) &= \left|\int_{-\infty}^\infty{u(t) \left[(u^*(t+\tau) e^{-j2\pi f_d t})*h(t)\right]\ dt}\right| \\
   &= \left|\int_{-\infty}^\infty{u(t) \int_{-\infty}^{\infty}{u^*(t'+\tau) e^{-j2\pi f_d t'}h^*(t'-t) dt'}\ dt}\right| \\
   &= \left|\int_{-\infty}^\infty{u^*(t'+\tau) e^{-j2\pi f_d t'}\int_{-\infty}^{\infty}{u(t) h^*(t'-t) dt}\ dt'}\right| 
\end{align*}

We can easily see that applying the impulse response to the return signal is equivalent to applying it to the original pulse $u(t)$. 

\subsection{Quantization}
Once the signal is received through the antenna, it is eventually converted into a digital form for further processing.
The first step in that process is to sample the signal in time. The second is called quantization, in which the
amplitude of each sample is converted to one of a finite possible values for the amplitude \citep{Richards:2005}.

For $b$ available bits there are $2^b$ possible values to represent the amplitude of the signal. These $b$ bits need to be able
to contain the whole dynamic range of possible return signals. For the WSR-88D network radars, the dynamic range at the receiver
is \SI{93}{\db}. However, that means that weaker return signals, e.g. rain at 15 dBZ, only get about 20 \%
of the available bits, so there must be a sufficient number of bits to adequately quantize any possible return signal.
For every additional bit, the dynamic range increases about 6 dB, so about 14 bits are needed to represent our whole
dynamic range (\cite{Richards:2005}).

\subsection{Volume Scattering}
