\section{Generating Waveforms}
The simulation is done in multiple parts. First, a waveform is generated based on a given
amplitude, phase, frequency modulation, and impulse response that is applied to the signal.
The returned waveform is properly sampled based on the desired signal bandwidth to avoid aliasing.

If an impulse response was supplied, we get both the distorted and clean waveform back, as well as
the new pulse length $\tau$ which will be approximately twice the original $\tau$. That lengthening
is due to the convolution of the original waveform $u(t)$ with the impulse response $h(t)$.

\section{Calculate the Ambiguity Function}
Once we have the waveform, we generate the AF and plot it as a surface plot. There are three components
to the AF: 

\begin{enumerate}
\item the original waveform, $u(t)$
\item the shifted complex conjugate of the waveform, $u^*(t+\tau)$
\item the Doppler shift, $e^{j2\pi f_d t}$
\end{enumerate}
These components are multiplied and the result summed up. To do that, we convert $u(t)$ into a diagonal
matrix, so that each data point of the signal is on its own row. The shifted complex conjugate is also 
a matrix, where each row contains the flipped signal shifted a little further. If we ignore the Doppler
shift for a moment and multiply what we have so far and sum over it, we get the auto-correlation of $u(t)$,
which is easy to verify.

Next we calculate the Doppler shifts for a given range of frequencies over the time $t$. The multiplication of 
$f_d$ and $t$ results in a matrix where each row corresponds to a certain Doppler shift. The first row at 
$f_d = 0$ gives us a vector of 1s, as we would expect.

Finally, we multiply all three components together and sum to get the AF in a 2D matrix that we can visualize
as a surface plot.
Note that the original signal $u(t)$ could be distorted by an arbitrary impulse response if we decide to include
that aspect in the simulation, whereas the shifted complex conjugate part is always based on the clean, original
signal.

\section{Optimizations}
We optimize the waveform on Integrated Sidelobe Level (ISL) and peak sidelobe.

- fixed, motion, volume scattering, transfer function, quantization effects
- nlfm, paramterized as polynomial



