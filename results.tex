\section{Simple pulse}
A pulse is the simplest waveform for a radar. These graphs show the AF for pulses of varying lengths.

\begin{figure}[H]
\includegraphics[width=\imgsize]{figures/pulse-1us.png}
\caption{ Simple pulse of length 1 $\mu s$ }
\label{fig:simplepulse-1us}
\end{figure}

\begin{figure}[H]
\includegraphics[width=\imgsize]{figures/pulse-15us.png}
\caption{ Simple pulse of length 15 $\mu s$ }
\label{fig:simplepulse-15us}
\end{figure}

\begin{figure}[H]
\includegraphics[width=\imgsize]{figures/pulse-50us.png}
\caption{ Simple pulse of length 50 $\mu s$ }
\label{fig:simplepulse-50us}
\end{figure}

\begin{figure}[H]
\includegraphics[width=\imgsize]{figures/pulse-100us.png}
\caption{ Simple pulse of length 100 $\mu s$ }
\label{fig:simplepulse-100us}
\end{figure}

\begin{figure}[H]
\includegraphics[width=\imgsize]{figures/pulse-200us.png}
\caption{ Simple pulse of length 200 $\mu s$ }
\label{fig:simplepulse-200us}
\end{figure}


\section{Phase-coded Barker code}
Next we will examine a length 13 Barker code of different pulse lengths. Again we notice that
Doppler performance gets worse with a longer pulse.

\begin{figure}[H]
\includegraphics[width=\imgsize]{figures/barker-1us.png}
\caption{ Barker 13 of length 1 $\mu s$ }
\label{fig:barker13-1us}
\end{figure}

\begin{figure}[H]
\includegraphics[width=\imgsize]{figures/barker-15us.png}
\caption{ Barker 13 of length 15 $\mu s$ }
\label{fig:barker13-15us}
\end{figure}

\begin{figure}[H]
\includegraphics[width=\imgsize]{figures/barker-50us.png}
\caption{ Barker 13 of length 50 $\mu s$ }
\label{fig:barker13-50us}
\end{figure}

\begin{figure}[H]
\includegraphics[width=\imgsize]{figures/barker-100us.png}
\caption{ Barker 13 of length 100 $\mu s$ }
\label{fig:barker13-100us}
\end{figure}

\begin{figure}[H]
\includegraphics[width=\imgsize]{figures/barker-200us.png}
\caption{ Barker 13 of length 200 $\mu s$ }
\label{fig:barker13-200us}
\end{figure}


\section{Linear Frequency Modulated waveform}
Here we will look at a linearly frequency modulated waveform. As mentioned before, LFM is a property of the
AF and thus very simple simple to implement in analog hardware. Also, the figures below show that the Doppler
performance of an LFM waveform is very high, especially for our weather radar application. There is no visible
difference between the AFs in the Doppler dimension.

\begin{figure}[H]
\includegraphics[width=\imgsize]{figures/lfm-15us.png}
\caption{ LFM of length 15 $\mu s$ }
\label{fig:lfm-15us}
\end{figure}

\begin{figure}[H]
\includegraphics[width=\imgsize]{figures/lfm-50us.png}
\caption{ LFM of length 50 $\mu s$ }
\label{fig:lfm-50us}
\end{figure}

\begin{figure}[H]
\includegraphics[width=\imgsize]{figures/lfm-100us.png}
\caption{ LFM of length 100 $\mu s$ }
\label{fig:lfm-100us}
\end{figure}

\begin{figure}[H]
\includegraphics[width=\imgsize]{figures/lfm-200us.png}
\caption{ LFM of length 200 $\mu s$ }
\label{fig:lfm-200us}
\end{figure}

\section{Quadratic Frequency Modulated waveform}
Here we will look at a quadradicly frequency modulated waveform. Also, the figures below show that the Doppler
performance of a QFM waveform is very high, especially for our weather radar application. There is no visible
difference between the AFs in the Doppler dimension.

\begin{figure}[H]
\includegraphics[width=\imgsize]{figures/qfm-15us.png}
\caption{ QFM of length 15 $\mu s$ }
\label{fig:qfm-15us}
\end{figure}

\begin{figure}[H]
\includegraphics[width=\imgsize]{figures/qfm-50us.png}
\caption{ QFM of length 50 $\mu s$ }
\label{fig:qfm-50us}
\end{figure}

\begin{figure}[H]
\includegraphics[width=\imgsize]{figures/qfm-100us.png}
\caption{ QFM of length 100 $\mu s$ }
\label{fig:qfm-100us}
\end{figure}

\begin{figure}[H]
\includegraphics[width=\imgsize]{figures/qfm-200us.png}
\caption{ QFM of length 200 $\mu s$ }
\label{fig:qfm-200us}
\end{figure}
