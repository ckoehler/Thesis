\section{Conclusions}
This work examined several different techniques of pulse compression and evaluated their performance 
in regards to weather radar applications. 

A rectangular pulse still performs fairly well, with good range resolution for very short pulses and no sidelobes.  For
pulse lengths of \SI{200}{\micro\second} there is some significant mismatch loss of approximately \SI{-7}{\decibel} over the
velocities commonly encountered in weather applications. As expected, the range resolution for the longer pulse decreases as well.

Using a waveform that is phase coded with a Barker 13 code does increase the range resolution dramatically for the same
pulse lengths to approximately \SI{110}{\metre}. However, the peak sidelobe level is only at approximately
\SI{-22}{\decibel}, which makes this waveform unusable for typical weather observation. It could still be used in
situations where a low PSL is not necessary, like clear air measurements. There is no Doppler mismatch for long
pulses and high Doppler velocities, and the mainlobe level is also not affected, unlike for a rectangular pulse.

An LFM waveform has much better range resolution than both a rectangular and Barker phase coded pulse at approximately
\SI{30}{\metre}.  The PSL is higher at \SI{-13}{\decibel}, so the LFM waveform has the same drawbacks as the Barker
code, being not very usable for typical weather applications. Sidelobe levels improved significantly with amplitude
tapering, with range resolution decreasing only moderately. Sensitivity could be the limiting factor because the PPR
decreases quickly with more aggressive tapering. With PPR at \SI{-3}{\decibel}, the ISL decreases to approximately
\SI{-36}{\decibel} with the PSL decreasing to \SI{-47}{\decibel}. The range resolution with this degree of tapering is
\SI{68}{\metre}, which would make this waveform adequate for weather radar applications.

Optimizing the frequency sweep for NLFM waveforms showed that it can improve performance even without amplitude
tapering. Simulations indicated ISLs around \SI{-20}{\decibel} to \SI{-25}{\decibel} and PSLs around \SI{-20}{\decibel}
to \SI{-40}{\decibel} while maintaining a range resolution of approximately \SI{100 }{\metre}. The PPR is \SI{0}{\decibel} because no
tapering is applied yet, so while a tapered LFM waveform has better performance than NLFM, there is also a loss in
sensitivity due to the tapering.

Applying amplitude tapering to the NLFM reduces the ISL and PSL significantly, as it did for LFM waveforms. The decrease
in sensitivity due to lower PPR has to be taken into account, as well as the loss in range resolution, which decreases
rapidly with increased tapering and larger $a$. For $a=2.75$ and $\beta=12.31$, the range resolution is approximately
\SI{280}{\metre},
with PSL at \SI{-70}{\decibel}. However, the loss of sensitivity due to the PPR is unacceptable at approximately
\SI{-6}{\decibel}.

The most significant result of taking the system impulse response into account is the asymmetric distortion of the AF in
the range delay dimension, which should be corrected if possible. The ideal course of action would be to measure the
actual system impulse response and repeat the simulations with it, as well as test the results in a real system.




\section{Future Work}
There are several ways to build upon on this work in the future.

First, other waveforms can be evaluated based on the presented metrics, such as more complex phase coded signals. Other
sweeping functions for the NLFM waveform could be evaluated to perhaps find waveforms that perform better for weather
radar applications. 

Second, the effects of the amplifier impulse response and other system components could be better quantified and the
results compared with the theoretical results that are expected. Particularly, attempts to correct for the distortion
caused by the impulse response could be presented. One possibility that has briefly been mentioned in Section
\ref{s:ireffectsopti} is to pre-distort the waveform on transmit to cancel out the effects caused by the receive chain.
The anti-distortion filter is designed to equalize the system response of the subsequent chain in between the waveform
generator and antenna feed horn, which includes up-converter and amplifier.

Third, the effects of volume scattering on the waveform should be examined. For weather radar applications, volumes of
point scatterers are considered instead of simple point targets. In the case of rain, millions of raindrops will be
present in a typical resolution volume of the radar, moving at different velocities and with different shapes. Both
their size and velocity will be distributed over some probability density function, introducing a random element to the
calculations. How exactly that affects the AF would be beneficial to determine. Additionally, the exact effects of
averaging the measurements from an up- and a down-chirp to determine range accurately in a distributed scatterer
environment should be examined.

Fourth, the connection between amplitude tapering and frequency modulation should be precisely quantified. Briefly
explained, tapering the amplitude of a signal to reduce range sidelobes delivers a similar result compared to simply
sweeping the frequency non-linearly. This relationship can be explained by considering the ACF of the signal. The
sidelobes appear in the ACF, which is the inverse Fourier transform of the power spectrum. Thus modifying the power
spectrum, e.g., by amplitude tapering, will likewise change the shape of the ACF.  From this follows that alternatively,
the spectrum can be shaped by varying the frequency sweep non-linearly to spend more time at frequencies that should be
enhanced \citep[p.  87]{Levanon:2004}. 

\begin{figure}[h]
\includegraphics[width=\imgsize]{figures/nlfmbwspectrum.png}
\caption{Frequency spectrum comparison of NLFM and amplitude tapering. It shows that applying a window to the amplitude
has a similar effect to chirping the frequency of the pulse non-linearly.}
\label{fig:nlfmbwspectrum}
\end{figure}
Figure \ref{fig:nlfmbwspectrum} shows the results of a NLFM pulse and a tapered LFM pulse. An untapered LFM pulse is
given for comparison. Both NLFM and tapering change the shape of the mainlobe and sidelobes in a similar way, so both
must be considered in waveform design. Describing the exact relationship between the two would be useful.


