\section{Conclusions}
This work examined several different techniques of pulse compression and evaluated their performance 
in regards to weather radar applications. 

A rectangular pulse still performs fairly well, with good range resolution and no sidelobes. Although much
effort was put into finding optimal NLFM waveforms, the performance evaluations suggest that amplitude tapered LFM
waveform is desirable in terms of sidelobes around \SI{-70}{\decibel}, with the only drawback being a \SI{3}{\decibel} loss in energy.


\section{Future Work}
There are several ways to improve on this work in the future.

First, other waveform can be evaluated based on the presented metrics, like more complex phase coded signals. Other
sweeping functions for the NLFM waveform could be evaluated to perhaps find waveforms that perform better for weather
radar applications. 

Second, the effects of the amplifier impulse response and other system components could be better quantified and the
results compared with the theoretical results that are expected. Particularly, attempts to correct for the distortion
caused by the impulse response could be presented. One possibility is to pre-undo the distortions that are caused by the
transmit and receive chain. The anti-distortion filter is designed to equalize the system response of the subsequent
chain in between the waveform generator and antenna feed horn, which includes up-converter and amplifier.


