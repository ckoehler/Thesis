\section{Conclusions}
This work examined several different techniques of pulse compression and evaluated their performance 
in regards to weather radar applications. 

A rectangular pulse still performs fairly well, with good range resolution for very short pulses and no sidelobes.  For
pulse lengths of \SI{200}{\micro\second} there is some significant mismatch loss of about \SI{-6}{\decibel} over the
velocities commonly encountered in weather applications. As expected, the range resolution decreases as well.

Using a waveform that is phase coded with a Barker 13 code does increase the range resolution dramatically for the same
pulse lengths to about \SI{110}{\metre} for the same pulse length. However, the peak sidelobe level is only at about
\SI{-22}{\decibel}, which makes this waveform unusable for typical weather observation. It could still be used in
situations where a low PSL is not necessary, like clear air measurements. There is also some Doppler mismatch for long
pulses and high Doppler velocities, although the mainlobe level is not affected, unlike for a rectangular pulse.
Sidelobes are distorted and the PSL is raised to about \SI{-13}{\decibel}.

Applying amplitude tapering to a Barker phase coded waveform has little benefit. The range resolution does not improve,
and neither to ISL and PSL. Since a Barker 13 code is already ``perfect,'' this result is to be expected.
TODO: check first. Need to taper sub pulse.

An LFM waveform has much better range resolution than both a rectangular and Barker phase coded pulse at about
\SI{30}{\metre}.
The PSL is higher at \SI{-13}{\decibel}, so the same discussion for the Barker code applies here. Sidelobe levels
improved significantly with amplitude tapering, with range resolution decreasing only moderately. Sensitivity could be
the limiting factor because the PPR decreases quickly with more aggressive tapering. With PPR at \SI{-3}{\decibel}, the
ISL decreases to about \SI{-36}{\decibel} with the PSL decreasing to \SI{-47}{\decibel}. The range resolution with this
degree of tapering is \SI{68}{\metre}, which would make this waveform adequate for weather radar applications.

Optimizing the frequency sweep for NLFM waveforms showed that it can improve performance even without amplitude
tapering. Simulations indicated ISLs around \SI{-20}{\decibel} to \SI{-25}{\decibel} and PSLs around \SI{-20}{\decibel}
to \SI{-40}{\decibel} while maintaining a range resolution of about \SI{100 }{\metre}. The PPR is \SI{0}{\decibel} because no
tapering is applied yet, so while a tapered LFM waveform has better performance than NLFM, there is also a loss in
sensitivity due to the tapering.

Applying amplitude tapering to the NLFM reduces the ISL and PSL significantly, as it did for LFM waveforms. The decrease
in sensitivity due to lower PPR has to be taken into account, as well as the loss in range resolution, which decreases
rapidly with increased tapering and larger $a$. For $a=2.75$ and $\beta=12.31$, the range resolution is about
\SI{280}{\metre},
with PSL at \SI{-70}{\decibel}. However, the loss of sensitivity due to the PPR is unacceptable at about
\SI{-6}{\decibel}.

The most significant result of taking the system impulse response into account is the asymmetric distortion of the AF in
the range delay dimension. If possible, this effect should be corrected. As mentioned before, a couple of assumptions
have been made in this simulation which will also change the result. The ideal course of action would be to measure the
actual system impulse response and repeat the simulations with it, as well as testing the results in a real system.




\section{Future Work}
There are several ways to build upon on this work in the future.

First, other waveforms can be evaluated based on the presented metrics, like more complex phase coded signals. Other
sweeping functions for the NLFM waveform could be evaluated to perhaps find waveforms that perform better for weather
radar applications. 

Second, the effects of the amplifier impulse response and other system components could be better quantified and the
results compared with the theoretical results that are expected. Particularly, attempts to correct for the distortion
caused by the impulse response could be presented. One possibility is to pre-distort the distortions that are caused by the
transmit and receive chain. The anti-distortion filter is designed to equalize the system response of the subsequent
chain in between the waveform generator and antenna feed horn, which includes up-converter and amplifier.

Third, the effects of volume scattering on the waveform should be examined. For weather radar applications, volumes of
point scatterers are considered instead of simple point targets. In the case of rain, millions of raindrops will
be present in a typical resolution volume of the radar, moving at different velocities and with different shapes. Both
their size and velocity will be distributed over some probability density function, introducing a random element to the
calculations. How exactly that affects the AF would be beneficial to determine.


