\documentclass[letter, 12pt]{article}
\title{Optimal Waveform Design for Weather Radars}
\author{Christoph Koehler}
\date{}

\begin{document}
\maketitle
\thispagestyle{empty}
\begin{abstract}
Most of the literature about radar waveforms has been done in the context of application other than weather.  This
thesis will focus the discussion on pulse compression for weather radar applications. First, a general overview of using
radar for weather applications is given, followed by a brief discussion of general waveform design, in particular the
different components of a waveform that can be manipulated for its design. Next, the concepts of pulse compression will
be presented and the need for pulse compression explained.  Then, various different waveforms, like rectangular pulses,
phase coded, linear frequency modulated, and non-linear frequency modulated signals will be examined in regards to their
ambiguity function, which serves as the basis for performance metrics.  Each will be quantified in respect to integrated
sidelobe level, peak sidelobe level, and range resolution. Further, amplitude tapering will be applied to determine its
performance benefits.  For the non-linear frequency modulated waveforms, the frequency sweep function is parameterized
and optimal values for weather applications are presented. Lastly, the effects of a system impulse response on the
signal will be discussed and their performance analyzed.
\end{abstract}
\end{document}



